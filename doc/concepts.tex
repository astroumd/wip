% This file is to be included by latex in ug.tex
%
% General Task Concepts
%
\mylabel{c:concepts}
\myfile{concepts.tex}

This chapter describes the general graphical concepts of \wip.
Many of \wip's commands are used to set the device type,
the device's environment,
location and limits of the border,
and annotate the plot with labels.
Most of these commands are called only one time; others several times.
These concepts and some of the commands needed to put them to use are
described in the sections that follow.

\section{Graphical Devices}
\mylabel{s:devices}
\index{Devices}

Plotting devices in \wip\ provide the destination for the plot.
In this sense, any graphics device, whether a computer terminal,
a high resolution laser printer, or even a file
may be considered a \wip\ device.
An initial device is selected when \wip\ first begins but
new devices may be
selected at any time during the interactive session.

There are several ways that the initial device may be chosen
when starting \wip.
The following list indicates the different ways an
initial \wip\ device is selected:
\begin{enumerate}
  \item An initial device command has been set up to be
    \index{Commands!{\tt device}}\index{Devices!device-type!/xwindow}%
    {\tt device /xwindow}.
    This initial device was fixed when \wip\ was compiled.
  \item If a default device is provided in
    your \wipinit\index{wipinit@\wipinit} file
    (see Appendix~\ref{a:wipinit} on the use of this file),
    it is loaded in place of the hard coded device.
  \item If the ``{\tt -d device}'' flag appears on the command line
    (see Appendix~\ref{a:cmdline} for a further description of
    command line options),
    it will always override any other initial device selected.
\end{enumerate}

A new device may be selected at any time in \wip.
To specify a new device interactively,
use the command {\tt device}\index{Commands!{\tt device}}
with the new device name as its argument.
For example, the command
\begin{wiplist}%
  \item {\tt device /ps}
\end{wiplist}
selects the landscape version of the PostScript driver as the new device.

Devices have a standard PGPLOT format of
``file-name/device-type'' or ``device-name/device-type''.
If the `file-name' or `device-name' contains a slash (/), it should be
enclosed in double quotation marks.
For example, the command
\begin{wiplist}%
  \item {\tt device "/dev/tty"/tek}
\end{wiplist}
sets the current device to a Tektronix terminal and sends the output to
the device named ``/dev/tty''.
Usually there exists a default `file-name' or `device-name'
(see Table~\ref{t:devices}).
The part after the slash (required)
identifies the plotter device type (`device-type').
This could be,
for example, ``{\tt /tek}'' for plotting on a Tektronix terminal,
or ``{\tt /ps}'' to generate a PostScript output file.
If the {\tt device} command argument is ``?'' (no preceding slash),
then \wip\ generates a list of
all currently recognized device types.
The command
{\tt ldev}\index{Commands!{\tt ldev}}\index{Devices!commands!{\tt ldev}}
will also
generate the list of the currently recognized device types.
Table~\ref{t:devices} lists a few of the typically recognized devices
and a short description about each.
\begin{table}[htb]
  \caption{A Sample of Available Plotting Devices}
  \mylabel{t:devices}
  \index{Devices}
  \index{Devices!?}
  \index{Devices!device-type!/null}
  \index{Devices!device-type!/tek}
  \index{Devices!device-type!/retro}
  \index{Devices!device-type!/xterm}
  \index{Devices!device-type!/xwindow}
  \index{Devices!device-type!/xserve}
  \index{Devices!device-type!/ps}
  \index{Devices!device-type!/vps}
  \index{Devices!device-type!/cps}
  \index{Devices!device-type!/vcps}
  \centering
  \settowidth{\tabentrylen}{A null device.  Useful for error testing of scripts.\ }
  \begin{tabular}{|c|p{\tabentrylen}|} \hline\hline
   Device & Description \\ \hline
          & \\
 ?        & Interactive help to list all available devices. \\
 /null    & A null device.  Useful for error testing of scripts. \\
 /tek     & A Tektronix terminal. \\
 /retro   & A VT100 with RETRO graphics card. \\
 /xterm   & The Tektronix window associated with an xterm window. \\
 {\em [[id]@][display]}/xwindow
          & A X-window display.  Multiple windows may be
            started on one display by using a unique {\em id} number.
            The default {\em id} is 0, which selects an inactive window
            if available, or creates a new one.  A display may
            also be specified to override the default (which is
            to use the {\tt DISPLAY} environment variable). \\
 {\em [[id]@][display]}/xserve
          & A persistent version of the {\tt /xwindow} server.
            In addition, this device may be dynamically resized
            during various \wip\ sessions. \\
 {\em file}/ps
          & File in PostScript printer format (landscape).
            The default name for the {\em file} is pgplot.ps. \\
 {\em file}/vps
          & File in PostScript printer format (portrait).
            The default name for the {\em file} is pgplot.ps. \\
 {\em size}/cps
          & The same as {\tt /ps} except in color. \\
 {\em size}/vcps
          & The same as {\tt /vps} except in color. \\
          & \\ \hline\hline
  \end{tabular}
\end{table}

{\sc Note:}
The {\tt /retro} and {\tt /tek4010} devices may have some
problems with the cursor when used interactively
(especially over a network) -- this is currently being investigated.

{\sc Note:}
The {\tt /xwindow} and {\tt /xserve} devices behave differently than
other devices.
These plotting devices start a server independently of \wip\ which is
used to communicate with each plotting window.  The server will stay
present on the display long after a \wip\ session is complete.
The difference between the {\tt /xwindow} and {\tt /xserve} drivers is
that the {\tt /xserve} remains on the screen after a \wip\ session.
There may be multiple {\tt /xserve} windows on a single display,
each having a unique id number associated with it.
The id number is used to identify the proper window to \wip.
As with many X-window applications, the server associated with these two
devices has several application defaults which may be set up in
your X-resource file.
These resources, acceptable values, and default values are shown in
listing that follows.
To target options at specific windows, replace the {\tt Win}
component of the specification with {\tt win\#}, where {\tt \#}
is the id number of the window.
\begin{description}
  \index{Devices}
  \index{Devices!device-type!/xwindow}
  \index{Devices!device-type!/xserve}
  \item[pgxwin.Win.acceptQuit: ] Either True or False; the default is False.
    If true, it allows the window manager to provide a way to destroy
    the window; even while being used.
  \item[pgxwin.Win.iconize: ] Either True or False; the default is False.
    By default, the /xserve window remains mapped when \wip\ finishes.
    If this option is true, the window will become iconic until opened
    by another \wip\ session.
  \item[pgxwin.Win.maxColors: ] Any number; the default is 100.
    This sets the maximum number of colors to try to allocate
    on startup.  Useful for stopping ``colormap flashing''.
  \item[pgxwin.Win.minColors: ] Any number; the default is 16.
    Rejects colormaps with fewer than this number of color entries.
  \item[pgxwin.Win.visual: ] Any one of: default, monochrome, PseudoColor,
    DirectColor, StaticColor, TrueColor, GrayScale, or StaticGray.
    The default value is default (any visual will do).
    This resource indicates the visual that should be tried first.
  \item[pgxwin.Win.crosshair: ] Either True or False; the default is False.
    If true, this sets the default active cursor to cross hairs.
  \item[pgxwin.server.visible: ] Either True or False; the default is True.
    If false, the server window will be hidden.
  \item[pgxwin.server.iconGeometry: ] An icon geometry format:
    {\tt +X+Y} that specifies where the server icon should appear.
  \item[pgxwin.Win.iconGeometry: ] An icon geometry format:
    {\tt +X+Y} that specifies where the window icon should appear.
  \item[pgxwin.Win.geometry: ] A window geometry format:
    {\tt WIDTHxHEIGHT+X+Y}.
    The default is something like 867x670.
\end{description}

Files created for hardcopy\index{Hardcopy} devices must be
spooled to the printer.
Two commands and one string variable exist to simplify this for the user.
The commands that spool a plot to a hardcopy device are
{\tt hardcopy}\index{Commands!{\tt hardcopy}}%
\index{Hardcopy!commands!{\tt hardcopy}}
and {\tt phard}\index{Commands!{\tt phard}}%
\index{Hardcopy!commands!{\tt phard}}.
The string variable {\em print}\index{String Variables!variables!{\em print}}%
\index{Hardcopy!the {\em print} string}
is used to control {\it how} the plot file will be spooled to the printer.

The command
{\tt hardcopy}\index{Commands!{\tt hardcopy}}
\index{Hardcopy!commands!{\tt hardcopy}}
is used when the current device is a hardcopy (\ie non-interactive) device
and you are finished generating output to that device.
This command closes the output file and releases it to be printed by \wip.
Alternatively, the command {\tt phard}\index{Commands!{\tt phard}} may be
used when the current device is an interactive device and the user
desires to playback the list of current commands to a hardcopy device
and then resume plotting on the initial interactive device.
Both commands use the string variable {\em print} to describe how to
spool the resulting plot file.

Since the method needed to spool the file to the selected printer
is site and device specific, a flexible method was needed to allow users
to spool their plots.
Generally, the command to send a plot to a printer is quite simple and
may be specified by a single statement.
The string variable {\em print}\index{String Variables!variables!{\em print}}
contains the command needed for the system to plot the file `file-name'
and overrides the initial default value set when \wip\ was compiled.
For example, the default value for
{\em print}\index{String Variables!variables!{\em print}}
on a Unix system is ``{\tt lpr \%s}'' where the ``{\tt \%s}''
represents a variable that is used
as a placeholder for the value of the plot file name.
The value for {\em print} may be customized for every instance of \wip\ if
it is defined in your \wipinit\index{wipinit@\wipinit} file
(see Appendix~\ref{a:wipinit} on how to set up this command in this file).
Alternatively, the command
\begin{wiplist}%
  \item {\tt set print myprog \%s \verb+|+ lpr}
\end{wiplist}
illustrates how a particular instance for the value of
{\em print}\index{String Variables!variables!{\em print}}
may be defined by using the command {\tt set}\index{Commands!{\tt set}}.
Chapter~\ref{c:uservar} provides more details about the use the
{\tt set} command.
In this example, the string variable {\em print} has been defined to
run the program {\tt myprog} with the plot file as its only
argument and then pipe the result to the standard printer.
Note the use of the {\tt \%s} variable to insure that the file is passed
to the program {\tt myprog} rather than appended to the end of the command.
If the {\tt \%s} was not used, the resulting command would have been
the same as if the user typed:
\begin{wiplist}%
  \item {\tt set print myprog \verb+|+ lpr \%s}
\end{wiplist}

Sometimes, it is useful to do everything except spool the file to a
printer.  This might happen when generating color PostScript files
and it is desired to screen preview the file before shipping them off to
the printer.
To override the printing command, set the {\em print} string to ``ignore''
as in the example:
\begin{wiplist}%
  \item {\tt set print ignore}
\end{wiplist}
This will cause the {\tt spool} and {\tt hardcopy} commands to perform
all of the post-processing on the hardcopy file but to ignore the print
statement.
Setting the {\em print} variable to an empty string will
have the same effect.

\section{Help, Echo, and Show}
\mylabel{s:help}
\index{Help}

Help is available on-line in \wip.
The location and name of the on-line help file is set when \wip\ is compiled.
Generally, this should never need to change.
However, there are two ways the location or name of this file can
be reassigned without having to recompile \wip.
The first is to set the environment variable
``WIPHELP''\index{Help!on-line help file!\$WIPHELP}.
The other way is for the user to do this via their \wipinit file
(see Appendix~\ref{a:wipinit} for details).

In addition to this manual, there are three modes of help
that are available within the interactive mode of \wip.
The command {\tt help} has an optional argument that, depending on its
use and whether or not it is present, provides three different styles of help.
They are:
\begin{description}
\index{Commands!{\tt help}}\index{Help!commands!{\tt help}}
  \item [{help ? --}] Lists the names of all currently available
\samepage
    commands and the names of the currently defined macros.
  \item [{help --}] Lists each command along with a one line
    description of what the command is suppose to do and the
    necessary and/or optional arguments.
  \item [{help {\em cmd} --}] Provides a complete description of the 
    command {\em cmd}.  Appendix~\ref{a:cmdname} provides the
    equivalent of this option for every available command.
  \end{description}

In addition to the on-line help, \wip\ also provides a way to see the
current value of many
of the attributes\index{Graphical Attributes},
coordinate system positions\index{Coordinate System},
user variables\index{User Variables}, and
image\index{Images} characteristics\index{Images!image characteristics}.
This may be done with either the command
{\tt echo}\index{Commands!{\tt echo}} or the command
{\tt show}\index{Commands!{\tt show}}.
The command {\tt echo} enables the user to display, at the command line,
the current value of almost any item available to \wip.
The {\tt echo} command will also evaluate expressions and print literal text.
Multiple items may also be presented using only one {\tt echo} command.
Throughout the rest of this manual, several examples of the command {\tt echo}
will be presented.
The command {\tt show} displays not only many of the items listed above,
but also the user variable ``name'' (shown in upper case) that
is associated with their values
(see the topics described in the Chapters of Part~\ref{p:advanced}
for a more detailed description on how to make use of these names).

\section{Coordinate Systems}
\mylabel{s:coord}
\index{Coordinate System}

There are two distinct coordinate systems in \wip.
They are the
{\em Device}\index{Device Coordinates|see{Viewport}}\index{Viewport}
and the
{\em World}\index{World Units}\index{Coordinate System!World}%
\index{User Limits|see {World Units}}
coordinates.
The device, or {\em viewport}, coordinates represent positions on the
device and are always in units that range between (0 -- 1).
The {\em world} coordinates represent what the
user usually associates as the limits or range of the plot.
For example, a plot of height versus weight might have
the abscissa ($x$-coordinate) represent weight
and the ordinate ($y$-coordinate) height.
The actual values of weight and height represent
the {\em world} coordinates.
Where these values map onto the device represent
the {\em device} coordinates.

This rectangular region of the {\em world} coordinate space
(termed the {\em window}\index{Window}\index{Coordinate System!Window})
is mapped onto a specified rectangle
(termed the {\em viewport}\index{Viewport}\index{Coordinate System!Viewport})
on the 
{\em view surface}\index{View Surface}\index{Coordinate System!View Surface}
(the screen of an interactive display or a sheet of paper on a hardcopy
plotter).
The following list of commands affect either the
{\em world} coordinate system or the position and size of the {\em viewport}.

\begin{itemize}
  \item World coordinates
  \begin{description}
    \item [{environment --}]\index{Commands!{\tt environment}}
      \index{Coordinate System!commands!{\tt environment}}
      sets the world coordinates based on the input arguments or
      the extreme minimum and maximum of the current X and Y arrays.
    \item [{header --}]\index{Commands!{\tt header}}
      \index{Coordinate System!commands!{\tt header}}
      reads header information from the current image file and
      uses the header values to set the world coordinates.
      This command expands the world limits by, effectively,
      the width of half a pixel in each direction.
    \item [{limits --}]\index{Commands!{\tt limits}}
      \index{Coordinate System!commands!{\tt limits}}
      sets the World coordinates based on the input arguments or
      the limits of the current X and Y arrays.
    \item [{winadj --}]\index{Commands!{\tt winadj}}
      \index{Coordinate System!commands!{\tt winadj}}
      sets the world coordinates (as with {\tt limits})
      and then adjusts the viewport
      values to have the same aspect ratio.
  \end{description}

  \item Viewport coordinates
  \begin{description}
    \item [{panel --}]\index{Commands!{\tt panel}}
      \index{Coordinate System!commands!{\tt panel}}
      selects a sub-region within the current viewport
      as a new, {\it temporary} viewport.  The gap between the panels
      created with this command may be adjusted using the command
      {\tt submargin}\index{Commands!{\tt submargin}}.
    \item [{paper --}]\index{Commands!{\tt paper}}
      \index{Coordinate System!commands!{\tt paper}}
      changes the size of the view surface to a specified width
      (in inches) and height (based on the input aspect ratio).
      This, in turn, sets the size of the viewport.
    \item [{scale --}]\index{Commands!{\tt scale}}
      \index{Coordinate System!commands!{\tt scale}}
      sets the viewport size scale based on the range
      of the world coordinates and the scaling factor given as input arguments.
    \item [{viewport --}]\index{Commands!{\tt viewport}}
      \index{Coordinate System!commands!{\tt viewport}}
      sets the viewport size with arguments given in viewport
      coordinates (ranging from 0 to 1 in all directions for all devices).
    \item [{viewsize --}]\index{Commands!{\tt viewsize}}
      \index{Coordinate System!commands!{\tt viewsize}}
      sets the viewport size with arguments given in inches.
    \item [{vstand --}]\index{Commands!{\tt vstand}}
      \index{Coordinate System!commands!{\tt vstand}}
      sets the viewport size to the full area of the view surface less a
      margin of 4 character heights on each side.
    \item [{winadj --}]\index{Commands!{\tt winadj}}
      \index{Coordinate System!commands!{\tt winadj}}
      sets the world coordinates (as with {\tt limits})
      and then adjusts the viewport
      values to have the same aspect ratio.
  \end{description}
\end{itemize}

\section{Attributes}
\mylabel{s:attrib}
\index{Graphical Attributes}

There are several graphic characteristics of \wip\ that may be set by either the
program or by the user.
These have the special behavior that they will remain in effect
until either the end of the plot or until they are changed again by the user.
These characteristics include the color index, text font, line style
and width, fill type, and character size expansion to name a few.
Table~\ref{t:attrib} contains a list of the attributes,
their default values,
and the commands used in \wip\ to modify them.
The values of the attributes listed in Table~\ref{t:attrib} remain intact
until changed by another call to the controlling command with one exception:
attributes identified with a `Yes' in the Reset column are reset
to their default value every time a new device is selected or the user
issues the {\tt reset}\index{Commands!{\tt reset}} command.
The examples in Appendix~\ref{a:samples} illustrate how to use
several of these graphical attributes
and Figure~\ref{p:symbol} displays the first 32 graphical markers used
by the command {\tt symbol}\index{Commands!{\tt symbol}}.
\begin{table}
  \caption{Graphical Attributes}
  \mylabel{t:attrib}
  \index{Graphical Attributes!table of}
  \index{Commands!{\tt angle}}
  \index{Commands!{\tt ask}}
  \index{Commands!{\tt color}}
  \index{Commands!{\tt expand}}
  \index{Commands!{\tt fill}}
  \index{Commands!{\tt font}}
  \index{Commands!{\tt lstyle}}
  \index{Commands!{\tt lwidth}}
  \index{Commands!{\tt symbol}}
  \index{Symbols}
  \centering
  \begin{tabular}{|l|c|l|l|} \hline\hline
    \multicolumn{1}{|c}{Attribute} & \multicolumn{1}{|c}{Reset} &
    \multicolumn{1}{|c}{Default value} & \multicolumn{1}{|c|}{Command} \\ \hline
    &&& \\
    Angle          & No  & 0 (No rotation -- degrees)      & ANGLE $D$  \\
    Prompting      & Yes & 0 (No new page prompting)       & ASK $n$    \\
    Color Index    & Yes & 1 (Foreground color index)      & COLOR $n$  \\
    Character Size & Yes & 1 (In character height units)   & EXPAND $n$ \\
    Fill Type      & Yes & 1 (Solid fill)                  & FILL $n$   \\
    Font Style     & Yes & 1 (Normal simple font)          & FONT $n$   \\
    Line Style     & Yes & 1 (Full line)                   & LSTYLE $n$ \\
    Line Width     & Yes & 1 (One stroke)                  & LWIDTH $n$ \\
    Current Symbol & No  & 0 (Open square)                 & SYMBOL $n$ \\
    &&& \\ \hline\hline
  \end{tabular}
\end{table}

Table~\ref{t:colors} presents the predefined colors representations
along with the corresponding Red-Green-Blue (RGB) and
Hue-Lightness-Saturation (HLS) intensities.
All but one of the color intensities have values that are in
the inclusive range of 0.0 -- 1.0, with 1.0 being the maximum intensity.
The exception is Hue which is a cyclic quantity expressed as an angle in
degrees and has the inclusive range of 0 -- 360.
If the three RGB intensities are the same, the color is a shade of gray.
\begin{table}[htb]
  \caption{Default Color Representations}
  \mylabel{t:colors}
  \index{Colors!default colors!table of}
  \centering
  \begin{tabular}{|c|l|r|r|} \hline\hline
    \multicolumn{1}{|c|}{Index} & \multicolumn{1}{c|}{Color} &
    \multicolumn{1}{|c|}{(H, L, S)} & \multicolumn{1}{c|}{(R, G, B)} \\ \hline
    &&& \\
     0 & Black (background)    &   0, 0.00, 0.00  &  0.00, 0.00, 0.00 \\
     1 & White (foreground)    &   0, 1.00, 0.00  &  1.00, 1.00, 1.00 \\
     2 & Red                   & 120, 0.50, 1.00  &  1.00, 0.00, 0.00 \\
     3 & Green                 & 240, 0.50, 1.00  &  0.00, 1.00, 0.00 \\
     4 & Blue                  &   0, 0.50, 1.00  &  0.00, 0.00, 1.00 \\
     5 & Cyan (Blue + Green)   & 300, 0.50, 1.00  &  0.00, 1.00, 1.00 \\
     6 & Magenta (Red + Blue)  &  60, 0.50, 1.00  &  1.00, 0.00, 1.00 \\
     7 & Yellow (Red + Green)  & 180, 0.50, 1.00  &  1.00, 1.00, 0.00 \\
     8 & Orange (Red + Yellow) & 150, 0.50, 1.00  &  1.00, 0.50, 0.00 \\
     9 & Green + Yellow        & 210, 0.50, 1.00  &  0.50, 1.00, 0.00 \\
    10 & Green + Cyan          & 270, 0.50, 1.00  &  0.00, 1.00, 0.50 \\
    11 & Blue + Cyan           & 330, 0.50, 1.00  &  0.00, 0.50, 1.00 \\
    12 & Blue + Magenta        &  30, 0.50, 1.00  &  0.50, 0.00, 1.00 \\
    13 & Red + Magenta         &  90, 0.50, 1.00  &  1.00, 0.00, 0.50 \\
    14 & Dark Gray             &   0, 0.33, 0.00  &  0.33, 0.33, 0.33 \\
    15 & Light Gray            &   0, 0.66, 0.00  &  0.66, 0.66, 0.66 \\
    16-255 & Undefined & & \\
    &&& \\ \hline\hline
\end{tabular}
\end{table}

\section{Boxes}
\mylabel{s:box}
\index{Boxes}

Fundamental to the display of graphics in \wip\ is an item called a box.
Boxes are constructs that border a plot and may identify the scale along
a particular axis.
There are several possible arrangements of the arguments
to the command
{\tt box}\index{Commands!{\tt box}}\index{Boxes!commands!{\tt box}}
that will lead to very different borders.
Many of the attributes listed in section~\ref{s:attrib}
also affect the resulting appearance of the box.

Prior to calling the {\tt box} command, the {\em world} coordinates,
or limits, should be set.
Refer to section~\ref{s:coord} for a collection of  commands which
allow the user to specify the
{\em world}\index{World Units}\index{Coordinate System!World}
coordinates of the plot.

Two other characteristics of the border may be specified prior to a call to
{\tt box}:
the major tick mark interval and the number of minor tick marks per major tick.
Use the command {\tt ticksize}\index{Boxes!commands!{\tt ticksize}}%
\index{Commands!{\tt ticksize}}%
\index{Commands!{\tt ticksize}!arguments}
to input the major interval and number of minor ticks for each axis.
If the major tick mark interval is specified as 0 (the default),
then {\tt box} will calculate its own interval.
Likewise, if the number of minor tick marks per major tick mark is set
to 0 (the default), then {\tt box} will attempt to calculate
a ``nice'' interval.
Setting the number of minor tick marks to 1 will inhibit any minor tick
marks (except for
logarithm) regardless of the arguments to the command {\tt box}.
The units of the arguments to {\tt ticksize} are the same used to
specify the limits of the {\em world} coordinates.

There are two arguments to the command {\tt box} which determine
how \wip\ will display the border.
Each argument consists of one or more letters (case independent; no
intervening spaces) which tell \wip\ how to construct the box.
The current letters (and their meanings) used to construct
an argument to {\tt box} are shown in Table~\ref{t:box}.
\begin{table}
  \caption{Box Option Characters}
  \mylabel{t:box}
  \index{Commands!{\tt box}}
  \index{Commands!{\tt box}!arguments}
  \index{Commands!{\tt limits}}
  \index{Commands!{\tt ticksize}}
  \index{Commands!{\tt ticksize}!arguments}
  \centering
  \setlength{\tabentrylen}{\textwidth}
  \addtolength{\tabentrylen}{-1.0in}
  \begin{tabular}{|c|p{\tabentrylen}|} \hline\hline
    \multicolumn{1}{|c}{Option}
   & \multicolumn{1}{|c|}{What It Produces} \\ \hline
   & \\
 A & Draw Axis line (X axis is a horizontal line at Y=0, Y axis is
     a vertical line at X=0). \\
 B & Draw Bottom (X) or Left (Y) edge of the frame. \\
 C & Draw Top (X) or Right (Y) edge of the frame. \\
 D & Include the superscript symbols ${}^\circ$, $'$, and $''$.
     If this option is used, the option `Y' is implicitly implied. \\
 F & Write only the last part of the label for the first time (left or
     bottom most) tick on the axis (see option `Z').
     (\eg if the full first label is 17 42 34.4, then only write 34.4.) \\
 G & Draw a Grid of vertical (X) or horizontal (Y) lines at each of
     the major tick mark positions. \\
 H & Include the superscript symbols `d', `h', `m', and `s'. \\
 I & Invert the tick marks; \ie\ draw them outside the viewport
     instead of on the inside. \\
 L & Label the axis Logarithmically.  If option `Z' is used, then this
     option is ignored.  The major tick interval is {\em always} 1.0; the
     number of minor ticks is {\em always} 8.  The numeric label is $10^{x}$,
     where $x$ is the world coordinate at the tick mark. \\
 M & Write Numeric labels in the unconventional location Above (X)
     or to the Right (Y) of the viewport. \\
 N & Write Numeric labels in the conventional location Below (X)
     or to the Left (Y) of the viewport. \\
 O & Omit leading zeros in numbers $< 10$ in time labels
     (\eg $3^h 3^m 1.^s2$ will be written rather than $03^h 03^m 01.^s2$).
     The day field is not affected by this option and this option is
     ignored unless used with the `Z' option. This option is ignored
     when used with the `V' option as it makes it impossible to align
     the labels nicely.
     This option is useful to help save space on the X-axis. \\
 P & Extend (``Project'') the major tick marks outside the box
     (ignored if the option `I' is also specified). \\
 S & Draw Minor Tick marks (Subticks). \\
 T & Draw Major Tick marks at the major coordinate interval. \\
 V & Orient numeric labels Vertically. This is only applicable to the
     Y axis.  The default is to write the Y-labels parallel to the axis. \\
 Y & Do not include the day field in time labeling.  This means that
     labels are ``HH MM SS.S'' rather than ``DD HH MM SS.S''.
     The hours will accumulate beyond 24, if necessary, in this case. \\
 Z & Use time labeling ((DD) HH MM SS style).  The World
     coordinates of the box (see {\tt limits}) and the tick mark increments
     (see {\tt ticksize}) should be given in units of seconds. \\
 1 & Force decimal labeling, instead of automatic choice. \\
 2 & Force exponential labeling, instead of automatic. \\
 0 & By itself, a zero produces a null-operation argument; useful
     to produce a box along only one axis. \\
   & \\ \hline\hline
  \end{tabular}
\end{table}
Unrecognized letters are quietly ignored as are any
multiple occurrences of letters.
To call {\tt box} with arguments for only one axis, use the null
argument (0); this will suppress any drawing along the unused axis.
For example, to produce a standard box but only along the $y$-axis, use
\begin{wiplist}%
  \item {\tt box 0 bcnstv}\hfill\# Draw only a $y$-axis box.
\end{wiplist}

By calling {\tt box} with one group of arguments, then changing one
or more attributes, and then calling {\tt box} again with different arguments,
one can produce a box with considerable variation.
The following list of commands provides such an example.
\begin{wiplist}%
  \item {\tt limits 0 10 0 100}\hfill\# Set the World coordinates.
  \item {\tt lstyle 1}\hfill\# Set the line style to a solid line.
  \item {\tt color 2}\hfill\# Set the color index to red.
  \item {\tt box bcst bcst}\hfill\# Draw the sides and tick marks.
  \item {\tt color 4}\hfill\# Set the color index to blue.
  \item {\tt box n nv}\hfill\# Numerically label the tick marks.
  \item {\tt color 14}\hfill\# Set the color index to dark gray.
  \item {\tt lstyle 2}\hfill\# Set the line style to a dashed line.
  \item {\tt box g g}\hfill\# Draw grid lines at each tick mark.
\end{wiplist}
As another example, the command
\begin{wiplist}%
  \item {\tt box bcnsthz bcnstvdyz}
\end{wiplist}
illustrates how to use the {\tt box} command with astronomical coordinates
(\ie right ascension for the $x$ axis and declination for the $y$ axis).
The first argument ({\tt bcnsthz}) tells {\tt box} to draw the bottom
and top edge of the frame, draw minor and major tick marks, and
numerically label the tick marks using time labeling and
hour, minute, and second superscripts.
The second argument ({\tt bcnstvdyz}) is the same as the first except that
it suppresses any day formatting, orients the label vertically, and
requests time labeling with degree, minute, and second superscripts.

Two string\index{String Variables}
variables associated with the $x$ and $y$ axis
arguments of {\tt box} are
{\em xbox}\index{String Variables!variables!{\em xbox}} and
{\em ybox}\index{String Variables!variables!{\em ybox}}, respectively.
If the {\tt box} command is called without any arguments,
these strings are substituted in their place.
Appendix~\ref{a:wipinit} shows how to have these strings loaded
as the defaults every time \wip\ is run.

\section{Text}
\mylabel{s:text}
\index{Labels|(}

\wip\ provides many commands which will display a string of characters.
Strings may be altered dramatically by judicious use of many of the
graphical attributes shown in Table~\ref{t:attrib}.
The commands that are used to write labels or titles on the plot are
shown in the list below along with a brief description.
Use {\tt help cmd}\index{Commands!{\tt help}} (where {\tt cmd} is one of
the following commands) or see Appendix~\ref{a:cmdname}
for a more detailed description of each of the following commands.

\begin{description}
  \item [{id --}]\index{Commands!{\tt id}}\index{Labels!commands!{\tt id}}
    displays identification information at the bottom of the current plot page.
  \item [{label --}]\index{Commands!{\tt label}}
    \index{Labels!commands!{\tt label}}
    writes a string at the current pen position.
  \item [{mtext --}]\index{Commands!{\tt mtext}}
    \index{Labels!commands!{\tt mtext}}
    writes a string relative to a side of the viewport.
  \item [{putlabel --}]\index{Commands!{\tt putlabel}}
    \index{Labels!commands!{\tt putlabel}}
    writes a string at the current pen position but justified
    by a requested amount.  If this amount is negative,
    the cursor\index{Cursor!commands!{\tt putlabel}} is used to
    specify a new current pen position (see Section~\ref{s:cursor}).
  \item [{xlabel --}]\index{Commands!{\tt xlabel}}
    \index{Labels!commands!{\tt xlabel}}
    writes a centered string below the bottom of the current box.
  \item [{ylabel --}]\index{Commands!{\tt ylabel}}
    \index{Labels!commands!{\tt ylabel}}
    writes a centered string left of the left edge of the current
    box.\footnote{Using {\tt ylabel} along with an annotated box generated
    using the ``Z'' option may not produce an optimally placed label.
    Try using
    {\tt mtext L 3.5 0.5 0.5 {\em ylabel-text}} instead of {\tt ylabel}
    (as shown in Example~\ref{p:image}).}
\end{description}

All of the text commands above (except {\tt id}) require a label as an argument.
This label may be modified by any of the escape sequences shown in
Table~\ref{t:text}.
\begin{table}[htb]
  \caption{Label Escape Sequences}
  \mylabel{t:text}
  \index{Labels!escape sequences!table of}
  \index{User Variables}
  \index{Symbols!markers}
  \index{Symbols!Hershey symbols}
  \centering
  \setlength{\tabentrylen}{\textwidth}
  \addtolength{\tabentrylen}{-1.25in}
  \begin{tabular}{|c|p{\tabentrylen}|} \hline\hline
 Escape Sequence & \multicolumn{1}{|c|}{What It Produces} \\ \hline
 & \\
 {\bf \esc{u}} & Either starts a superscript or ends a subscript. \\
 {\bf \esc{d}} & Either starts a subscript or ends a superscript. \\
               & ({\sc Note}: {\bf \esc{u}} and {\bf \esc{d}}
                  must always appear in pairs). \\
 {\bf \esc{b}} & Backspace (\ie do not advance the text pointer
                 after plotting the previous character). \\
 {\bf {$\backslash\backslash$}}
               & The backslash (escape) character (\esc{\,}). \\
 {\bf \esc{x}} & The multiplication sign. \\
 {\bf \esc{A}} & {\AA}ngstr\"{o}m symbol (\AA). \\
 {\bf \esc{m{\em n}}}
               & The graph marker {\em n} ({\em n} may be 1 or 2 digits). \\
 {\bf \esc{g{\em x}}}
               & Greek letter corresponding to the roman letter {\em x}. \\
 {\bf \esc{fn}} & Switch to Normal font (1). \\
 {\bf \esc{fr}} & Switch to Roman  font (2). \\
 {\bf \esc{fi}} & Switch to Italic font (3). \\
 {\bf \esc{fs}} & Switch to Script font (4). \\
 {\bf \esc{({\em n})}}
               & Hershey number {\em n} (1 to 4 decimal digits); the closing
                 parenthesis may be omitted if the next character is
                 neither a digit nor `)'.  This makes a number of special
                 characters (\eg mathematical, musical, astronomical,
                 and cartographical) available.  See section~\ref{p:hershey}
                 to see how to display the Hershey symbols available. \\
 {\bf \esc{[{\em x}]}}
               & User variable {\em x} (see Section~\ref{c:uservar}).
                 The current value of the specified user variable is
                 substituted into the string. The closing bracket may
                 be omitted only if the user variable is the last item
                 in the string. \\
 & \\ \hline\hline
  \end{tabular}
\end{table}
About half of the escape sequences listed in Table~\ref{t:text} remain in
effect until the end of the string or until another escape sequence
terminates its scope.
The following list of commands illustrates the use of a few of the text
commands along with several escape sequences.
\begin{wiplist}%
  \index{Commands!{\tt xlabel}}
  \index{Commands!{\tt label}}
  \index{Commands!{\tt mtext}}
  \item {\tt xlabel \esc{ga} (1950)}\hfill\# Labels $x$ axis as $\alpha$ (1950).
\samepage
  \item {\tt label HC\esc{d3}\esc{u}N}\hfill\# Makes the label HC${}_3$N.
  \item {\tt mtext t 2 .5 .5 Title}\hfill\# Label at top center of the panel.
\end{wiplist}

By far, the most flexible text command is
{\tt mtext}.\index{Commands!{\tt mtext}}%
\index{Labels!commands!{\tt mtext}}
The first four arguments to {\tt mtext} specify where the label will
appear on the viewport;
the remaining arguments represent the label to display.
The first argument is a character describing along which viewport side
the label will appear.
The character must be either `b', `l', `t', or `r' to specify,
respectively, the bottom, left, top, or right viewport side.
In addition, an optional character `v' may be appended (as in `lv' or
`rv') to the left or right character to specify that the label be
presented perpendicular to the viewport side.
The second argument to {\tt mtext} is a displacement
(in character height units) away from the viewport edge.
Positive values position the label ``away'' from the edge; negative values
move the label ``inside'' the viewport.
The next two arguments (arguments three and four to {\tt mtext})
specify the justification of the label.
The first of these two arguments specifies
a position along the viewport.
This position is a value relative to the total extent of the viewport edge
(\ie running from 0 to 1 left to right and bottom to top).
The second of these two arguments specifies how to justify the
label relative to this position with a value of 0 indicating
left justified; 0.5 centered; and 1 right justified.%
\index{Labels|)}

\section{Data Files}
\mylabel{s:data}
\index{Data Files}

Data files are simple ASCII text files that can be read to generate
point and line graphics (\ie 1-dimensional plots).
The command 
{\tt data}\index{Commands!{\tt data}}%
\index{Data Files!commands!{\tt data}}
opens a file and prepares it for reading.
The data is assumed to be column ordered
with any two columns separated by a comma or blank space (or both).
Currently, there is no restriction on the
contents of the data in each column except that
lines beginning with
the comment character (\#)\index{Comments!comment character(\#)}
are ignored.
Non-existent and empty data files are warned about and then ignored.

Data is usually read in by column format into \wip\ with the commands
{\tt xcolumn}\index{Commands!{\tt xcolumn}}%
\index{Data Files!commands!{\tt xcolumn}}
and
{\tt ycolumn}\index{Commands!{\tt ycolumn}}%
\index{Data Files!commands!{\tt ycolumn}}.
The required argument to these commands is the column number from which
to read the data
(\ie each line read is parsed into words and the command
evaluates only the selected word).
Incorrect data lines are usually printed out with a warning message.

Which lines are actually read from the data file may be restricted by the user.
The command 
{\tt lines}\index{Commands!{\tt lines}}
\index{Data Files!commands!{\tt lines}}
limits the range that the input data file is read from line {\em L1} to
{\em L2}, where {\em L1} and {\em L2} are the two required
arguments given with the command.
If the data file has fewer than {\em L2\/} lines, reading is stopped at
the end of the file.

In addition to the {\bf X} and {\bf Y} arrays discussed above,
there are presently two other predefined arrays:
the {\bf ERR} and {\bf PSTYLE} arrays.
The {\bf ERR} array is used to store error bar distances from the corresponding
{\bf X} and {\bf Y} positions
(see the command {\tt errorbar} in Section~\ref{s:onevarplts})
and
the {\bf PSTYLE} array is used to hold a symbol number for
each {\bf X} and {\bf Y} position
(see the command {\tt points} in Section~\ref{s:onevarplts}).
The {\bf ERR} and {\bf PSTYLE} arrays are used by other commands for other
purposes too; these usage variations will be identified when the
specific commands are discussed later.

Another use of the data files is to assign strings to
string variables\index{String Variables}.
String variables are a type of User Variable\index{User Variables}
and will be introduced in Chapter~\ref{c:uservar}.
The command {\tt string}\index{Commands!{\tt string}}
can be used to assign a word or a string to a string variable.
The only line read in the data file is defined by the value of
{\em L1} (see above) and which (space separated) words are assigned to
the string variable is determined by the arguments to the command.
Use {\tt help string}\index{Commands!{\tt help}}
or see Appendix~\ref{a:cmdname}
for a detailed description of the arguments to this command.

\subsection*{Dynamic Data Files}
\mylabel{ss:dynamicdata}
\index{Data Files!dynamic data}

There is a straightforward way to input data directly on the command
line within \wip.
If the argument to the {\tt data} command is {\tt stdin},
then the current mode is switched to a data input mode.
\wip\ will stay in that
mode until the string {\tt enddata} appears on a line by itself.
While in data input mode, every line is treated as input to this
dynamic data set.
When the input mode is terminated, the data may be read using any of
the techniques described above.
The data set created will remain active until either another data set is
opened (with the {\tt data} command) or the user exits from \wip.

\section{Cursor Interaction}
\mylabel{s:cursor}
\index{Cursor}

The command {\tt cursor}\index{Commands!{\tt cursor}}
\index{Cursor!commands!{\tt cursor}}
permits a simple means of interacting with the plot.
Several keys have been assigned to the {\tt cursor} command
which allow complicated plotting commands to be easily constructed.
Each command typed while interactively using the cursor can be saved in the
command buffer\index{Macros!buffer} (see Chapter~\ref{c:macros} for a
general description of macros and the command buffer)
and may be edited\index{Macros!editing} (using
the techniques described in section~\ref{s:macedit}) or
re-executed (with the command
{\tt playback}\index{Commands!{\tt playback}}).
The keystroke commands that are currently recognized are shown
in the list below and are {\em not} case sensitive.

\begin{description}
\index{Cursor!keystroke commands}
  \item [{A --}] Display the present cursor position and,
    if an image is loaded, the image intensity at that position,
  \item [{D --}] Draw to the current cursor position.
  \item [{M --}] Move to the present cursor position.
  \item [{P --}] Draw a point of the current symbol type at the present
    cursor position.
  \item [{S --}] Toggle the ability to save the keystrokes
    that follow as commands.
  \item [{X --}] Exit the interactive cursor routine.
  \item [{? --}] Present the list of current keystroke commands
    (this is the same as the command {\tt help cursor}).
\end{description}

There are a few commands in \wip\ which use the cursor to specify the
location where the command is to act.
Most significant is the command
{\tt putlabel},\index{Commands!{\tt putlabel}}
\index{Labels!commands!{\tt putlabel}}
which takes two arguments:
a justification value and a string to write.
The justification value ranges from 0 -- 1 where 0 means left
justification relative to the current cursor position, 1 means right
justification, and 0.5 means center the text on the cursor position.
However, \wip\ allows negative justifications.
Negative justifications signal \wip\ to
activate the cursor prior to displaying the text so that the user may
specify a new current cursor position.
The minus sign is just a flag to \wip, the justification that will
eventually reach the text displaying routine will still range
between 0 -- 1 (\ie abs(justification)).\footnote{Note that to
left justify the text {\em and} activate the cursor,
the justification argument should be something like -0.001.}
As an example, suppose a user were to type the command
\begin{wiplist}%
  \index{Commands!{\tt putlabel}}
  \item[\wipp] {\tt putlabel -0.5 A string to center}
\end{wiplist}
The cursor would then become active and allow the user to move to any
position on the screen.
Now, suppose the user
moved the cursor to the position (5,~8) and hit any key.
In this case, the keystroke commands are ``turned off'' and the first
key hit by the user terminates the cursor routine.
The string entered would appear centered at the position (5,~8)
just as if the user had typed the following two lines:
\begin{wiplist}%
  \index{Commands!{\tt move}}
  \index{Commands!{\tt putlabel}}
  \item {\tt move 5 8}
\samepage
  \item {\tt putlabel 0.5 A string to center}.
\end{wiplist}
In fact, if this command were typed interactively (or when reading an
input file off the command line\index{Command Line!input files}),
these latter two commands are exactly what would be saved in
the user's command buffer\index{Macros!buffer}.

There are a few instances, however, when the cursor command
will not function interactively (as shown in the previous example).
In these cases,
the cursor will only be used to return the cursor position selected;
the keystroke commands will be ignored.
These instances are when either of the
commands {\tt cursor}\index{Commands!{\tt cursor}}%
\index{Cursor!non-interactive}
or
{\tt putlabel}\index{Commands!{\tt putlabel}}%
\index{Labels!commands!{\tt putlabel}}%
\index{Cursor!commands!{\tt putlabel}}
are (i) executed from inside a macro\index{Macros};
(ii) entered into the command buffer\index{Macros!buffer} with the
command {\tt insert}\index{Commands!{\tt insert}}%
\index{Macros!commands!{\tt insert}}\index{Macros!editing};
or (iii) executed using the
command {\tt input}\index{Commands!{\tt input}}.
For these special cases,
the cursor will only return the current cursor position
and ignore any subsequent keystrokes.

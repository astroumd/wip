%% This file is to be included by latex in wip.tex
%
% Chapter:  Data Fitting
%
\mylabel{c:fit}
\myfile{fit.tex}

This chapter describes methods of how to fit data.
There are currently five different ways to fit data loaded in
the {\bf X} and {\bf Y} arrays.
Two of these methods fit linear data, one fits gaussians, and two
fit polynomials.
In addition, only a subset of the data need be fit.
Once a fit has been found, the fit may also be plotted.
The commands that perform these tasks are described below
(and also illustrated in Figure~\ref{p:fit}).

\section{Fitting}
\mylabel{s:fit}

The {\tt fit}\index{Commands!{\tt fit}}\index{Fitting!commands!{\tt fit}}
command is what is used to do all the fitting of data.
The first argument to {\tt fit} is required and is the type of fit to perform.
The first argument must be one of the following values:
\begin{itemize}
  \item lsqfit
  \item medfit
  \item polynomial
  \item legendre
  \item gaussian
\end{itemize}
Minimum match is in effect when identifying the fit type.

There are two different ways to fit linear data:
the Least Squares Method and the Method of Least Absolute Deviations.
Data that is exponential in form can also be fit with either of these
linear methods by first taking the logarithm of the data.
There are also two different ways to fit polynomial data.
One is a simple polynomial and the other is a fit of Legendre polynomials.
In addition to fitting linear and polynomial data, one can also fit gaussians.
More than one gaussian curve can be fit simultaneously.
Information particular to each type of fitting are discussed below.

In each method, the data from
the {\bf X} and {\bf Y} arrays will be fit.
And, except for the ``medfit'' method, a measure of the
error in the Y-direction will be taken from the data in
the {\bf ERR} array.
If no data has been previously loaded in the {\bf ERR} array,
then values of unity will be used.

Additionally, each time a fit is performed,
information about the fit will be displayed on the command line.
The information will contain the fit parameters as well as a measure
of the goodness of the fit.
Additional (optional) parameters provided to the {\tt fit} command permit
the user to assign these fit terms and their errors to
user variables\index{User Variables}.

\subsection*{Least Squares Method}
\mylabel{ss:lsqfit}
\index{Fitting!commands!{\tt fit}}
\index{Fitting!Methods!Least Squares}

The method of Least Squares trys to fit the input data
to a straight line model of the form $y(x) = a + bx$.
This is done by minimizing the chi-squared function:
\[ \chi^2(a,b) = \sum_{i=1}^{N}\left(\frac{y_i - a - bx_i}{\sigma_i}\right)^2 \]
where $\sigma_i$ is the error estimate for each data point (the {\bf ERR} data).

When a fit is found, this method returns the parameters $a$ and $b$,
the chi-squared value for the fit, and the probable uncertainties in the
estimates of $a$ and $b$.
If the data in the {\bf ERR} array was used to specify the errors on
the observation points ({\bf Y} array), a ``goodness of fit'' is also returned.
If, however, no errors are specified (there are no points in the {\bf ERR}
array), then the correlation coefficient ($r$) is returned
in place of the ``goodness of fit''.
If the ``goodness of fit'' term (represented by the letter `q') is
larger than about 0.1, then the fit is probably reasonable.
Values for the correlation coefficient ($r$) near $\pm 1$ are considered good.

\subsection*{Method of Least Absolute Deviations}
\mylabel{ss:medfit}
\index{Fitting!commands!{\tt fit}}
\index{Fitting!Methods!Least Absolute Deviations}

This is similar to the Least Squares Method described above except that
the uncertainties in the data (the {\bf ERR} array data) are ignored and,
instead of minimizing the chi-squared function, the merit function to
minimize is
\[ \sum_{i=1}^{N} \left|y_i - a - bx_i\right| \]
The use of the median in this type of fitting provides a more robust method
of fitting than the Least Squares method and is especially
useful when the data contains outlying points.

When a fit is found, this method returns the parameters $a$ and $b$ and
a measure of the mean absolute deviation (in y) of the data points
from the fitted line.

\subsection*{Simple Polynomial Fitting}
\mylabel{ss:polynomial}
\index{Fitting!commands!{\tt fit}}
\index{Fitting!Methods!Simple Polynomials}

A simple polynomial has the form:
\[ y(x) = \sum_{i=0}^{N-1} \left(a_{i} x^{i}\right) \]
where $a_{j}$ is the $j$th coefficient of a polynomial of degree $N$.
The method used to find a fit is just a generalization of the Least
Squares method extended to include more than two coefficients.

When a fit is found, this method returns the $N$ coefficients and then
the $N$ probable uncertainties in the estimates of the coefficients.
Finally, the chi-squared value of the fit is returned.

\subsection*{Legendre Polynomial Fitting}
\mylabel{ss:legendre}
\index{Fitting!commands!{\tt fit}}
\index{Fitting!Methods!Legendre Polynomials}

The method is identical to the Simple Polynomial Fit except that it uses
Legendre Polynomials.
Legendre Polynomials have the following recurrence relation:
\[ (n + 1)P_{n + 1}(x) = (2n + 1)xP_n(x) - nP_{n - 1}(x) \]
where $P_0(x) = 1$ and $P_1(x) - x$.

When a fit is found, this method returns the $N$ Legendre coefficients
and then the $N$ probable uncertainties in the estimates of the coefficients.
Finally, the chi-squared value of the fit is returned.

\subsection*{Gaussian Fitting}
\mylabel{ss:gaussfit}
\index{Fitting!commands!{\tt fit}}
\index{Fitting!Methods!Multiple Gaussians}

This method tries to fit one or more Gaussians to the data.
In addition to the style (which must be ``{\tt gaussian}''),
the {\em number} of Gaussians to fit must be supplied.
Additional parameters may also be supplied to help the fitting routine.
For each Gaussian fit, there are three parameters used to describe it:
the amplitude, the X-peak position, and the full width at half the peak
amplitude (FWHM).
Or as an equation:
\[ G_{i}(x)=A_i\times\exp\left(\frac{\left(x-B_{i}\right)^2}{C_{i}^2}\right) \]
where $A_i$ is the amplitude of the $i$th Gaussian;
$B_i$, the peak position;
and $C_i$, the FWHM.
The resulting curve will be the sum of the $N$ Gaussians:
\[ y(x) = \sum_{i=1}^{N} G_{i}(x) \]
Note that if any of the supplied Gaussian terms are negative,
they will be held fixed during the fitting.

When a fit is found, this method returns three coefficients for each
Gaussian fit and the chi-square value of the fit.

\section{Limiting and Displaying Fits}
\subsection*{Limiting A Fit}
\mylabel{ss:fitrange}
\index{Fitting!commands!{\tt range}}
\index{Fitting!Limiting A Fit}

There will be some data sets that should only have a subset fit.
This could be done by limiting the amount of data loaded into \wip.
But that is cumbersome at the least and error prone.
The {\tt range}\index{Commands!{\tt range}} command
allows a portion of the data to be selected and subsequent fitting
will only pertain to this subset.
The range command can limit data in the X-axis direction only
or both the X and Y-axis directions.
If the range arguments in either direction are equal, then the range is
set as unlimited.
Examples of how to use this command are shown in Figure~\ref{p:fit}.

\subsection*{Displaying A Fit}
\mylabel{ss:plotfit}
\index{Fitting!commands!{\tt plotfit}}
\index{Fitting!Plotting A Fit}

The {\tt plotfit}\index{Commands!{\tt plotfit}} command
provides an easy way to display the results of a fit.
With no arguments, the fit is displayed over the entire X-axis range.
The first two optional arguments provide a way to limit the X-axis
extent of the displayed fit.
The third argument is an optional step size.
This permits a way to display the fit with finer or coarser resolution.
By default, the command will use the data in the X array
to guess a reasonable step size.
Figure~\ref{p:fit} provides an example of how this command is used.

%% This file is to be included by latex in wip.tex
%
% Chapter:  Glossary
%
\mylabel{a:glossary}
\myfile{glossary.tex}

The following is a loose list of terms used throughout the \wip\ manual,
and perhaps not always explained in detail.

\begin{description}
  \item[{\bf ASCII --}]\index{ASCII} A numeric code standard for characters.
    Literally, ASCII stands for American Standard Code
    for Information Interchange.
    This is the most common character to integer translation code.
  \item[{\bf BASIC --}]\index{Images!image formats!Basic} An
    image type used by \wip\ to reflect many different image formats.
    The Basic image file is generally a binary file with an optional
    header section followed by the packed data.
    The image name used with the {\tt image}\index{Commands!{\tt image}}%
    \index{Images!commands!{\tt image}} command generally describes
    the format of the data and the size of the optional header section.
  \item[{\bf BIMA}] \htmladdnormallink{The Berkeley-Illinois-Maryland Association.}{http://bima.astro.umd.edu/}
  \item[{\bf C-shell --}] In UNIX, this is one of the
    possible shells that can be used to communicate with UNIX.
    Other shells are {\tt tcsh} (the T-shell), {\tt sh} (the Bourne Shell),
    and {\tt bash} (the GNU Shell).
  \item[{\bf FITS --}]\index{FITS}\index{Images!image formats!FITS}
    This stands for the
    \htmladdnormallink{Flexible Image Transport System}{http://fits.cv.nrao.edu/}
    and is a standard data format used to interchange and archive data
    between different computers.
    It can be used for image as well as other types of data.
  \item[{\bf host --}] Your local computer.
    The term {\it host} is often used as in 
    {\it host interpreter} and {\it host commands}, and is really meant to
    warn you that in this context commands may differ depending on which
    host/machine (often VMS versus UNIX) you work.
  \item[{\bf Miriad --}]\index{Images!image formats!Miriad}
    The \htmladdnormallink{Multichannel Image Reconstruction, Image Analysis
    and Display}{http://bima.astro.umd.edu/programmer.html}
    data reduction package is developed by the BIMA group and is used
    here to refer to the format of the images generated with that package.
  \item[{\bf PGPLOT --}]\index{PGPLOT}
    A graphics subroutine package developed by Tim Pearson at Caltech used
    to generate publication quality graphics on various graphic display devices.
    \htmladdnormallink{The PGPLOT subroutines}{http://astro.caltech.edu/~tjp/pgplot/}
    are the graphics routines used by \wip.
  \item[{\bf PostScript file --}] \index{PostScript} An ASCII\index{ASCII}
    text file in the PostScript language.  PostScript is a page
    description language, and has become an industry standard
    for printing high quality text and graphics.  There are
    PGPLOT\index{PGPLOT} device drivers which can create a PostScript
    file.  This file can be sent to a printer to get hardcopy output.
  \item[{\bf READLINE --}] The
    Readline Library\index{Readline Library}\index{Commands!readline!editing}%
    \index{Commands!readline!recall}
    is a set of routines for providing {\tt Emacs} style line input
    distributed by the Free Software Foundation.
    The library is not included within the \wip\ distribution.
    If, however, it is present on the system when \wip\ is built,
    then a compile time option can incorporate the library into
    the user interface of \wip\ providing {\tt Emacs} style command line
    recall and editing.
  \item[{\bf shell --}] A supposedly easier to handle
    front-end command processor used to communicate with a lower level,
    and often more difficult to handle program or operating system.
  \item[{\bf WIPHELP --}] \index{Help!on-line help file!\$WIPHELP}
    The name of the on-line help file.
    This environment variable defines the full directory path and file name
    of the file used by \wip\ to present on-line help.
  \item[{\bf X-WINDOWS --}] \index{X-Windows} The name of a windowing
    environment developed by MIT and available (free) via anonymous ftp.
\end{description}

\begin{rawhtml}

<DL>
<HR><A NAME=AITOFF>
<DT>
AITOFF [L B] - converts L-b coordinate values to equivalent x-y positions.
<DD>
	<CODE>AITOFF</CODE> will convert the X values (longitude) from
	the most recent <A HREF=#XCOLUMN><CODE>XCOLUMN</CODE></A>
	command and the Y values (latitude) from the most recent
	<A HREF=#YCOLUMN><CODE>YCOLUMN</CODE></A> command and convert them
	to the equivalent x-y positions for use in an Aitoff (more
	precisely, a Hammer) projection.  The <CODE>AITOFF</CODE>
	command MUST be called every time new data is inserted either
	with the <A HREF=#XCOLUMN><CODE>XCOLUMN</CODE></A> or
	<A HREF=#YCOLUMN><CODE>YCOLUMN</CODE></A> commands.  If the
	optional arguments <CODE>L</CODE> and <CODE>B</CODE> are present,
	they are converted to the equivalent x-y positions for use in
	an Aitoff projection.  For the equivalent Aitoff "box", see the
	<A HREF=#GLOBE><CODE>GLOBE</CODE></A> command.

<HR><A NAME=ANGLE>
<DT>
ANGLE D or ANGLE X1 X2 Y1 Y2 - sets angle of text or points to D degrees.
<DD>
	<CODE>ANGLE</CODE> will cause text from
	<A HREF=#LABEL><CODE>LABEL</CODE></A> and
	<A HREF=#PUTLABEL><CODE>PUTLABEL</CODE></A> commands to be
	presented at <CODE>D</CODE> degrees counterclockwise from the
	horizontal (+x direction).  Alternatively, if there are four
	arguments present, then they are taken to be the World (user)
	coordinates of two positions and the <CODE>ANGLE</CODE> is set
	to the slope of the line.

<HR><A NAME=ARC>
<DT>
ARC MAJX MAJY [ANG [STARTANG]] - draws an arc with major axes <CODE>MAJX</CODE>, <CODE>MAJY</CODE>.
<DD>
	<CODE>ARC</CODE> draws an arc at the current cursor position
	with major axes <CODE>MAJX</CODE> and <CODE>MAJY</CODE> (in World
	coordinate units).  The curve is drawn as a polygon with the
	characteristics specified by the command
	<A HREF=#FILL><CODE>FILL</CODE></A>.  The number of points used to
	generate the curve can be controlled by adjusting the value of
	the character expansion
	(see the command <A HREF=#EXPAND><CODE>EXPAND</CODE></A>).
	The optional parameter <CODE>ANG</CODE> defines the angular extent
	of the arc in degrees (set to 360 degrees if not given).
	An additional optional parameter, <CODE>STARTANG</CODE>,
	may be used to determine an initial offset angle of the arc relative to
	<CODE>MAJX</CODE> (this defaults to 0 degrees if this parameter is
	not present).  Use the command <A HREF=#ANGLE><CODE>ANGLE</CODE></A>
	to set the rotation of <CODE>MAJX</CODE> with respect to the
	positive x-axis.  All angles are in units of degrees and are
	defined to be zero toward the right of the viewport
	(regardless of the direction of the World Coordinate system)
	and increasing counter-clockwise.
	See also the <A HREF=#BEAM><CODE>BEAM</CODE></A> command.
	NOTE: If the World Coordinate system
	is currently RA and DEC, a rotated arc may not plot correctly.
	Use <A HREF=#HEADER><CODE>'header px px'</CODE></A> or
	<A HREF=#HEADER><CODE>'header so so'</CODE></A>
	(or some other equivalent offset option) to set the coordinate
	system to something comparable in scale.

<HR><A NAME=ARROW>
<DT>
ARROW X Y [ANGLE [VENT]] - Draws an arrow.
<DD>
	<CODE>ARROW</CODE> draws an arrow from the current cursor
	position to the World (user) coordinate (<CODE>X,Y</CODE>).
	The acute angle of the arrow point, in degrees, is specified by
	the optional input <CODE>ANGLE</CODE>; angles in the range
	20.0 to 90.0 give reasonable results.  If <CODE>ANGLE</CODE> is
	not present on the command line, a value of 45.0 degrees is used.
	The fraction of the triangular arrow-head that is cut away from the
	back is specified by the optional parameter <CODE>VENT</CODE>.
	A value of 0.0 gives a triangular wedge arrow-head; 1.0 gives
	an open &gt;.  Values of 0.3 to 0.7 give reasonable results.  The
	default value for VENT (if it is not present) is 0.3.
	<A HREF=#EXPAND><CODE>EXPAND</CODE></A> can be used to vary
	the length of the arrowhead.  For
	<A HREF=#EXPAND><CODE>EXPAND</CODE></A> set to 1.0, the default
	size of the arrowhead is 1/40th of the smaller of the width or
	height of the view surface.

<HR><A NAME=ASK>
<DT>
ASK K - Changes the current device prompt state.
<DD>
	The command <CODE>ASK</CODE> will set the prompt state to ON if
	<CODE>K</CODE> is set to 1 and the current device selected is
	interactive; otherwise, <CODE>ASK</CODE> will set the prompt
	state to OFF.  If <CODE>ASK</CODE> is set ON, then every command
	that erases the page will prompt the user before clearing the
	screen.  The default state of this is OFF; also, every time a new
	device is selected (see <A HREF=#DEVICE><CODE>DEVICE</CODE></A>),
	<CODE>ASK</CODE> is set to OFF.

<HR><A NAME=AUTOLEVS>
<DT>
AUTOLEVS NLEV [TYPE [MIN [MAX]]] - sets up the contour levels automatically.
<DD>
	<CODE>AUTOLEVS</CODE> will fill an internal contour level array
	with <CODE>NLEV</CODE> contours between the values <CODE>MIN</CODE>
	and <CODE>MAX</CODE>.  <CODE>NLEV</CODE> must be input, but the
	rest of the parameters are optional.  If either of or both
	<CODE>MIN</CODE> and <CODE>MAX</CODE> are not given, they are
	default to the minimum and maximum of the current array loaded
	using the command <A HREF=#IMAGE><CODE>IMAGE</CODE></A>.  The
	optional parameter <CODE>TYPE</CODE> is a string of characters
	specifying the type of contour levels to generate.  If
	<CODE>TYPE</CODE> is <CODE>LIN</CODE>, then the contour levels
	are linearly spaced (this is the default if <CODE>TYPE</CODE>
	is not present); if <CODE>TYPE</CODE> is <CODE>LOG</CODE>, then
	the levels are spaced evenly in the logarithm.

<HR><A NAME=BAR>
<DT>
BAR K [THRESH [GAP]] - draws bar graphs on (x,y) pairs in direction <CODE>90(K-1)</CODE>.
<DD>
	<CODE>BAR</CODE> is analogous to the command
	<A HREF=#BIN><CODE>BIN</CODE></A> except that it will draw a
	bar graph on all (x,y) coordinates read by
	<A HREF=#XCOLUMN><CODE>XCOLUMN</CODE></A> and
	<A HREF=#YCOLUMN><CODE>YCOLUMN</CODE></A> in the direction
	specified by the argument <CODE>K</CODE>.  Set the argument
	<CODE>K</CODE> to 1 to put the bar extending towards the +X
	direction; 2 for +Y; 3 for -X; and 4 for -Y.  The values read
	by the most recent call to <A HREF=#ECOLUMN><CODE>ECOLUMN</CODE></A>
	determine which color to draw individual bars.  If this array is
	empty, the current color index is used.  The optional argument
	<CODE>THRESH</CODE> may be used to specify the "bottom" of the
	bar.  If this optional argument is present, then it is used to
	anchor the beginning of the bar graph and should be entered in
	the units of that axis.  If this optional argument is not
	present, the box edge opposite the direction the bar extends
	acts as the threshold.  An additional optional argument
	<CODE>GAP</CODE> may be used to specify the "width" of the bar.
	If this is not present, the width is defined as the difference
	between the mean of the current point and the next point and the
	mean of the current point and the previous point.  If
	<CODE>GAP</CODE> is present, it should be in entered in units of
	that axis (i.e.  for <CODE>K == 2</CODE>, <CODE>GAP</CODE>
	should be in the units of the X-axis) and the bar will be
	centered on the point with half the width on either side.

<HR><A NAME=BEAM>
<DT>
BEAM MAJ MIN PA [OFFX OFFY [SCALE [FILLCOLOR [BGRECT]]]] - draws a beam.
<DD>
	<CODE>BEAM</CODE> draws a beam at the current cursor position
	outlined using the current color and line width properties.
	It has a major axis <CODE>MAJ</CODE> and minor axis
	<CODE>MIN</CODE> and is rotated by a position angle <CODE>PA</CODE>
	(which is in units of degrees defined as zero along the positive
	Y-axis (regardless of the current World Coordinate system) and
	increasing counter-clockwise).  The optional parameters
	<CODE>OFFX</CODE> and <CODE>OFFY</CODE> (which both default to 0)
	define the offset of the center of the beam from the current
	cursor position.  These offsets are in units of the width or
	height of the box that bounds the beam.  The beam will first be
	drawn solid using the optional input <CODE>FILLCOLOR</CODE> and is
	then drawn hollow with the current color.  If <CODE>FILLCOLOR</CODE>
	is not provided, it defaults to color index 15.
	By default, the axes will be scaled as if <CODE>header rd rd</CODE>
	has been previously called.  This means that the X-axis will be
	scaled by 15*cos(declination).  To override this scaling, the optional
	input <CODE>SCALE</CODE> can be provided and set to whatever
	scaling is desired (no scaling can be achieved by setting
	<CODE>SCALE</CODE> to 1).  If <CODE>SCALE</CODE> is set to any
	negative value (or is not present), then RA-Dec type scaling will be
	done.  The major and minor axes of the beam should be in the same
	units as the two axes.  This means if <CODE>SCALE</CODE> is set
	to anything but 1, then they should both be in the same units
	as the Y-axis.
	If the optional argument <CODE>BGRECT</CODE> is provided
	(it defaults to 0) and is not negative, then a solid rectangle
	will be drawn in that color (color index <CODE>BGRECT</CODE>) first.
	The rectangle will only be as large as the width and height of the beam.

<HR><A NAME=BGCI>
<DT>
BGCI N - sets the text background color index to <CODE>N</CODE>.
<DD>
	<CODE>BGCI</CODE> sets the text background color index for
	subsequent text.  By default, text does not obscure underlying
	graphics.  If the text background color index is positive,
	however, text is opaque:  the bounding box of the text is
	filled with the color specified by the value of <CODE>N</CODE>
	before drawing the text characters.  Use color index 0 to erase
	underlying graphics before drawing text.  If <CODE>N</CODE> is
	less than 0, the text will be transparent; if greater than or
	equal to 0, the text will be drawn on an opaque background with
	color index <CODE>N</CODE>.  The initial value of the text
	background color index is -1 (transparent) and is reset to -1
	whenever a new device is selected.

<HR><A NAME=BIN>
<DT>
BIN [K [GAP] ] - draws a histogram of (x,y) pairs.
<DD>
	<CODE>BIN</CODE> connects the coordinates read by
	<A HREF=#XCOLUMN><CODE>XCOLUMN</CODE></A> and
	<A HREF=#YCOLUMN><CODE>YCOLUMN</CODE></A> as a histogram.  If
	the optional argument <CODE>K</CODE> is present and equal to 0,
	then the individual X values denote the lower left edge (in X)
	of the bin.  If <CODE>K</CODE> is set to 1 or is not present,
	then the individual X values denote the center location of the
	bin.  If the optional parameter <CODE>GAP</CODE> is present, it
	represents the distance (in X axis units) that signals a gap in
	the histogram not to be connected.  This is useful for histograms
	with irregular spacing or directional changes.  A good value
	for <CODE>GAP</CODE> is 2.

<HR><A NAME=BOX>
<DT>
BOX [x [y] ] - makes a box labeled according to <CODE>LIMITS</CODE> and <CODE>TICKSIZE</CODE>.
<DD>
	<CODE>BOX</CODE> annotates the viewport with a frame, axes,
	numeric labels, etc.  The two arguments <CODE>X</CODE> and
	<CODE>Y</CODE> are strings of options that determine the type
	of box and the labeling that will be drawn.  The option letters
	may be in any order and do not depend on case.  If either
	parameter is omitted, the missing parameter is given the
	default value set by the string variables XBOX and YBOX (see
	the command <A HREF=#SET><CODE>SET</CODE></A>).  If none of the
	letters `A', `B', or `C' are specified, then tick marks will
	not be drawn.  The options currently available
<!-- "WIP: t:box are found in Table" -->
<!-- "WIP BEGIN Table" -->
	are:
	<PRE>
  A : Draw Axis line (X axis is a horizontal line at Y=0, Y axis
      is a vertical line at X=0).
  B : Draw Bottom (X) or Left (Y) edge of the frame.
  C : Draw Top (X) or Right (Y) edge of the frame.
  D : Include the superscript symbols 'o', ''', and '"'.
      If this option is used, the option `Y' is implicitly implied.
  F : Write only the last part of the label for the first time
      (left or bottom most) tick on the axis (see Z).  For example,
      if the full first label is 17 42 34.4, then only write 34.4.
      This option is useful for sub-panels that join each other.
      Care is needed because first label carries sign as well.
  G : Draw a Grid of vertical (X) or horizontal (Y) lines at each of
      the major tick mark positions.
  H : Include the superscript symbols 'd', 'h', 'm', and 's'.
  I : Invert the tick marks; i.e. draw them outside the viewport
      instead of on the inside.
  L : Label axis Logarithmically.  If option `Z' is used, then this
      option is ignored.  The major tick interval is ALWAYS 1.0;
      the number of minor ticks is ALWAYS 8.  The numeric label is
      10**(x) where x is the world coordinate at the tick mark.
  M : Write Numeric labels in the unconventional location Above (X)
      or to the Right (Y) of the viewport.
  N : Write Numeric labels in the conventional location Below (X)
      or to the Left (Y) of the viewport.
  O : Omit leading zeros in numbers < 10 in time labels.  For
      example, the label 3h 3m 1.2s will be written rather than
      03h 03m 01.2s.  The day field is not affected by this option
      and this option is ignored unless used with the `Z' option.
      This option is ignored when used with the `V' option as it
      makes it impossible to align the labels nicely.  This option
      is useful to help save space on the X-axis.
  P : Extend ("Project") the major tick marks outside the box.  This
      option will be ignored if option `I' is also specified.
  S : Draw Minor Tick marks (Subticks).
  T : Draw Major Tick marks at the major coordinate interval.
  V : Orient numeric labels Vertically.  This is only applicable
      to the Y axis.  The default is to write the Y-labels parallel
      to the axis.
  Y : Do not include the day field in time labeling.  This means
      that labels are "HH MM SS.S" rather than "DD HH MM SS.S".
      The hours will accumulate beyond 24, if necessary, in this case.
  Z : Use time labeling ((DD) HH MM SS style).  The World
      coordinates of the box (see LIMITS) and the tick mark increments
      (see TICKSIZE) should be given in units of seconds.
  1 : Force decimal labeling, instead of automatic choice.
  2 : Force exponential labeling, instead of automatic.
  0 : By itself, a zero produces a null operation argument; useful
      to produce a box along only one axis.
	</PRE>
<!-- "WIP END Table" -->

<HR><A NAME=BUFFER>
<DT>
BUFFER - predefined macro name that refers to the entire command buffer.
<DD>
	Typing <CODE>BUFFER</CODE> will execute the contents of the
	command buffer (the same as the command
	"<A HREF=#PLAYBACK><CODE>playback</CODE></A>" or
	"<A HREF=#PLAYBACK><CODE>playback buffer</CODE></A>").
	<CODE>BUFFER</CODE> may also be used as an argument for commands
	requiring a macro name (e.g.  <A HREF=#LIST><CODE>LIST</CODE></A>
	and <A HREF=#WRITE><CODE>WRITE</CODE></A>).

<HR><A NAME=COLOR>
<DT>
COLOR N - select color for lines and characters.
<DD>
	<CODE>COLOR</CODE> defines the current color index used for
	filling polygons, drawing lines, and labeling characters for
	the selected device.  <CODE>COLOR</CODE> is originally set to 1
	and is reset to this value whenever a new device is selected.
	The number of colors available is device dependent (the range
	can be displayed with the command
	'<A HREF=#ECHO><CODE>echo cmin cmax</CODE></A>').  If the
	requested color index is not available on the selected device,
	a color index 1 will be substituted.  The first 16 colors
	usually have predefined values.  Additional values are device
	dependent.  Each color associated with the color index
	<CODE>N</CODE> is listed in
<!-- "WIP: t:colors Table" -->
<!-- "WIP BEGIN Table" -->
	the following table:
	<PRE>
   n    Color                     n     Color
--------------------------------------------------------------
   0    Black (background)        8     Orange (Red + Yellow)
   1    White (foreground)        9     Green + Yellow
   2    Red                      10     Green + Cyan
   3    Green                    11     Blue + Cyan
   4    Blue                     12     Blue + Magenta
   5    Cyan (Blue + Green)      13     Red + Magenta
   6    Magenta (Red + Blue)     14     Dark Gray
   7    Yellow (Red + Green)     15     Light Gray
                                 16-255 Undefined
	</PRE>
<!-- "WIP END Table" -->

<HR><A NAME=CONNECT>
<DT>
CONNECT - connects (x,y) pairs with line segments.
<DD>
	<CODE>CONNECT</CODE> draws line segments connecting the
	coordinates read by <A HREF=#XCOLUMN><CODE>XCOLUMN</CODE></A>
	and <A HREF=#YCOLUMN><CODE>YCOLUMN</CODE></A>.  The lines are
	drawn using the current line style (set by the command
	<A HREF=#LSTYLE><CODE>LSTYLE</CODE></A>), in the current color
	(set by the command <A HREF=#COLOR><CODE>COLOR</CODE></A>), and
	the current thickness (set by the command
	<A HREF=#LWIDTH><CODE>LWIDTH</CODE></A>).

<HR><A NAME=CONTOUR>
<DT>
CONTOUR [N [BLANK [MIN]]] - makes a contour plot of an array read with <CODE>IMAGE</CODE>.
<DD>
	<CODE>CONTOUR</CODE> makes a contour plot of an array read with
	the <A HREF=#IMAGE><CODE>IMAGE</CODE></A> command at the
	contour levels set using either the
	<A HREF=#LEVELS><CODE>LEVELS</CODE></A> or
	<A HREF=#AUTOLEVS><CODE>AUTOLEVS</CODE></A> commands.  The contour
	levels MUST be set before a call to <CODE>CONTOUR</CODE>.  The
	contour level array will be modified according to the current
	value of the user variable SLEVEL and the string variable
	LEVTYPE (also see the command <A HREF=#SLEVEL><CODE>SLEVEL</CODE></A>)
	before calling any of the contour methods below.
	<P>
	Currently, there are four contour routines available.  The
	optional parameter <CODE>N</CODE> allows the user to select
	which of four contour drawing methods will be used to generate
	the plot.  If <CODE>N</CODE> is the character `s', then the
	command will produce a fast contour plot (this is the default
	if <CODE>N</CODE> is not present); if <CODE>N</CODE> is `t',
	then a smoother contour is produced; if <CODE>N</CODE> is `b',
	then the same contour plot as with `s' is produced but with
	blanking; and `l' produces the same contour plot as with `t',
	but with contour level labeling.
	<P>
	For <CODE>N</CODE> set to `b', an optional argument
	<CODE>BLANK</CODE> may be given to specify the blanking value
	such that array elements equal in value to <CODE>BLANK</CODE>
	are ignored (blanked).  For <CODE>N</CODE> set to `l', the
	optional argument <CODE>BLANK</CODE> defines the spacing along
	the contour between labels (in grid cells) and the optional
	parameter <CODE>MIN</CODE> specifies the minimum number of
	cells that must be crossed before drawing the first contour
	label.  The default value for <CODE>BLANK</CODE> is 0.0 when
	<CODE>N</CODE> = `b' and 16 for <CODE>N</CODE> = `l';
	<CODE>MIN</CODE> defaults to 8.  The optional parameters are
	ignored by the contour options <CODE>N</CODE> = `s' and `t'.
	<P>
	By default, each contour line is drawn with the current line
	attributes (color index, line style, and line width).  However,
	for options `t' and `l', the default is to draw negative contours
	values with dashed and positive contours values with solid lines.
	If, however, the parameter <CODE>N</CODE> is set to `-t'
	(option `t' prepended with a minus sign), then negative contours
	values will be drawn with the same line style as positive
	contours values.

<HR><A NAME=CURSOR>
<DT>
CURSOR [x y] - enables cursor and returns x-y location and the key pressed.
<DD>
	<CODE>CURSOR</CODE> enables the graphics cursor and returns the
	x-y position of the cursor along with the key pressed.
	<CODE>CURSOR</CODE> may also display the image value of the
	selected position if an array is currently loaded (see the
	<A HREF=#IMAGE><CODE>IMAGE</CODE></A> command) and the cursor
	position lies within the scope of the image region (see the
	command <A HREF=#TRANSFER><CODE>TRANSFER</CODE></A>).  The two
	optional arguments may be used to determine the starting
	position of the cursor.  If both parameters are given, they are
	used as the World (user) coordinates for the starting
	position.  If only one parameter is present, it is used as the
	X starting position and the Y position is set as the current
	cursor position.  Finally, if no arguments are present, the
	current cursor position is used as a starting position.
	<P>
	If <CODE>CURSOR</CODE> is called inside a macro (see the command
	<A HREF=#DEFINE><CODE>DEFINE</CODE></A>), with the command
	<A HREF=#PLAYBACK><CODE>PLAYBACK</CODE></A>, or is executed inside
	a file via the command <A HREF=#INPUT><CODE>INPUT</CODE></A>, then
	only one keystroke is allowed and no saving of commands will be
	done.  If, on the other hand, <CODE>CURSOR</CODE> is called
	interactively, then <CODE>CURSOR</CODE> may perform several
	functions, depending on the character typed.  Internal calls to
	<CODE>CURSOR</CODE> will occur repetitively until the letter `X'
	is typed.  On some devices, the left most mouse button will be
	associated with the keystroke `A'; the middle with `D'; and the
	right with `X'.
	<P>
	Other keystroke functions are:
	<PRE>
    A : Display the cursor position and image intensity.
    D : Draw to the current cursor position.
    M : Move to the present cursor position.
    P : Draw a point of the current symbol type.
    S : Toggles ability to save following keys as commands.
    X : Exit the cursor routine.
    ? : Present this list of key commands.
	</PRE>

<HR><A NAME=DATA>
<DT>
DATA fspec - opens the file "fspec" for reading data.
<DD>
	<CODE>DATA</CODE> opens an external file for reading data.  The
	file is assumed to have data in columns separated by spaces,
	tabs or commas.  Any column can be read as x or y coordinates
	with the <A HREF=#XCOLUMN><CODE>XCOLUMN</CODE></A> and
	<A HREF=#YCOLUMN><CODE>YCOLUMN</CODE></A> commands.

<HR><A NAME=DEFINE>
<DT>
DEFINE xxx - creates the macro xxx and enters define mode.
<DD>
	<CODE>DEFINE</CODE> creates an entry for the macro <CODE>xxx</CODE>,
	and subsequent commands (until the <A HREF=#END><CODE>END</CODE></A>
	command) are considered the body of the macro.  The prompt changes
	to <CODE>DEFINE</CODE>&gt; to indicate the mode.  A macro is just
	a collection of commands (or other macros) that can be executed
	with one call statement.  A macro is invoked by using its name as
	a command.  The contents of the macro may be listed by using the
	macro name as an argument to the command
	<A HREF=#LIST><CODE>LIST</CODE></A>.  Macros may be saved using the
	command <A HREF=#WRITE><CODE>WRITE</CODE></A> and may be played
	back using the command <A HREF=#PLAYBACK><CODE>PLAYBACK</CODE></A>.
	Macros may be passed arguments or variables whenever they are
	executed.  The ten tokens used inside the macro to identify the
	macro arguments are $1, $2, ..., $9, and $0.  These tokens may be
	used in the body of the macro, and when the macro is invoked its
	arguments will be substituted for these tokens.  These tokens MAY
	NOT be redefined inside the macro (for example, using the command
	<CODE>SET</CODE>).  If an argument is not present when the macro
	is invoked, the token is replaced with a blank space.

<HR><A NAME=DELETE>
<DT>
DELETE [N1 [N2 [MACRO]]] - removes the commands <CODE>N1 - N2</CODE> from a macro buffer.
<DD>
	<CODE>DELETE</CODE> removes commands from the macro listing
	(numbered according to what is shown with the command
	<A HREF=#LIST><CODE>LIST</CODE></A>).  If no arguments are present,
	then the last command in the command buffer (see
	<A HREF=#BUFFER><CODE>BUFFER</CODE></A>) is deleted.  If
	<CODE>N2</CODE> is not present, then only the command numbered
	<CODE>N1</CODE> in the command buffer
	(<A HREF=#BUFFER><CODE>BUFFER</CODE></A>) is removed.  If both
	<CODE>N1</CODE> and <CODE>N2</CODE> are present, then all commands
	from <CODE>N1</CODE> to <CODE>N2</CODE> (inclusive) are deleted
	from the command buffer (<A HREF=#BUFFER><CODE>BUFFER</CODE></A>).
	If the optional macro name is given, then lines <CODE>N1</CODE> to
	<CODE>N2</CODE> (inclusive) are removed from <CODE>MACRO</CODE>.
	For example, to remove line 4 from a macro named "<CODE>dojob</CODE>",
	use the command <CODE>delete 4 4 dojob</CODE>.
	Note that the command buffer macro name is "<CODE>BUFFER</CODE>".

<HR><A NAME=DEVICE>
<DT>
DEVICE device - initializes output to a graphics device.
<DD>
	<CODE>DEVICE</CODE> defines the "device specification" for the
	plot device.  The specification is system dependent, but usually
	has the form "device/type" or "file/type".  If the argument is a
	question mark ("?"), the user will be prompted to supply a device
	name after a list of the currently available devices is
	presented.  When a new device is selected, the
	<A HREF=#PANEL><CODE>PANEL</CODE></A> command is reset and the
	<A HREF=#COLOR><CODE>COLOR</CODE></A>,
	<A HREF=#EXPAND><CODE>EXPAND</CODE></A>,
	<A HREF=#FILL><CODE>FILL</CODE></A>,
	<A HREF=#FONT><CODE>FONT</CODE></A>,
	<A HREF=#LSTYLE><CODE>LSTYLE</CODE></A>, and
	<A HREF=#LWIDTH><CODE>LWIDTH</CODE></A> commands are called to reset
	each to a value of 1.  The color palette (see the command
	<A HREF=#PALETTE><CODE>PALETTE</CODE></A>) is also reset to
	the default palette and the coordinate transfer function
	(<A HREF=#TRANSFER><CODE>TRANSFER</CODE></A>) is reset to linear.
	Finally, the inquire state is turned off (see the command
	<A HREF=#ASK><CODE>ASK</CODE></A>).

<HR><A NAME=DOT>
<DT>
DOT [X Y] - makes a point of the current style at the current location.
<DD>
	<CODE>DOT</CODE> draws a point at the current location (set by
	<A HREF=#MOVE><CODE>MOVE</CODE></A>,
	<A HREF=#DRAW><CODE>DRAW</CODE></A>, etc.)
	(or at the position (X,Y) if provided) in the style determined by
	the latest call to <A HREF=#SYMBOL><CODE>SYMBOL</CODE></A> or
	<A HREF=#PCOLUMN><CODE>PCOLUMN</CODE></A>.  The symbol's size is
	governed by the last call to the command
	<A HREF=#EXPAND><CODE>EXPAND</CODE></A>.

<HR><A NAME=DRAW>
<DT>
DRAW X Y - draws a line to <CODE>(X,Y)</CODE> from the current coordinate position.
<DD>
	<CODE>DRAW</CODE> draws a line to the World (user) coordinate (X,Y)
	from the current location (set by a previous call to
	<A HREF=#MOVE><CODE>MOVE</CODE></A>,
	<A HREF=#DRAW><CODE>DRAW</CODE></A>, etc.).
	<CODE>DRAW</CODE> then makes (X,Y) the current position.

<HR><A NAME=ECHO>
<DT>
ECHO EXPRESSION [...] - displays the result of <CODE>EXPRESSION</CODE> on the screen.
<DD>
	<CODE>ECHO</CODE> acts just like the command
	<A HREF=#SET><CODE>SET</CODE></A> except that rather than assigning
	the result of the <CODE>EXPRESSION</CODE> to anything it displays
	the result on the screen.  See the command
	<A HREF=#SET><CODE>SET</CODE></A> for an explanation of
	<CODE>EXPRESSION</CODE>.  There may be more than one
	<CODE>EXPRESSION</CODE> present, but each one should be delimited
	by enclosing parenthesis.  Literal strings may also be displayed
	by enclosing the text in double quotes.  String variables may be
	displayed by listing their names.  For example, to display the
	current cursor position, use the command:
	<BR><CODE>echo "cx = " cx "cy = " cy</CODE>.

<HR><A NAME=ECOLUMN>
<DT>
ECOLUMN N - reads error bar data from column <CODE>N</CODE> of the data file.
<DD>
	<CODE>ECOLUMN</CODE> reads column N from the data file as the
	magnitudes of the error bar displacements from the corresponding (x,y)
	coordinates.  Note that the command
	<A HREF=#LOGARITHM><CODE>LOGARITHM</CODE></A>
	does not adjust the error values to logarithmic scale.
	<CODE>ECOLUMN</CODE> values are also used by the command
	<A HREF=#LOOKUP><CODE>LOOKUP</CODE></A>.

<HR><A NAME=END>
<DT>
END - terminates define mode, insert mode, or exits from the program.
<DD>
	In DEFINE mode, <CODE>END</CODE> finishes the macro definition
	and returns to EXECUTE mode.  In INSERT mode, <CODE>END</CODE>
	terminates inserting and returns to EXECUTE mode.  In EXECUTE
	mode, <CODE>END</CODE> exits from the program altogether.
	The commands <CODE>EXIT</CODE> and <CODE>QUIT</CODE> may be
	used as a substitute for the command <CODE>END</CODE>.

<HR><A NAME=ENVIRONMENT>
<DT>
ENVIRONMENT [X1 X2 Y1 Y2 [N [K]]] - sets the user limits and draws a box.
<DD>
	<CODE>ENVIRONMENT</CODE> can clear the current page, set up the
	World (user) limits and draw a standard box.  The first four
	optional arguments establish the World (user) limits of the plot
	region.  If they are not present, then the data read from the most
	recent call to the commands <A HREF=#XCOLUMN><CODE>XCOLUMN</CODE></A>
	and <A HREF=#YCOLUMN><CODE>YCOLUMN</CODE></A> is used to autoscale
	the limits.  If the limits are present, then two other optional
	parameters may be selected.  The first of these parameters,
	<CODE>N</CODE>, will set the scales of the X and Y axes to be equal,
	if it is equal to 1; otherwise the two axes will be scaled
	independently.  The final optional parameter <CODE>K</CODE> will
	determine what type of box to draw and may have the following values:
	<PRE>
    -2 : draw no box, axes or labels;
    -1 : draw box only;
     0 : draw box and label it with coordinates;
     1 : same as K=0, but also draw the coordinate axes (X=0, Y=0);
     2 : same as K=1, but also draw grid lines at the major
       : increments of the coordinates;
    10 : draw box and label X-axis logarithmically;
    20 : draw box and label Y-axis logarithmically;
    30 : draw box and label both axes logarithmically.
	</PRE>
	Both <CODE>N</CODE> and <CODE>K</CODE> default to 0 if not provided.

<HR><A NAME=ERASE>
<DT>
ERASE - erases the graphics screen.
<DD>
	<CODE>ERASE</CODE> erases the graphics screen.  If the current
	device is a hardcopy device, <CODE>ERASE</CODE> will cause the
	plot to advance to the next page.  If <A HREF=#ASK><CODE>ASK</CODE></A>
	is set ON and the current device is not a hardcopy device, then the
	user will be prompted before advancing to the next page.

<HR><A NAME=ERRORBAR>
<DT>
ERRORBAR K - draws error bars on (x,y) pairs in the direction <CODE>90(K-1)</CODE>.
<DD>
	<CODE>ERRORBAR</CODE> is analogous to
	<A HREF=#POINTS><CODE>POINTS</CODE></A>; it draws error bars at
	each of the (x,y) points (read by the most recent call to the
	commands <A HREF=#XCOLUMN><CODE>XCOLUMN</CODE></A> and
	<A HREF=#YCOLUMN><CODE>YCOLUMN</CODE></A>) at a length specified
	by the points read by the command
	<A HREF=#ECOLUMN><CODE>ECOLUMN</CODE></A>.  The argument <CODE>K</CODE>
	should be 1 to put the bar along the +x direction; 2 for +y;
	3 for -x; and 4 for -y.  If <CODE>K</CODE> is 5, then error bars
	are drawn in both the +x and -x directions.
	Likewise, if <CODE>K</CODE> is 6, then error bars
	are drawn in both the +y and -y directions.
	Use <A HREF=#EXPAND><CODE>EXPAND</CODE></A>
	to govern the size (extent) of the caps.
	Setting the expansion to zero means no caps are drawn.

<HR><A NAME=ETXT>
<DT>
ETXT - erases the text from the view surface without affecting graphics.
<DD>
	<CODE>ETXT</CODE> erases the text from the view surface (if the
	device is capable of doing this).  This should erase text without
	affecting the currently displayed graphics.  Nothing is done on
	devices without this capability.

<HR><A NAME=EXPAND>
<DT>
EXPAND E - expands all characters and points by a factor <CODE>E</CODE>.
<DD>
	<CODE>EXPAND</CODE> sets the expansion for all characters and points.
	It is initially set to 1.0 and is reset to 1.0 whenever
	a new device is chosen. 

<HR><A NAME=FILL>
<DT>
FILL N [ANGLE [SEPN [PHASE]]] - Sets the fill area style to <CODE>N</CODE>.
<DD>
	<CODE>FILL</CODE> sets the fill type.  The fill type is
	identified by <CODE>N</CODE>, where <CODE>N</CODE> refers to
	one of the following:
	<PRE>
    1 : solid
    2 : hollow
    3 : hatched
    4 : cross-hatched
	</PRE>
	The fill type is initially set to 1 (solid) and is reset to 1
	whenever a new device is chosen.  If the fill type is hatched or
	cross-hatched, optional arguments can be supplied that affect
	the way the hatching will be done.  Up to three additional
	arguments may be provided.  <CODE>ANGLE</CODE> specifies the
	angle the hatch lines make with the horizontal, in degrees,
	increasing counterclockwise.  <CODE>SEPN</CODE> is the spacing
	of the hatch lines.  The unit spacing is 1 percent of the smaller
	of the height or width of the view surface.  <CODE>PHASE</CODE>
	is an offset value and is a real number between 0 and 1.  The
	hatch lines are displaced by this fraction of <CODE>SEPN</CODE>
	from a fixed reference.  Adjacent regions hatched with the
	same <CODE>PHASE</CODE> have contiguous hatch lines.  To hatch
	a region with alternating lines of two colors, fill the area
	twice: once with <CODE>PHASE=0.0</CODE> in one color and
	<CODE>PHASE=0.5</CODE> in the other color.  The default value
	for <CODE>ANGLE</CODE> is 45.0 degrees, <CODE>SEPN</CODE> is 1.0,
	and <CODE>PHASE</CODE> is 0.0.

<HR><A NAME=FIT>
<DT>
FIT style [N] [P1 P2 P3 ...] - Fits a curve to the (x,y) data pairs.
<DD>
	<CODE>FIT</CODE> will solve either for a linear fit of the form:
	"<CODE>y = a + bx</CODE>" by either the
	(a) Least Squares method ("lsqfit"); or
	(b) Criterion of Least Absolute Deviations ("medfit");
	or will solve for a Gaussian fit of the form
	"<CODE>y = a * exp(-((x - b)/c)**2)</CODE>".
	The data that comprise the "x" and the "y" in these equations are
	from the most recent call to the
	<A HREF=#XCOLUMN><CODE>XCOLUMN</CODE></A> and
	<A HREF=#YCOLUMN><CODE>YCOLUMN</CODE></A> commands.  For a
	linear fit, there is only one required argument, <CODE>STYLE</CODE>,
	which specifies which type of linear fit to perform.  The only
	possible values of <CODE>STYLE</CODE> are "lsqfit" and "medfit".
	The fit coefficients and the corresponding errors in the fit are
	displayed at the command line.  When using the "lsqfit" technique,
	estimates of the measurement errors may be included by reading the
	error values with the <A HREF=#ECOLUMN><CODE>ECOLUMN</CODE></A>
	command.  The optional parameters provide a way to assign the fits
	and the error estimates to User Variables (see the
	<A HREF=#SET><CODE>SET</CODE></A> command).  The "lsqfit" returns
	(in this order) the parameters "a" and "b" along with a chi-square
	for the fit and sigma errors on the estimates of "a" and "b".  If data
	in the most recent call to <A HREF=#ECOLUMN><CODE>ECOLUMN</CODE></A>
	was used to specify errors on the observation points, a "goodness
	of fit" is also returned.  If, however, no errors are specified,
	the correlation coefficient is returned in place of the "goodness
	of fit".  The "medfit" returns (in this order) the fit parameters
	"a" and "b" along with "absdev" (the mean absolute deviation in
	the "y" direction).  As an example, the command
	<CODE>fit lsqfit \0 \1</CODE>
	will only assign the fit parameters "a" and "b" to the User
	Variables \0 and \1 and the remaining items will be ignored (but
	still displayed at the command line).
	<P>
	For Gaussian fits, there are two required arguments:
	<CODE>STYLE</CODE> and <CODE>N</CODE>, where <CODE>STYLE</CODE>
	must be "gaussian" and <CODE>N</CODE> specifies the number of
	Gaussians to fit.  The number of optional parameters that
	may follow is governed by the value of <CODE>N</CODE>.  Optional
	arguments should follow in groups of 3's in the following order:
	amplitude, mean, and full width at half maximum.  Assignment of the
	output values follows as for linear fits.  If the <CODE>3*N+1</CODE>
	parameter is present, it is set to the chi-square of the fit.
	Setting any parameter to it's negative holds that value fixed
	during the fit.  For example:
	<PRE>
    set \1 1.0   # Guess at amplitude.
    set \2 5.0   # First guess at X-position of peak.
    set \3 -2.3  # Input FHWM and hold this fixed during fit.
    fit gaussian 1 \1 \2 \3 \4  # Fit 1 Gaussian and store fit
          # and chi-square in \1, \2, \3, and \4, respectively.
	</PRE>

<HR><A NAME=FONT>
<DT>
FONT N - Sets the font type to type <CODE>N</CODE>.
<DD>
	<CODE>FONT</CODE> sets the font type to one of the available fonts.
	The argument <CODE>N</CODE> selects the font used for subsequent
	text plotting.  The font type is initially set to 1 ("normal")
	and is reset to 1 whenever a new device is chosen.  Currently,
	four fonts are available and can be set by setting the parameter
	<CODE>N</CODE> to one of the following values:
	<PRE>
    1 : (default) a simple single-stoke font ("normal" font)
    2 : roman font
    3 : italic font
    4 : script font
	</PRE>
	The font may also be temporarily changed within a text string
	by using the escape sequences: \fn, \fr, \fi, or \fs.

<HR><A NAME=FREE>
<DT>
FREE item1 [item2 [...]] - releases items created with the <CODE>NEW</CODE> command.
<DD>
	<CODE>FREE</CODE> will remove the user specified variable(s) from
	WIP.  In addition, it releases (i.e. frees) the memory associated
	with each <CODE>ITEM</CODE>.  At least one <CODE>ITEM</CODE> is
	required and each <CODE>ITEM</CODE> listed specifies the name of
	the string variable, user variable, or vector (without any index)
	to release.

<HR><A NAME=GLOBE>
<DT>
GLOBE [nlong nlat] - draws a "globe" with nlong/nlat long/lat lines.
<DD>
	<CODE>GLOBE</CODE> draws a "globe" for use in Aitoff (Hammer)
	projections.  The "globe" that is drawn is an outline of the
	entire sky centered on the position (L, b) = (0, 0).  There
	are two optional parameters.  The first optional parameter
	<CODE>NLONG</CODE> represents the number of longitude lines
	that are drawn and the optional parameter <CODE>NLAT</CODE>
	represents the number of latitude lines that are drawn.  If
	these arguments are omitted, the values of <CODE>NLONG = 5</CODE>
	and <CODE>NLAT = 3</CODE> will be used for defaults.
	Also see the command <A HREF=#AITOFF><CODE>AITOFF</CODE></A>.

<HR><A NAME=HALFTONE>
<DT>
HALFTONE [MIN MAX [N [BLANK]]] - produces a halftone plot of an image.
<DD>
	<CODE>HALFTONE</CODE> draws a plot of the array read with the
	<A HREF=#IMAGE><CODE>IMAGE</CODE></A> command over the subimage
	region selected with the <A HREF=#SUBIMAGE><CODE>SUBIMAGE</CODE></A>
	command according to the coordinate transformation matrix
	specified by the <A HREF=#TRANSFER><CODE>TRANSFER</CODE></A>
	command and the current image transfer function (see the
	<A HREF=#ITF><CODE>ITF</CODE></A> command).
	If the device permits, colors
	can be preassigned to the halftone values with any of the
	<A HREF=#PALETTE><CODE>PALETTE,</CODE></A>
	<A HREF=#LOOKUP><CODE>LOOKUP,</CODE></A>
	<A HREF=#RGB><CODE>RGB,</CODE></A> or
	<A HREF=#HLS><CODE>HLS</CODE></A> commands
	(whenever a new device is selected, the palette is reset).
	The first two optional parameters allow the user to specify the
	intensity range for the plot.
	If the optional parameters <CODE>MIN</CODE> and
	<CODE>MAX</CODE> are not present,
	then <CODE>HALFTONE</CODE> will use the image minimum and maximum.
	The next optional parameter, <CODE>N</CODE>, if present,
	indicates that histogram equalization should be performed using
	<CODE>N</CODE> bins.  If this parameter is present and is set to
	zero, then the full number of color table entries will be used
	in the binning.  If this is not present, no equalization will be done.
	The last optional parameter <CODE>BLANK</CODE> is used to
	specify a blank value in the image to ignore when doing
	histogram equalization.  It defaults to -99.
	Histogram equalization is only applied to data within the
	current subimage region and within the minimum and maximum intensity.
	<P>
	NOTE: Histogram equalization changes the current LUT.  The LUT
	must be reset before calling the <CODE>HALFTONE</CODE> command again.

<HR><A NAME=HARDCOPY>
<DT>
HARDCOPY - causes a stored printer plot to be plotted.
<DD>
	If the current device selected is a hardcopy unit, then the
	<CODE>HARDCOPY</CODE> command will cause the output file to be
	closed and the file spooled to the printer.  How the file is
	printed is determined by the string variable "print" (see the
	command <A HREF=#SET><CODE>SET</CODE></A>).  By setting "print"
	to the string "ignore", automatic printing can be disabled.
	By default, the string variable "print" is set to "lpr"
	(use the command '<A HREF=#ECHO><CODE>echo print</CODE></A>'
	to display the current value of "print").

<HR><A NAME=HEADER>
<DT>
HEADER [XTYPE [YTYPE]] - loads header information of the current image.
<DD>
	<CODE>HEADER</CODE> sets the transfer function and World (User)
	limits of the box region for the current image.  This function
	makes use of the current <A HREF=#SUBIMAGE><CODE>SUBIMAGE</CODE></A>
	values to fix the World coordinate range.  If the optional
	arguments <CODE>XTYPE</CODE> and <CODE>YTYPE</CODE> are
	not present, they are set to the current values of the string
	variables "XHEADER" and "YHEADER", respectively.  If only one
	argument is present, both <CODE>XTYPE</CODE> and <CODE>YTYPE</CODE>
	are set to this value.  Currently, there are 7 possible values
	for <CODE>XTYPE</CODE> and <CODE>YTYPE</CODE> used to control
	the resulting header arrangement:
	<PRE>
    rd : Right ascension/declination (absolute coordinates).
    so : Arcsecond offset positions.
    mo : Arcminute offset positions.
    po : Pixel offset positions.
    px : Absolute pixel positions.
    gl : General linear coordinates.
    go : General linear offset coordinates.
	</PRE>
	All offset arguments are relative to the value of the user variables
	CRVALX and CRVALY.  The option "rd" includes a 15*cos(dec) factor
	applied to the right ascension term.  Incorrect options are treated
	as if "px" was selected.  The options "gl" and "go" apply no
	additional scaling factor to the header variables.  If the image
	headers CRVAL, CRPIX, and CDELT are not present for that axis or
	are all set to 0, the coordinate transfer function will be set to
	its default value (see the command
	<A HREF=#TRANSFER><CODE>TRANSFER</CODE></A>) and the world limits
	to the image range in pixels.
	NOTE: Currently, the conversion is
	fixed to be from radians to arcseconds (or seconds of time) for
	MIRIAD images; degrees to arcseconds for FITS images; and no
	conversion for BASIC images.

<HR><A NAME=HELP>
<DT>
HELP xxx - prints an explanation of the command <CODE>xxx</CODE>.
<DD>
	Without any argument, <CODE>HELP</CODE> prints a one line
	description of every command.  With the name of a command as
	an argument, <CODE>HELP</CODE> prints more information on that
	particular command.  The command '<CODE>help ?</CODE>' will
	provide a list of all available command and macro names.

<HR><A NAME=HI2D>
<DT>
HI2D bias [slant [center]] - draws a histogram of the data read by <CODE>IMAGE</CODE>.
<DD>
	<CODE>HI2D</CODE> plots a series of cross-sections of the current
	image (see <A HREF=#IMAGE><CODE>IMAGE</CODE></A>).  Each
	cross-section is plotted as a hidden line histogram.  The one
	required argument, <CODE>BIAS</CODE>, specifies a bias value
	to be applied to each successive cross-section (in order to
	raise it above the previous one).  <CODE>BIAS</CODE> is in the
	same units as the image data.  If the optional parameter
	<CODE>SLANT</CODE> is present, it is an integer value offset to
	be applied to each successive cross-section (in the X-direction).
	If <CODE>SLANT</CODE> is greater than 0, then the plot slants to
	the right; less than 0, it slants to the left.  The default value
	for <CODE>SLANT</CODE> is 0 (no slant).  If the optional argument
	<CODE>CENTER</CODE> is present and equal to 1 (the default), then
	the individual X values denote the center location of the bin;
	otherwise, the X values denote the lower edge (in X) of the bin.

<HR><A NAME=HISTOGRAM>
<DT>
HISTOGRAM [MIN MAX [N]] - draws a histogram of the data read by <CODE>XCOLUMN</CODE>.
<DD>
	<CODE>HISTOGRAM</CODE> connects the coordinates read by
	<A HREF=#XCOLUMN><CODE>XCOLUMN</CODE></A> as a histogram.  If the
	optional arguments <CODE>MIN</CODE> and <CODE>MAX</CODE> are present,
	they are used to specify the minimum and maximum of the histogram.
	Furthermore, an optional parameter <CODE>N</CODE> can specify the
	number of bins to divide up the histogram.  At present, <CODE>N</CODE>
	may not exceed 200.  By default, <CODE>MIN</CODE> and <CODE>MAX</CODE>
	are set to the minimum and maximum of the data read by
	<A HREF=#XCOLUMN><CODE>XCOLUMN</CODE></A> and <CODE>N</CODE> is set
	to 5.  The user should set the limits (see the command
	<A HREF=#LIMITS><CODE>LIMITS</CODE></A>) of the box before calling
	the <CODE>HISTOGRAM</CODE> command.

<HR><A NAME=HLS>
<DT>
HLS K hue light sat - sets the color representation using the HLS system.
<DD>
	<CODE>HLS</CODE> sets the color representation of the color index
	<CODE>K</CODE> to the provided values of Hue-Lightness-Saturation.
	<CODE>HUE</CODE> is represented by an angle in degrees, with red
	at 120, green at 240, and blue at 0 (or 360).  <CODE>LIGHT</CODE>
	ranges from 0 to 1, with black at lightness of 0 and white at
	lightness 1.  <CODE>SAT</CODE> ranges from 0 (gray) to 1 (pure color).
	<CODE>HUE</CODE> is irrelevant when <CODE>SAT</CODE> is 0.  If
	<CODE>K</CODE> is greater than the maximum color index of the device,
	then this call will be ignored (use the command
	'<A HREF=#ECHO><CODE>echo cmin cmax</CODE></A>'
	to display the current color range).

<HR><A NAME=ID>
<DT>
ID - puts an identification label at the bottom of a plot.
<DD>
	<CODE>ID</CODE> puts the name of the user and the current date
	and time at the bottom of the plot region.  If the environment
	variable PGPLOT_IDENT is defined and the current device is a hardcopy
	device (the command '<A HREF=#ECHO><CODE>echo hardcopy</CODE></A>'
	will return 1 if this is true), then this call is automatically
	done when the device is finally closed (see the command
	<A HREF=#HARDCOPY><CODE>HARDCOPY</CODE></A>).

<HR><A NAME=IF>
<DT>
IF EXPRESSION XXX [...] - executes <CODE>xxx</CODE> if <CODE>EXPRESSION</CODE> is true.
<DD>
	<CODE>IF</CODE> allows conditional execution of macros and commands.
	The macro or command <CODE>XXX</CODE> is executed only if the
	conditional test of the expression is TRUE (non-zero).  The
	<CODE>EXPRESSION</CODE> may be a simple <CODE>(X OP Y)</CODE>
	expression where the parameters <CODE>X</CODE> and <CODE>Y</CODE>
	may be any numerical value or user variable (see the
	<A HREF=#SET><CODE>SET</CODE></A> command) and <CODE>OP</CODE> is
	the condition to test and is defined TRUE for the following:
	<PRE>
    op : Test
------------------------------------------------------
    == : X equal to Y;
    != : X not equal to Y;
    <  : X less than Y;
    >  : X greater than Y;
    <= : X less than or equal to Y;
    >= : X greater than or equal to Y;
    && : X and Y are non-zero;
    || : X or Y are non-zero;
    ^  : X or Y, but not both, are non-zero.
	</PRE>
	If the test consists of more than just a simple <CODE>(X op Y)</CODE>
	test, then the entire expression MUST be enclosed by parenthesis.
	As an example,
	<PRE>
    if ((x1 < x2) && (y1 < y2)) limits x2 x1 y1 y2
	</PRE>
	will only execute the <A HREF=#LIMITS><CODE>LIMITS</CODE></A>
	command if X1 is less than X2 AND Y1 is less than Y2.  Note also
	that this example illustrates that the usual parameters/arguments
	may follow the macro or command name.  See the command
	<A HREF=#SET><CODE>SET</CODE></A> for a more detailed description
	of <CODE>EXPRESSION</CODE>.

<HR><A NAME=IMAGE>
<DT>
IMAGE fspec [plane [badpixel [order]]] - reads in an image from file "fspec".
<DD>
	<CODE>IMAGE</CODE> loads an array into WIP for use with plotting
	commands like <A HREF=#CONTOUR><CODE>CONTOUR</CODE></A> and
	<A HREF=#HALFTONE><CODE>HALFTONE</CODE></A>.
	Currently, WIP tests the file to determine its file type (i.e.
	FITS, MIRIAD, BASIC, etc.) and then loads it into available memory.
	The optional parameter <CODE>PLANE</CODE> selects which plane of
	a data cube to load.  The default value for <CODE>PLANE</CODE> is 1.
	<CODE>IMAGE</CODE> trys to read available header information which
	will be useful for setting the coordinate transformation matrix
	(see the command <A HREF=#TRANSFER><CODE>TRANSFER</CODE></A>)
	after the image is read into memory (with the command
	<A HREF=#HEADER><CODE>HEADER</CODE></A>).  The current subimage
	values (see the command <A HREF=#SUBIMAGE><CODE>SUBIMAGE</CODE></A>)
	are reset to the full array size of the new image and the user
	variables CRVALX/Y, CRPIXX/Y, and CDELTX/Y may be set to the newly
	read header values (if they can be read).  The subimage values,
	6 user variables, and the transformation matrix are used to
	determine where the plot will appear.  In order to correctly
	set the transformation matrix with the command
	<A HREF=#HEADER><CODE>HEADER</CODE></A>, the values of CRVAL,
	CRPIX, CDELT, and CTYPE must be present for the first two axes.
	The next optional parameter <CODE>BADPIXEL</CODE> may be
	specified (after <CODE>PLANE</CODE>) as the value for blanked or
	masked pixels.
	By default, these will be set to -99.
	This value may be helpful for use
	with the <A HREF=#CONTOUR><CODE>CONTOUR</CODE></A> command.
	The final optional parameter <CODE>ORDER</CODE> permits the
	image to be smoothed on reading.  The parameter <CODE>ORDER</CODE>
	may take on a value of 0, 1, or 2 and represents the order of
	the fit.  A value of 0 (or if it is not provided) means to apply
	no smoothing (the default).
	<P>
	NOTE: Current image types include:
	basic, Miriad, and FITS.
	BASIC image name syntax is: filename`columnsXrows[`offset] where
	rows and columns are in pixels and the optional offset is in
	bytes from the beginning of the file.  The character 'X' determines
	the size of each element and is 'b' for unsigned bytes; 's',
	signed 2 byte integers; 'l', signed 4 byte integers; 'r', 4 byte
	floating point; 'd', 8 byte double precision floating point.

<HR><A NAME=INITIALIZE>
<DT>
INITIALIZE V N EXPRESSION - Sets <CODE>V</CODE> to the result of <CODE>EXPRESSION</CODE>.
<DD>
	<CODE>INITIALIZE</CODE> evaluates the <CODE>EXPRESSION</CODE>
	(see the command <A HREF=#SET><CODE>SET</CODE></A>) and sets
	every element of the array (vector) user variable <CODE>V</CODE>
	to the result.  If <CODE>N</CODE> is less than zero, the current
	size of the vector variable <CODE>V</CODE> is used.  If
	<CODE>N</CODE> is zero, no expression is needed and the vector
	is set to size zero (empty).  If <CODE>N</CODE> is greater than
	zero and an expression is present, the first <CODE>N</CODE>
	elements of the vector are set to the result of the expression.
	If the expression is missing, the size of the vector will be reset.

<HR><A NAME=INPUT>
<DT>
INPUT fspec - reads plot commands from file "fspec" and executes them.
<DD>
	<CODE>INPUT</CODE> reads plot commands from a file and treats
	them exactly as if they were typed:  commands are executed or
	can be inserted or macros may be defined.  Unlike interactive
	mode, individual commands are NOT saved in the command buffer
	(see <A HREF=#BUFFER><CODE>BUFFER</CODE></A>).
	NOTE: Macros are not redefined so care should be taken to
	insure that the proper macro is being executed.

<HR><A NAME=INSERT>
<DT>
INSERT [N [macro]] - commands are inserted before command <CODE>N</CODE> in a macro.
<DD>
	<CODE>INSERT</CODE> enters insert mode (indicated by the
	<CODE>INSERT</CODE>&gt; prompt) and
	subsequent commands are inserted, without execution, into the
	named macro until an <A HREF=#END><CODE>END</CODE></A> command
	is encountered.  Commands are numbered as shown by the command
	<A HREF=#LIST><CODE>LIST</CODE></A>.  Without any arguments,
	<CODE>INSERT</CODE> inserts commands at the end of the command
	buffer.  If <CODE>N</CODE> is present but no macro name is given,
	then commands are inserted before line <CODE>N</CODE> in the
	command macro (<A HREF=#BUFFER><CODE>BUFFER</CODE></A>).  With
	the <CODE>MACRO</CODE> argument, <CODE>INSERT</CODE> starts
	inserting just before line number <CODE>N</CODE> in macro
	<CODE>MACRO</CODE>.
	NOTE: The command buffer macro name is
	<A HREF=#BUFFER><CODE>BUFFER</CODE></A>.

<HR><A NAME=ITF>
<DT>
ITF N - sets the current image transfer function to <CODE>N</CODE>.
<DD>
	<CODE>ITF</CODE> sets an internal parameter such that all subsequent
	image drawing functions (<A HREF=#IMAGE><CODE>IMAGE</CODE></A> and
	<A HREF=#WEDGE><CODE>WEDGE</CODE></A>) act on the transformed image.
	Acceptable values for <CODE>N</CODE> and their meanings follow:
	<PRE>
    0 : linear
    1 : logarithmic
    2 : square-root
	</PRE>
	The initial value of the image transfer function is 0 and is
	reset to 0 whenever a new device is selected.

<HR><A NAME=LABEL>
<DT>
LABEL STR - writes the string <CODE>STR</CODE> at the current cursor position.
<DD>
	<CODE>LABEL</CODE> writes the string <CODE>STR</CODE> (which
	starts at the current cursor position and continues to the
	last non-space character in the label string).  The string's 
	size and the angle it is displayed are controlled by the most
	recent call to the commands <A HREF=#EXPAND><CODE>EXPAND</CODE></A>
	and <A HREF=#ANGLE><CODE>ANGLE</CODE></A>.

<HR><A NAME=LCUR>
<DT>
LCUR [NPTS] - draws a line using the cursor.
<DD>
	<CODE>LCUR</CODE> allows a polyline to be drawn using the cursor.
	Vertices may be added or removed by locating the cursor at a
	desired point and selecting the key 'A' (Add) or 'D' (Delete) or
	may terminate the command with the key 'X' (eXit).  The vertices
	are stored in the <A HREF=#XCOLUMN><CODE>XCOLUMN</CODE></A> and
	<A HREF=#YCOLUMN><CODE>YCOLUMN</CODE></A> arrays and may be
	redrawn by subsequent calls to this command.  If the optional
	parameter <CODE>NPTS</CODE> is given, it is used as the number
	of points in the previous call to this command.  To initialize
	the command, set <CODE>NPTS</CODE> to 0.  By default,
	<CODE>NPTS</CODE> is set to the current value of the number of
	points in the most recent call to the
	<A HREF=#XCOLUMN><CODE>XCOLUMN</CODE></A> and
	<A HREF=#YCOLUMN><CODE>YCOLUMN</CODE></A> commands.

<HR><A NAME=LDEV>
<DT>
LDEV - lists the devices currently available.
<DD>
	<CODE>LDEV</CODE> lists (to standard output) the names of
	the installed devices known to the current version of WIP.

<HR><A NAME=LEVELS>
<DT>
LEVELS [L1 L2 ...] - sets the contour levels for a contour plot.
<DD>
	<CODE>LEVELS</CODE> establishes the contour values for use in a
	contour plot.  Currently, the maximum number of levels available
	is 40.  This array of levels may be augmented if a call to the
	command <A HREF=#SLEVEL><CODE>SLEVEL</CODE></A> is done prior to
	a call to the <A HREF=#CONTOUR><CODE>CONTOUR</CODE></A> command.

<HR><A NAME=LIMITS>
<DT>
LIMITS [X1 X2 Y1 Y2] - sets the World limits of the plot.
<DD>
	<CODE>LIMITS</CODE> sets the World (user) coordinate system of
	the plot window.  All coordinates from
	<A HREF=#XCOLUMN><CODE>XCOLUMN</CODE></A>,
	<A HREF=#YCOLUMN><CODE>YCOLUMN</CODE></A>,
	<A HREF=#MOVE><CODE>MOVE</CODE></A>,
	<A HREF=#DRAW><CODE>DRAW</CODE></A>, etc.,
	are relative to these limits.  If no
	arguments are present, then the data from the most recent
	<A HREF=#XCOLUMN><CODE>XCOLUMN</CODE></A> and
	<A HREF=#YCOLUMN><CODE>YCOLUMN</CODE></A>
	commands are used to set the limits.

<HR><A NAME=LINES>
<DT>
LINES L1 L2 - limits the <CODE>X, Y, E,</CODE> and <CODE>PCOLUMN</CODE> file read to lines <CODE>L1-L2</CODE>.
<DD>
	<CODE>LINES</CODE> sets the range of lines read from the
	data file by the commands
	<A HREF=#XCOLUMN><CODE>XCOLUMN</CODE></A>,
	<A HREF=#YCOLUMN><CODE>YCOLUMN</CODE></A>,
	<A HREF=#ECOLUMN><CODE>ECOLUMN</CODE></A>, and
	<A HREF=#PCOLUMN><CODE>PCOLUMN</CODE></A>.
	This is useful to avoid non-data lines.  The argument <CODE>L1</CODE>
	is also used to specify the line that the command
	<A HREF=#STRING><CODE>STRING</CODE></A>
	will use to extract a string variable.

<HR><A NAME=LIST>
<DT>
LIST [L1 [L2 [xxx]]] - lists the commands of macro <CODE>xxx</CODE>.
<DD>
	<CODE>LIST</CODE> lists the commands of the named macro.  If there
	are no arguments, the macro <A HREF=#BUFFER><CODE>BUFFER</CODE></A>
	(the command buffer) is listed.  If one argument is given, then
	the commands from <CODE>L1</CODE> to the end of the macro
	<A HREF=#BUFFER><CODE>BUFFER</CODE></A> is shown.  With two
	arguments, lines <CODE>L1</CODE> to <CODE>L2</CODE> are shown.
	If the macro name <CODE>xxx</CODE> is also entered as the third
	argument, then that macro is listed over the inclusive line range
	of <CODE>L1</CODE> to <CODE>L2</CODE>.  The numbers assigned to
	commands by <CODE>LIST</CODE> should be the ranges used by calls
	to the commands <A HREF=#DELETE><CODE>DELETE</CODE></A> and
	<A HREF=#INSERT><CODE>INSERT</CODE></A>.
	NOTE: The command buffer macro name is
	<A HREF=#BUFFER><CODE>BUFFER</CODE></A>.

<HR><A NAME=LOGARITHM>
<DT>
LOGARITHM name [scale] - takes the scaled logarithm of vectors and images.
<DD>
	<CODE>LOGARITHM</CODE> takes the base 10 logarithm of either
	vectors or images.  The argument <CODE>NAME</CODE> specifies
	either a vector name (e.g. "x") or an image name (currently,
	only the name "image" is acceptable).  If the value of the
	vector or image is greater than zero, the logarithm is taken
	and this result is multiplied by the value of <CODE>SCALE</CODE>.
	If <CODE>SCALE</CODE> is not present, a value of 1.0 is used.
	<CODE>SCALE</CODE> may be any valid numeric expression (see
	the command <A HREF=#SET><CODE>SET</CODE></A>).
	<CODE>LOGARITHM</CODE> uses -50 for the logarithm
	of 0 or negative numbers and does not scale these values.

<HR><A NAME=LOOKUP>
<DT>
LOOKUP [N] - loads a RGB color lookup table.
<DD>
	<CODE>LOOKUP</CODE> uses the values read by the most recent call to
	the commands <A HREF=#XCOLUMN><CODE>XCOLUMN</CODE></A>,
	<A HREF=#YCOLUMN><CODE>YCOLUMN</CODE></A>,
	<A HREF=#ECOLUMN><CODE>ECOLUMN</CODE></A>, and
	<A HREF=#PCOLUMN><CODE>PCOLUMN</CODE></A>
	to load values into the color lookup table.  This is a more
	compact (and correct) method for loading an entire table than
	the single color index commands <A HREF=#RGB><CODE>RGB</CODE></A> and
	<A HREF=#HLS><CODE>HLS</CODE></A>.  All values in the
	<A HREF=#XCOLUMN><CODE>XCOLUMN</CODE></A>,
	<A HREF=#YCOLUMN><CODE>YCOLUMN</CODE></A>,
	<A HREF=#ECOLUMN><CODE>ECOLUMN</CODE></A>, and
	<A HREF=#PCOLUMN><CODE>PCOLUMN</CODE></A> data must be between 0 and 1.
	The <A HREF=#XCOLUMN><CODE>XCOLUMN</CODE></A> data correspond to the
	red values, <A HREF=#YCOLUMN><CODE>YCOLUMN</CODE></A> to green, and
	<A HREF=#ECOLUMN><CODE>ECOLUMN</CODE></A> to blue.  The
	<A HREF=#PCOLUMN><CODE>PCOLUMN</CODE></A> data identifies the
	ramp intensity level for the corresponding RGB values.  Colors
	on the ramp are linearly interpolated from neighboring levels.
	If the optional parameter <CODE>N</CODE> is present and is negative,
	then the resulting lookup table will be flipped.  By default (if the
	optional argument <CODE>N</CODE> is missing or is positive), the
	RGB table is loaded directly (not flipped).

<HR><A NAME=LOOP>
<DT>
LOOP count xxx [p1 p2 ...] - executes the macro <CODE>xxx COUNT</CODE> times.
<DD>
	<CODE>LOOP</CODE> allows for multiple execution of a macro.
	The macro <CODE>xxx</CODE> is executed <CODE>COUNT</CODE> times.
	The optional parameters <CODE>P1, P2, ...</CODE>
	are any arguments needed to be passed to the macro (see the
	<A HREF=#DEFINE><CODE>DEFINE</CODE></A> command).  The arguments
	<CODE>P1, P2, ...</CODE> passed to the macro are NOT altered
	during each execution of the loop.

<HR><A NAME=LSTYLE>
<DT>
LSTYLE N - sets the current line style to <CODE>N</CODE>.
<DD>
	All subsequent lines except for points and characters are drawn
	with line style <CODE>N</CODE>, where <CODE>N</CODE> refers to
	lines of the following form:
	<PRE>
    1 : solid
    2 : dashed
    3 : dot - dash - dot - dash
    4 : dotted
    5 : dash - dot - dot - dot
	</PRE>
	If <CODE>N</CODE> is some number other than one of the line style
	values above, it is set to 1.  Also, the line style is set to 1
	initially and is reset to 1 whenever a new device is selected.

<HR><A NAME=LWIDTH>
<DT>
LWIDTH N - sets the current line width attribute to <CODE>N</CODE>.
<DD>
	<CODE>LWIDTH</CODE> sets an internal parameter such that all
	subsequent lines, text, and graph markers are drawn with lines
	of width <CODE>N</CODE>.  This number may range from the default
	of 1 to a maximum of 201.  Any other value is set to 1.  The
	actual thickness drawn is device dependent, but is approximately
	equal to 0.005 inches times the value of <CODE>N</CODE>.  The
	initial value of the line width is 1 and is reset to 1 whenever
	a new device is selected.

<HR><A NAME=MACRO>
<DT>
MACRO filename - Used to define macros using an external file.
<DD>
	<CODE>MACRO</CODE> will read the commands from the file
	<CODE>filename</CODE> and define macros exactly as if they were
	typed interactively.  This allows macros saved from previous WIP
	sessions (using the <A HREF=#WRITE><CODE>WRITE</CODE></A> command)
	to be read into WIP and defined.  Only macro definitions are allowed.

<HR><A NAME=MINMAX>
<DT>
MINMAX [N] - list the maximum and minimum values of the current image.
<DD>
	<CODE>MINMAX</CODE> list the maximum and minimum values of the
	image that has been read by the <A HREF=#IMAGE><CODE>IMAGE</CODE></A>
	command.  This command may be useful for setting contour levels,
	or adjusting a halftone plot.  The minimum and maximum of the
	current array are stored in the user variables IMMIN and IMMAX.
	(The minimum and maximum may also be displayed using the command
	'<A HREF=#ECHO><CODE>echo immin immax</CODE></A>'.)  If the
	optional parameter <CODE>N</CODE> is present and set to any value
	greater than 0, then the minimum and maximum are explicitly
	recalculated.  This may be useful if the minimum and maximum has
	been found and then the logarithm of the image is computed.

<HR><A NAME=MOVE>
<DT>
MOVE X Y - sets the current World (user) position to <CODE>(x,y).</CODE>
<DD>
	<CODE>MOVE</CODE> sets the current World (user) location to (x,y)
	without drawing any lines or text.  This simply provides a way to
	specify the current position of the cursor.

<HR><A NAME=MTEXT>
<DT>
MTEXT side disp coord just str - writes the string <CODE>STR</CODE> relative to <CODE>SIDE</CODE>.
<DD>
	<CODE>MTEXT</CODE> writes text at the position specified relative
	to a viewport side.  The parameter <CODE>SIDE</CODE> specifies
	which side of the viewport to place the string.  <CODE>SIDE</CODE>
	must be either `B', `L', `T', or `R' to specify the
	Bottom, Left, Top, or Right side of the viewport.
	Additionally, if <CODE>SIDE</CODE> is `LV' or `RV', then the string
	will be written perpendicular to the frame rather than parallel
	to it.  The parameter <CODE>DISP</CODE> specifies the displacement
	from the frame edge (in character height units).  If <CODE>DISP</CODE>
	is negative, then the string is written inside the viewport.  The
	parameter <CODE>COORD</CODE> is the location of the string as a
	fractional length of the specified edge of the viewport.
	<CODE>JUST</CODE> specifies how to justify the string relative to
	<CODE>COORD</CODE> with <CODE>JUST = 0</CODE> meaning left justified;
	0.5, centered; and 1.0, right justified.  The text <CODE>STR</CODE>
	is the (multiword) string to display.
	As an example, "good" values used to place a string along the X
	axis, the Y axis, and the top of the plot box are:
	<PRE>
    mtext B 3.2 0.5 0.5 Xaxis label string
    mtext L 2.2 0.5 0.5 Yaxis label string
    mtext T 2.0 0.5 0.5 Top label string
	</PRE>
	The first two examples above correspond to the syntax used by
	the commands <A HREF=#XLABEL><CODE>XLABEL</CODE></A> and
	<A HREF=#YLABEL><CODE>YLABEL</CODE></A>.

<HR><A NAME=NCURSE>
<DT>
NCURSE [n] - marks a set of points using the cursor.
<DD>
	<CODE>NCURSE</CODE> allows a set of points to be drawn using the
	cursor.  Points may be added or removed by locating the cursor
	at a desired point and selecting the key `A' (Add) or `D' (Delete)
	or may terminate the command with the key `X' (eXit).  The points
	are stored in the <A HREF=#XCOLUMN><CODE>XCOLUMN</CODE></A> and
	<A HREF=#YCOLUMN><CODE>YCOLUMN</CODE></A> arrays and may be redrawn
	by subsequent calls to this command.  If the optional parameter
	<CODE>N</CODE> is given, it is used as the number of points in the
	previous call to this command.  To initialize the command, set
	<CODE>N</CODE> to 0.  By default, <CODE>N</CODE> is set to the
	current number of points in the most recent call to the
	<A HREF=#XCOLUMN><CODE>XCOLUMN</CODE></A> and
	<A HREF=#YCOLUMN><CODE>YCOLUMN</CODE></A> commands.  On return,
	the cursor points are returned in increasing order of X (unlike
	the command <A HREF=#OLIN><CODE>OLIN</CODE></A>).

<HR><A NAME=NEW>
<DT>
NEW item [...] - creates a new string variable, user variable, or vector.
<DD>
	<CODE>NEW</CODE> creates a new variable.  If more than one
	<CODE>ITEM</CODE> is present on the command line, then each will
	represent the name of a new variable to be created.  Each
	<CODE>ITEM</CODE> name must be unique and may not correspond
	to a pre-defined variable or any other defined variable.  Once
	these variables are created, they may be used just like the
	pre-defined variables:  the command <A HREF=#SET><CODE>SET</CODE></A>
	will permit the value to be set/changed and the command
	<A HREF=#ECHO><CODE>ECHO</CODE></A> may be used to display the
	current value.  Use the command <A HREF=#FREE><CODE>FREE</CODE></A>
	to release the memory associated with each <CODE>ITEM</CODE>.  If
	the <CODE>ITEM</CODE> is enclosed in double quotes
	(e.g. <CODE>"ITEM"</CODE>), a string variable is created.  If the
	<CODE>ITEM</CODE> has an optional size argument included, a vector
	is created.  Note that when defining a vector variable, the
	maximum number of elements should be enclosed in brackets and
	appear affixed to the end of the <CODE>ITEM</CODE> name
	(e.g.  <CODE>array[10]</CODE>).  If <CODE>ITEM</CODE> is not a
	vector or string variable, then <CODE>ITEM</CODE> represents a
	simple user variable.

<HR><A NAME=OLIN>
<DT>
OLIN [n] - marks a set of points using the cursor.
<DD>
	<CODE>OLIN</CODE> allows a set of points to be drawn using the
	cursor.  Points may be added or removed by locating the cursor
	at a desired point and selecting the key `A' (Add) or `D' (Delete)
	or may terminate the command with the key `X' (eXit).  The points
	are stored in the <A HREF=#XCOLUMN><CODE>XCOLUMN</CODE></A> and
	<A HREF=#YCOLUMN><CODE>YCOLUMN</CODE></A> arrays and may be
	redrawn by subsequent calls to this command.  If the optional
	parameter <CODE>N</CODE> is given, it is used as the number of
	points in the previous call to this command.  To initialize the
	command, set <CODE>N</CODE> to 0.  By default, <CODE>N</CODE>
	is set to the current number of points in the most recent call
	to the <A HREF=#XCOLUMN><CODE>XCOLUMN</CODE></A>
	and <A HREF=#YCOLUMN><CODE>YCOLUMN</CODE></A> commands.  On
	return, the cursor points are returned in the order they were
	added (unlike the command <A HREF=#NCURSE><CODE>NCURSE</CODE></A>).

<HR><A NAME=PALETTE>
<DT>
PALETTE K [LEV] - Sets the color palette to entry <CODE>K</CODE>.
<DD>
	<CODE>PALETTE</CODE> sets the current image color array to a
	known look up table (LUT).  The argument <CODE>K</CODE> specifies
	which palette is loaded and legal values are displayed below.
	If <CODE>K</CODE> is not associated with a legal palette value,
	a warning is issued and the palette remains unchanged.  The
	value of <CODE>K</CODE> can be negative in which case, the
	absolute value of <CODE>K</CODE> is used and the sense of the LUT
	is reversed.
	NOTE: Not all devices have color LUTs so this
	command may have no effect on them.  Whenever a device is
	changed, the palette resets to the default.
	<PRE>
    K   Palette set to...
-------------------------------
    0 : Background to Foreground (default).
    1 : Gray scale.
    2 : A rainbow.
    3 : Heat scale.
    4 : IRAF scale.
    5 : AIPS scale.
    6 : PGPLOT scale.
    7 : Saoimage A scale.
    8 : Saoimage BB scale.
    9 : Saoimage HE scale.
   10 : Saoimage I8 scale.
   11 : DS scale.
   12 : Cyclic scale.
	</PRE>
	The optional parameter <CODE>LEV</CODE> may be provided to
	specify the number of levels into which to squeeze the palette.
	If <CODE>LEV</CODE> is not present (or is negative), then the
	number of levels is not changed.  If <CODE>LEV</CODE> is present
	and set to 0, the number is reset to the maximum range for the
	current device.  If <CODE>LEV</CODE> is present and positive,
	the number of color levels is set to this value.

<HR><A NAME=PANEL>
<DT>
PANEL nx ny k - Sets the plot location to a subpanel.
<DD>
	<CODE>PANEL</CODE> makes the current plot location panel
	<CODE>K</CODE>, where there are <CODE>NX</CODE> panels across
	and <CODE>NY</CODE> panels vertically.  The direction the panels
	run depends on the sign of <CODE>K</CODE>.  If <CODE>K</CODE> is
	positive, then the panels are counted across from the lower left
	and upwards (<CODE>K = 1</CODE> specifies the first panel).  If
	<CODE>K</CODE> is negative, then the panels are counted across
	from the upper left and downwards.  <CODE>PANEL</CODE> retains a
	copy of the <A HREF=#VIEWPORT><CODE>VIEWPORT</CODE></A> coordinate
	system that is present when <CODE>PANEL</CODE> is first called.
	To reset to those values, use the command <CODE>panel 1 1 1</CODE>;
	this will reset the plot location to the previous
	<A HREF=#VIEWPORT><CODE>VIEWPORT</CODE></A> area (this is also
	done automatically every time the current device is changed).
	Be sure to also call <CODE>panel 1 1 1</CODE> prior to changing
	the plot location with any other
	<A HREF=#VIEWPORT><CODE>VIEWPORT</CODE></A> command!  The
	separation between adjacent panels is controlled by the value of
	<A HREF=#EXPAND><CODE>EXPAND</CODE></A> as well as by the user
	variables XSUBMAR and YSUBMAR (see the
	<A HREF=#SUBMARGIN><CODE>SUBMARGIN</CODE></A> command for more
	details on these variables).  If either <CODE>NX</CODE> or
	<CODE>NY</CODE> is negative, then the gap between the subpanels in
	that direction is removed.  The absolute value of <CODE>NX</CODE>
	and <CODE>NY</CODE> is always used to determine the number of subpanels.

<HR><A NAME=PAPER>
<DT>
PAPER width aspect - change the size of the view surface.
<DD>
	<CODE>PAPER</CODE> changes the size of the view surface to a
	specified width (inches) and aspect ratio (height/width).  This
	command should only be used immediately after a call to the
	<A HREF=#DEVICE><CODE>DEVICE</CODE></A> command.

<HR><A NAME=PCOLUMN>
<DT>
PCOLUMN n - reads point type data from column <CODE>N</CODE> of the current data file.
<DD>
	<CODE>PCOLUMN</CODE> reads column <CODE>N</CODE> from the data
	file as the values of a symbol array.  The data read by
	<CODE>PCOLUMN</CODE> is used by the commands
	<A HREF=#POINTS><CODE>POINTS</CODE></A> and
	<A HREF=#BIN><CODE>BIN</CODE></A> (as well as others) to
	represent additional data information.

<HR><A NAME=PHARD>
<DT>
PHARD devname [xxx args] - spool a plot to an alternative device.
<DD>
	<CODE>PHARD</CODE> allows the commands in macro <CODE>xxx</CODE>
	(or the commands in the command buffer) to be spooled to an
	alternative device.  The one required argument <CODE>DEVNAME</CODE>
	is the name of the alternative device (see the command
	<A HREF=#DEVICE><CODE>DEVICE</CODE></A>).  If no other
	arguments are present, then the commands in the command buffer
	are played back (see the command
	<A HREF=#PLAYBACK><CODE>PLAYBACK</CODE></A>);
	otherwise, the named macro (<CODE>xxx</CODE>) is called (with
	any further arguments on the command line used as input arguments
	to the named macro).  If the named device is a hardcopy device, the
	<A HREF=#HARDCOPY><CODE>HARDCOPY</CODE></A> command is automatically
	executed at the end of the playback.  Finally, the current device
	is set to the device that was present when this command was called.

<HR><A NAME=PLAYBACK>
<DT>
PLAYBACK [xxx args] - replay macro <CODE>xxx</CODE> or commands in the command buffer.
<DD>
	<CODE>PLAYBACK</CODE> executes the macro <CODE>xxx</CODE> without
	storing the commands in the command buffer.  With an argument of
	<A HREF=#BUFFER><CODE>BUFFER</CODE></A> (or no argument),
	<CODE>PLAYBACK</CODE> replays all commands in the command buffer.
	The optional arguments (<CODE>args</CODE>) are passed to the macro.

<HR><A NAME=PLOTFIT>
<DT>
PLOTFIT [X1 X2 [STEP]] - draws a plot of the most recent fit.
<DD>
	<CODE>PLOTFIT</CODE> uses the parameters from the most recent
	call to <A HREF=#FIT><CODE>FIT</CODE></A> to display a smooth
	curve.  The optional parameters <CODE>X1</CODE> and <CODE>X2</CODE>
	can be used to limit the X-range to be plotted.  They default to
	the full range of the window.  The other optional parameter is
	the <CODE>STEP</CODE> size of the plot.  It provides a means to
	make the plot smoother or more coarse.  The <CODE>STEP</CODE>
	defaults to the mean of the difference between data in the
	most recent call to the <A HREF=#XCOLUMN><CODE>XCOLUMN</CODE></A>
	command.

<HR><A NAME=POINTS>
<DT>
POINTS [K] - draws points of the current style at each <CODE>(x,y).</CODE>
<DD>
	<CODE>POINTS</CODE> draws points of the current style (see the command
	<A HREF=#SYMBOL><CODE>SYMBOL</CODE></A>) and size (see the command
	<A HREF=#EXPAND><CODE>EXPAND</CODE></A>) at the coordinates read by the
	commands <A HREF=#XCOLUMN><CODE>XCOLUMN</CODE></A> and
	<A HREF=#YCOLUMN><CODE>YCOLUMN</CODE></A>.
	The command <A HREF=#PCOLUMN><CODE>PCOLUMN</CODE></A> may be used to
	specify a <A HREF=#SYMBOL><CODE>SYMBOL</CODE></A> corresponding to
	each (x,y) point plotted.  The format of the data read by the
	<A HREF=#PCOLUMN><CODE>PCOLUMN</CODE></A> command is the same as the
	argument for the <A HREF=#SYMBOL><CODE>SYMBOL</CODE></A> command,
	except that an optional fractional part may be added to each symbol.
	If the entry has a fractional part, it is treated as an expansion
	factor (fractional parts less than 0.01 gives the default expansion).
	For example, an entry of 93.5 in a point column is the same as
	<A HREF=#SYMBOL><CODE>symbol 93</CODE></A>,
	<A HREF=#EXPAND><CODE>expand 0.5</CODE></A>.
	If the optional argument <CODE>K</CODE> is present, the command
	<A HREF=#ECOLUMN><CODE>ECOLUMN</CODE></A> may be used to specify a
	color corresponding to each (x,y) point plotted.  Each item of
	data read by <A HREF=#ECOLUMN><CODE>ECOLUMN</CODE></A> corresponds
	to a color index.

<HR><A NAME=POLY>
<DT>
POLY - draws a polygon.
<DD>
	<CODE>POLY</CODE> will draw a closed polygon based on the data
	from the most recent call to the
	<A HREF=#XCOLUMN><CODE>XCOLUMN</CODE></A> and
	<A HREF=#YCOLUMN><CODE>YCOLUMN</CODE></A> commands.
	<CODE>POLY</CODE> will draw the polygon according to the
	graphical characteristics such as
	<A HREF=#FILL><CODE>FILL</CODE></A> and
	<A HREF=#COLOR><CODE>COLOR</CODE></A>.
	Also see the command <A HREF=#RECT><CODE>RECT</CODE></A>.

<HR><A NAME=PUTLABEL>
<DT>
PUTLABEL just str - writes justified text <CODE>STR</CODE> at the current location.
<DD>
	<CODE>PUTLABEL</CODE> writes a string at the current cursor
	location with rotation and size specified by the
	<A HREF=#ANGLE><CODE>ANGLE</CODE></A> and
	<A HREF=#EXPAND><CODE>EXPAND</CODE></A> commands
	(exactly like <A HREF=#LABEL><CODE>LABEL</CODE></A>).
	The label is justified with respect to the current
	location according to the argument <CODE>JUST</CODE>: 0 for left
	justified; 0.5 for centered justified; and 1 for right justified.
	If <CODE>JUST</CODE> is negative, the cursor is activated and the
	position of the cursor when any key is struck is used as the
	current location and the absolute value of <CODE>JUST</CODE> is
	used to determine the string justification.
	NOTE: Because -0 will be interpreted as +0 by the
	string parser, use something like -0.001 for left justification
	when using the cursor to locate the position.

<HR><A NAME=QUARTER>
<DT>
QUARTER [N] - allows quick selection of a subsection of the current image.
<DD>
	<CODE>QUARTER</CODE> is a short cut to selecting a quadrant of
	the current image.  Rather than determining pixel ranges and
	then calling <A HREF=#SUBIMAGE><CODE>SUBIMAGE</CODE></A>,
	<CODE>QUARTER</CODE> allows the user to quickly set the
	corresponding pixel range.  If the optional parameter <CODE>N</CODE>
	is present and set to 1, <CODE>QUARTER</CODE> selects the lower
	left quadrant; set to 2, the lower right; 3, the upper left;
	and 4, the upper right.  <CODE>QUARTER</CODE>, with any other
	value for <CODE>N</CODE> or if no argument is present, selects
	the inner quarter of the current image.

<HR><A NAME=RANGE>
<DT>
RANGE [X1 X2 Y1 Y2] - limits the range over which to fit.
<DD>
	<CODE>RANGE</CODE> permits a way to limit the data used by the
	<A HREF=#FIT><CODE>FIT</CODE></A> command.  By default, the
	<CODE>FIT</CODE> command uses all the data read by the most
	recent call to the <A HREF=#XCOLUMN><CODE>XCOLUMN</CODE></A>
	and <A HREF=#YCOLUMN><CODE>YCOLUMN</CODE></A> commands.
	Specifying the limits in one direction restricts which data
	will be used by the <A HREF=#FIT><CODE>FIT</CODE></A> command.
	If the limits in either direction are equal, then the fit is
	unbounded in that direction.

<HR><A NAME=READ>
<DT>
READ fspec - reads plot commands from file "fspec".
<DD>
	<CODE>READ</CODE> reads plot commands from a file and places
	them in the command buffer.  Commands are not executed and macros
	are ignored.  To execute the commands, see the command
	<A HREF=#XCOLUMN><CODE>PLAYBACK</CODE></A>; to define macros,
	see the commands <A HREF=#DEFINE><CODE>DEFINE</CODE></A> and
	<A HREF=#MACRO><CODE>MACRO</CODE></A>.

<HR><A NAME=RECT>
<DT>
RECT xmin xmax ymin ymax - draw a rectangle, using fill-area attributes.
<DD>
	<CODE>RECT</CODE> is like the command
	<A HREF=#POLY><CODE>POLY</CODE></A> but draws a
	rectangle bordered by the World (user) coordinate vertices
	<CODE>(XMIN, YMIN)</CODE> and <CODE>(XMAX, YMAX)</CODE>.
	<CODE>RECT</CODE> uses the current
	<A HREF=#FILL>fill</A>, <A HREF=#LSTYLE>line style</A>, and
	<A HREF=#COLOR>line color</A> characteristics.

<HR><A NAME=RESET>
<DT>
RESET - full reset of the graphics state of the current plotting device.
<DD>
	<CODE>RESET</CODE> executes a full reset of graphics state of the
	current plotting device.  The coordinate transformation matrix is
	reset to the default value (see the command
	<A HREF=#TRANSFER><CODE>TRANSFER</CODE></A>); the current
	angle is set to 0 (see the command
	<A HREF=#ANGLE><CODE>ANGLE</CODE></A>); the
	tick mark interval and the number of minor tick marks is reset to 0
	(see the command <A HREF=#TICKSIZE><CODE>TICKSIZE</CODE></A>); the
	image transfer function is reset to linear (see the command
	<A HREF=#ITF><CODE>ITF</CODE></A>); the text background color index
	is set to transparent (see the command
	<A HREF=#BGCI><CODE>BGCI</CODE></A>); and the
	color, expand, fill, font, line style, and line width are all set
	to 1 (see the commands <A HREF=#COLOR><CODE>COLOR</CODE></A>,
	<A HREF=#EXPAND><CODE>EXPAND</CODE></A>,
	<A HREF=#FILL><CODE>FILL</CODE></A>,
	<A HREF=#FONT><CODE>FONT</CODE></A>,
	<A HREF=#LSTYLE><CODE>LSTYLE</CODE></A>, and
	<A HREF=#LWIDTH><CODE>LWIDTH</CODE></A>).  Finally, the Palette
	(see the command <A HREF=#PALETTE><CODE>PALETTE</CODE></A>) is
	reset to the default palette and the number of halftone levels is
	reset to the maximum range possible for the current device.

<HR><A NAME=RGB>
<DT>
RGB K red [green blue] - sets the color representation using the RGB system.
<DD>
	<CODE>RGB</CODE> sets the color representation of color index
	<CODE>K</CODE> to the Red-Green-Blue values.  The range of values
	for the color value inputs are [0 - 1], inclusive.  If only the
	first two arguments are present, the <CODE>RED</CODE> value is
	also used to set the <CODE>GREEN</CODE> and <CODE>BLUE</CODE> values
	(useful when creating just a gray scale).
	If <CODE>K</CODE> is larger than the maximum color index of the
	device (use the command '<A HREF=#ECHO><CODE>echo cmin cmax</CODE></A>'
	to display the color index range), then this call is ignored.

<HR><A NAME=SCALE>
<DT>
SCALE X [Y [K]] - sets the viewport size scale.
<DD>
	<CODE>SCALE</CODE> allows the viewport (device) size scale to be
	set based on the range of the World (user) coordinate range.  The
	viewport is changed to the World range divided by the input scale
	factor (i.e. <CODE>X</CODE> and <CODE>Y</CODE> are in World units h
	per desired unit).  The optional parameter <CODE>K</CODE> specifies
	the desired conversion the two scale parameters will be performing
	and scales as shown below:
	<PRE>
    K   X and Y are dimensioned as...
---------------------------------------------------------------
    0 : World (user) units per normalized device coordinate.
    1 : World units per inch.
    2 : World units per mm.
    3 : World units per pixel.
	</PRE>
	If <CODE>K</CODE> is not present on the command line, then it is
	set to 2.  The value of <CODE>Y</CODE> may be omitted, in which
	case it is set to the same value as <CODE>X</CODE>.  However,
	if a value other than <CODE>K = 2</CODE> is desired, both
	<CODE>X</CODE> and <CODE>Y</CODE> MUST be present.  If <CODE>K</CODE>
	is not one of the above accepted values, no conversion is performed.
	Also, if either <CODE>X</CODE> or <CODE>Y</CODE> are less than
	or equal to 0, no conversion is performed.  As an example,
	if the limits of the current plot are 0 - 10 in the x-direction and
	0 - 12 in the y-direction, <CODE>scale 2 3 1</CODE> will produce
	a viewport region 5x4 inches in size.  Hence, it is possible to
	produce a viewport larger than the view size.

<HR><A NAME=SET>
<DT>
SET V EXPRESSION - sets the user variable <CODE>V</CODE> to result of <CODE>EXPRESSION</CODE>.
<DD>
	<CODE>SET</CODE> sets the symbolic variable <CODE>V</CODE> to the
	result from evaluating the statement <CODE>EXPRESSION</CODE>.
	<CODE>V</CODE> is a string variable if it is enclosed in
	double quotes.  If <CODE>V</CODE> is a string variable, then
	<CODE>EXPRESSION</CODE> represents the value to assign to the
	variable (Example: '<CODE>set "print" lpr -Pps</CODE>').  For
	user and vector variables, <CODE>EXPRESSION</CODE> may be a
	simple syntax statement that will be evaluated.  Precedence in
	the expression is ALWAYS left to right with the inner most set
	of parenthesis evaluated first.  Hence
	<CODE>(5 * 3 + 2) != (2 + 3 * 5)</CODE> but <CODE>((5 * 3) + 2)</CODE>
	does equal <CODE>(2 + (3 * 5))</CODE>.  Currently, a maximum of
	20 nested levels of parenthesis is permitted.  Also note that a
	set of parenthesis is understood to mean "()", "[]", or "{}".
	At the most basic level, <CODE>EXPRESSION</CODE> may just be an
	item that is either a number, a pre-defined user variable, or a
	vector element.  More complex <CODE>EXPRESSION</CODE>'s may be
	generated using the operators and functions listed below.  If
	an item begins with '-', it is treated as a unary minus sign.
	The current defined operations that act on two variables are:
	<PRE>
   +          add |  /         divide | max               maximum
   -     subtract |  %  modulo divide | min               minimum
   *     multiply |  \ integer divide |  **        exponentiation
  ==     equal to |  >   greater than |  >= greater than or equal
  != not equal to |  <      less than |  <=    less than or equal
  ||   logical OR | &&    logical AND |   ^           logical XOR
	</PRE>
	Note that <CODE>min</CODE> and <CODE>max</CODE> are used as
	<CODE>(a min b)</CODE> and <CODE>(a max b)</CODE>.  Also, note
	that logical operations return 1 if true; 0 if false.
	Current pre-defined functions (i.e. f(x)) are:
	<PRE>
  sqrt(x)          square root |    abs(x)        absolute value
   int(x)   integer truncation |   nint(x)           nearest int
    ln(x)    natural logarithm |    log(x)     base-10 logarithm
   exp(x) base-e antilogarithm |  log10(x)     base-10 logarithm
   sin(x)      sine in radians |   sind(x)       sine in degrees
  asin(x)   arcsine in radians |  asind(x)    arcsine in degrees
	</PRE>
        The four types of sin functions apply just as well for the
	four functions of cos and tan.
	Note that either <CODE>log(x)</CODE> or <CODE>log10(x)</CODE>
	may be used for the log base 10.  Many of the symbolic and user
	variables are displayed using the <A HREF=#SHOW><CODE>SHOW</CODE></A>
	command, where symbolic names are displayed in CAPS.  All other
	variables can be displayed using the
	<A HREF=#ECHO><CODE>ECHO</CODE></A> command.  All of the user
	variable and vectors can then be used as arguments to any other
	command.
	NOTE: The macro arguments ($1, $2, ...) may be redefined using
	this command but only if the macro argument is a numeric user variable.

<HR><A NAME=SHOW>
<DT>
SHOW [what] - shows current limits and attributes.
<DD>
	<CODE>SHOW</CODE> lists the values of some the symbolic and user
	variables, including the current location and plot region limits
	in World (user) and Viewport (device) coordinates, the value of
	the expansion and angle variables, the line style and width, etc.
	If no arguments are present, then everything below is displayed.
	If the optional argument is present, then it represents a string
	used to define what will be shown.  The options that are currently
	available are:
	<PRE>
    B : Box attributes.
    C : Coordinate system information.
    D : Device, Printer, and Data file information.
    G : Graphical attributes.
    I : Image characteristics.
    R : Register variables (\0 - \20).
	</PRE>
	More than one option may be selected (e.g. '<CODE>show bci</CODE>')
	and unknown options are ignored.  The command
	<A HREF=#ECHO><CODE>ECHO</CODE></A> may also be used to display
	user variable values.

<HR><A NAME=SLEVEL>
<DT>
SLEVEL type [value] - sets the type and value used to scale contour levels.
<DD>
	<CODE>SLEVEL</CODE> is used to change the <CODE>TYPE</CODE> and
	<CODE>VALUE</CODE> used to scale the contour level array.
	Initially, <CODE>TYPE</CODE> (the user string variable "LEVTYPE")
	is set to <CODE>A</CODE> and <CODE>VALUE</CODE> (the user variable
	SLEVEL) is set to 1.0 so that no scaling is performed.
	<CODE>TYPE</CODE> may only be either <CODE>A</CODE> or <CODE>P</CODE>
	(all other values are treated as <CODE>A</CODE>) and defines the
	type of contour scale factor.  If <CODE>TYPE</CODE> is <CODE>P</CODE>,
	then scaling is done by percentage; <CODE>A</CODE> for absolute
	scaling.  <CODE>VALUE</CODE> represents how much the array will
	be scaled.  For example, <CODE>slevel P 1</CODE> will multiply
	each contour level by 1% of the image peak value.  Likewise,
	<CODE>slevel A 1E-4</CODE> would multiply each contour level
	by 0.0001.  If the parameter <CODE>VALUE</CODE> is not given,
	<CODE>SLEVEL</CODE> will not change.  The contour level array
	affected is defined with the commands
	<A HREF=#LEVELS><CODE>LEVELS</CODE></A> or
	<A HREF=#AUTOLEVS><CODE>AUTOLEVS</CODE></A>
	and is altered internally every time the command
	<A HREF=#CONTOUR><CODE>CONTOUR</CODE></A> is called.

<HR><A NAME=STRING>
<DT>
STRING name [W1 [W2]] - sets a string variable "name" from a file.
<DD>
	<CODE>STRING</CODE> allows certain string variables to be
	redefined using data in an external file.  <CODE>STRING</CODE>
	reads one line (see the commands <A HREF=#DATA><CODE>DATA</CODE></A>
	and <A HREF=#LINES><CODE>LINES</CODE></A>) and assigns the string
	starting with word <CODE>W1</CODE> and ending with <CODE>W2</CODE>
	to the string variable <CODE>NAME</CODE>.  If <CODE>W1</CODE> is
	omitted or set to 0, the entire line is assigned.  If <CODE>W1</CODE>
	is greater than 0, it represents the starting word of the
	assignment.  If <CODE>W2</CODE> is not present, it is set to the
	value of <CODE>W1</CODE>.  If <CODE>W2</CODE> is set to 0, it
	corresponds to the end of the line; if it less than zero, the
	absolute value represents the number of words to read; if it
	is greater than zero, it specifies the last word to read.
	<P>
	Currently, the pre-defined string variables are: XBOX and YBOX
	(which are the default X and Y arguments for a call to the
	command <A HREF=#BOX><CODE>BOX</CODE></A> and are initially
	set to "BCNST" and "BCNSTV", respectively); PRINT (which is
	the default print command to spool a plot when the command
	<A HREF=#HARDCOPY><CODE>HARDCOPY</CODE></A> or
	<A HREF=#PHARD><CODE>PHARD</CODE></A> is used); and
	HELPFILE (which is the full name of the help file).  Other
	pre-defined string names include XHEADER and YHEADER (see
	the command <A HREF=#HEADER><CODE>HEADER</CODE></A>) and LEVTYPE
	(see the command <A HREF=#SLEVEL><CODE>SLEVEL</CODE></A>).  New
	string variables may be defined using the
	<A HREF=#NEW><CODE>NEW</CODE></A> command, released with the
	command <A HREF=#FREE><CODE>FREE</CODE></A>, and may also be
	assigned to using the <A HREF=#SET><CODE>SET</CODE></A> command.

<HR><A NAME=SUBIMAGE>
<DT>
SUBIMAGE xmin xmax ymin ymax - sets the index range of a subimage.
<DD>
	<CODE>SUBIMAGE</CODE> allows the user to specify the indices
	of a subimage read by the <A HREF=#IMAGE><CODE>IMAGE</CODE></A>
	command.  When <A HREF=#IMAGE><CODE>IMAGE</CODE></A> reads in
	a new image, the minimum values are set to 1 and the maximum
	values are set to the number of pixels in the X and Y directions
	(user variables NX and NY).  Use this command to select a subportion
	of the image by entering the starting and ending pixels of
	interest.  Using the command <A HREF=#HEADER><CODE>HEADER</CODE></A>
	after the command <A HREF=#SUBIMAGE><CODE>SUBIMAGE</CODE></A> will
	reset the limits of the plot region (World coordinates) to the
	entire subimage region selected.

<HR><A NAME=SUBIMAGE>
<DT>
SUBMARGIN xmarg [ymarg] - sets the gap between individual panels.
<DD>
	<CODE>SUBMARGIN</CODE> may be used to fix the gap size between
	individual frames when using the command
	<A HREF=#PANEL><CODE>PANEL</CODE></A>.  The gap between
	individual panels is controlled by the user variables
	XSUBMAR and YSUBMAR (which are initially set to 2 in character
	height units).  This command is a convenience function to
	facilitate setting these variables.  The arguments <CODE>XMARG</CODE>
	and <CODE>YMARG</CODE> are used to set each of the two user variables.
	If this command is called with only one argument, it is used to
	set both XSUBMAR and YSUBMAR.

<HR><A NAME=SYMBOL>
<DT>
SYMBOL N - sets the current point symbol to <CODE>N</CODE>.
<DD>
	<CODE>SYMBOL</CODE> causes points to be drawn as the symbol
	<CODE>N</CODE>, where <CODE>N</CODE> refers to the graph marker
	or a Hershey symbol number.  The symbol number corresponds to
	the type of symbol drawn.  If the symbol number is -1, it
	draws a dot of the smallest possible size.  If the symbol number
	is between 0 and 31, it corresponds to a set of pre-defined symbols;
	between 33 and 127, it corresponds to the ASCII character in the
	currently selected font; and larger than 127, corresponds to the
	Hershey symbol of the same number.

<HR><A NAME=TICKSIZE>
<DT>
TICKSIZE XTICK NXSUB YTICK NYSUB - sets tick intervals for the <CODE>BOX</CODE> command.
<DD>
	<CODE>TICKSIZE</CODE> determines tick intervals for the command
	<A HREF=#BOX><CODE>BOX</CODE></A>.  The argument <CODE>NXSUB</CODE>
	refers to the number of intervals between major tick marks on
	the X-axis and <CODE>XTICK</CODE> refers to the interval between
	large ticks in World (user) coordinates.  If <CODE>XTICK</CODE>
	or <CODE>NXSUB</CODE> are 0, the box routine will supply it's
	own intervals according to the World coordinate limits (at
	least 3 major tick marks).  Likewise for <CODE>NYSUB</CODE> and
	<CODE>YTICK</CODE>, except for the Y-axis.  Setting
	<CODE>NXSUB</CODE> or <CODE>NYSUB</CODE> to 1 will inhibit any
	minor tick marks regardless of the arguments to the
	<A HREF=#BOX><CODE>BOX</CODE></A> command.

<HR><A NAME=TRANSFER>
<DT>
TRANSFER T1 T2 T3 T4 T5 T6 - specifies the image coordinate transformation.
<DD>
	<CODE>TRANSFER</CODE> specifies the coordinate transformation
	between indices I and J of the image array and World (user)
	coordinates.  All two dimensional plot commands (for example,
	<A HREF=#CONTOUR><CODE>CONTOUR</CODE></A> or
	<A HREF=#HALFTONE><CODE>HALFTONE</CODE></A>) require a
	coordinate transformation function to place the image on the
	plot surface.  The World coordinates (X, Y) of the array point
	(I, J) are related by:
	<PRE>
    X = T1 + (T2 * I) + (T3 * J)
    Y = T4 + (T5 * I) + (T6 * J)
	</PRE>
	Usually <CODE>T3</CODE> and <CODE>T5</CODE> are zero unless
	the coordinate transformation involves a rotation or sheer.
	The default transformation is (0, 1, 0, 0, 0, 1).  Calling
	<CODE>TRANSFER</CODE> with no arguments resets the array to
	these default values.  Usually, the command
	<A HREF=#HEADER><CODE>HEADER</CODE></A> is used to set the
	transformation matrix.

<HR><A NAME=VECTOR>
<DT>
VECTOR [ANGLE [VENT]] - Draws a vector field as a sequence of arrows.
<DD>
	<CODE>VECTOR</CODE> draws a vector field as a number of arrows.
	The tails of the arrows lie at points (x,y) read by
	<A HREF=#XCOLUMN><CODE>XCOLUMN</CODE></A> and
	<A HREF=#YCOLUMN><CODE>YCOLUMN</CODE></A>, the lengths of the
	arrows "r" are read by <A HREF=#PCOLUMN><CODE>PCOLUMN</CODE></A>
	and the direction of the arrow (counter-clockwise from +X in
	degrees) is read with <A HREF=#ECOLUMN><CODE>ECOLUMN</CODE></A>.
	See the command <A HREF=#ARROW><CODE>ARROW</CODE></A> for a
	description of the optional parameters <CODE>ANGLE</CODE> and
	<CODE>VENT</CODE>.
	NOTE: The use of <A HREF=#PCOLUMN><CODE>PCOLUMN</CODE></A>
	in this context requires <A HREF=#SYMBOL><CODE>SYMBOL</CODE></A>
	to be re-executed before plotting points.

<HR><A NAME=VIEWPORT>
<DT>
VIEWPORT VX1 VX2 VY1 VY2 - sets the physical location of the plot.
<DD>
	The plot region is a rectangle within the coordinates allowed
	by the plotting device (for example, the Viewport is the region
	where the command <A HREF=#BOX><CODE>BOX</CODE></A> will draw its grid).
	Vectors and points are truncated at the edge of this plot region.
	<CODE>VIEWPORT</CODE> specifies (in normalized device coordinates)
	where the plot region is located.  Viewport arguments for this
	command range from 0 to 1, inclusive, in each direction for all
	devices.  Note that the command <A HREF=#VSTAND><CODE>VSTAND</CODE></A>
	is called to reset the Viewport coordinates every time a new
	device is opened.

<HR><A NAME=VSIZE>
<DT>
VSIZE VX1 VX2 VY1 VY2 - sets the physical location of the plot in inches.
<DD>
	<CODE>VSIZE</CODE> specifies (in inches) where the plot region
	is located.
	See also the <A HREF=#VIEWPORT><CODE>VIEWPORT</CODE></A> command.

<HR><A NAME=VSTAND>
<DT>
VSTAND - sets the standard (default) viewport.
<DD>
	<CODE>VSTAND</CODE> defines the standard viewport scale.
	Currently, this is a box with margins on each side of about
	4 character heights.  This command is executed every time a
	new device is chosen.
	See also the <A HREF=#VIEWPORT><CODE>VIEWPORT</CODE></A> and
	<A HREF=#VSIZE><CODE>VSIZE</CODE></A> commands.

<HR><A NAME=WEDGE>
<DT>
WEDGE SIDE DISP THICK [MIN MAX BOXARG] - draws a halftone wedge.
<DD>
	<CODE>WEDGE</CODE> illustrates the range of values drawn by
	the halftone command.  The <CODE>SIDE</CODE> argument is
	used to specify the size and orientation of the wedge.
	The parameter <CODE>SIDE</CODE> must be either
	`B', `L', `T', or `R' to specify the Bottom, Left, Top, or Right
	side of the viewport.  The <CODE>DISP</CODE> argument specifies
	the displacement from the frame edge (in character height
	units).  If <CODE>DISP</CODE> is negative, then the wedge is
	drawn inside the viewport.  The parameter <CODE>THICK</CODE>
	is the thickness of the wedge (in character height units).
	The optional arguments <CODE>MIN</CODE> and <CODE>MAX</CODE>
	allow the user to specify the intensity range of the wedge.
	By default, <CODE>MIN</CODE> and <CODE>MAX</CODE> are the
	values used by the most recent call to the
	<A HREF=#HALFTONE><CODE>HALFTONE</CODE></A> command.
	Providing the values of <CODE>MIN</CODE> and <CODE>MAX</CODE>
	will cause <CODE>WEDGE</CODE> to display a different range
	of intensities when labeling the wedge box.  The optional
	argument <CODE>BOXARG</CODE> permits the user to control how
	the box around the wedge is drawn (or omit it altogether).
	By default, a simple box is drawn setting <CODE>BOXARG</CODE>
	to BCSTN/M (depending on the orientation); to eliminate the box,
	set <CODE>BOXARG</CODE> to 0 (for more information on permitted
	box arguments, see the <A HREF=#BOX><CODE>BOX</CODE></A> command).
	The arguments from the most recent call to the
	<A HREF=#TICKSIZE><CODE>TICKSIZE</CODE></A> command are used
	to draw a box around the wedge and numerically label it.
	To use the optional <CODE>BOXARG</CODE> argument, the
	<CODE>MIN</CODE> and <CODE>MAX</CODE> arguments must be provided.

<HR><A NAME=WINADJ>
<DT>
WINADJ [xmin xmax ymin ymax] - sets limits and viewport to same aspect ratio.
<DD>
	<CODE>WINADJ</CODE> changes the World (user) limits and
	adjusts the viewport so that the World coordinate scales are
	equal in x and y.  If the four optional parameters are present,
	they are used as the new World coordinate values; otherwise,
	the minimum and maximum values of the
	<A HREF=#XCOLUMN><CODE>XCOLUMN</CODE></A> and
	<A HREF=#YCOLUMN><CODE>YCOLUMN</CODE></A> arrays are used.
	Typical usage is (after an image is loaded) to rescale the
	viewport so that the image has the proper aspect ratio.
	For example, using the command '<CODE>winadj 0 nx 0 ny</CODE>'
	after the <A HREF=#IMAGE><CODE>IMAGE</CODE></A> command BUT
	before the <A HREF=#HEADER><CODE>HEADER</CODE></A> command
	will produce square pixels.

<HR><A NAME=WRITE>
<DT>
WRITE fspec xxx - writes macro <CODE>xxx</CODE> to file "fspec".
<DD>
	<CODE>WRITE</CODE> the commands making up a macro to a file.
	If the macro name <CODE>XXX</CODE> given is <CODE>all</CODE>,
	all macros currently defined and the entire command buffer
	(<A HREF=#BUFFER><CODE>BUFFER</CODE></A>) are written to the
	file.  If the macro name <CODE>XXX</CODE> is <CODE>macros</CODE>,
	then only the currently defined macros are written to the file
	(i.e. the command buffer <A HREF=#BUFFER><CODE>BUFFER</CODE></A>
	is not written).  Otherwise, <CODE>XXX</CODE> represents a list
	of macro names to be written in the output file.
	NOTE: This command will OVERWRITE any currently existing file.

<HR><A NAME=XCOLUMN>
<DT>
XCOLUMN N - reads X data from column <CODE>N</CODE> of the current file.
<DD>
	<CODE>XCOLUMN</CODE> reads column <CODE>N</CODE> from the
	data file as the values of the X coordinates of the data points.

<HR><A NAME=XLABEL>
<DT>
XLABEL str - writes the label <CODE>STR</CODE> centered under the X axis.
<DD>
	<CODE>XLABEL</CODE> writes a string centered under the
	X axis drawn by the command <A HREF=#BOX><CODE>BOX</CODE></A>.
	Also see the command <A HREF=#MTEXT><CODE>MTEXT</CODE></A>.

<HR><A NAME=YCOLUMN>
<DT>
YCOLUMN N - reads Y data from column <CODE>N</CODE> of the current file.
<DD>
	<CODE>YCOLUMN</CODE> reads column <CODE>N</CODE> from the
	data file as the values of the Y coordinates of the data points.

<HR><A NAME=YLABEL>
<DT>
YLABEL str - writes the label <CODE>STR</CODE> centered left of the Y axis.
<DD>
	<CODE>YLABEL</CODE> writes a string centered left of the Y axis
	drawn by the command <A HREF=#BOX><CODE>BOX</CODE></A>.  Also
	see the command <A HREF=#MTEXT><CODE>MTEXT</CODE></A>;
	especially if the displacement of the text is insufficient to
	avoid the axis numbers.
</DL>

\end{rawhtml}

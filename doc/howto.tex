%% This file is to be included by latex in wip.tex
%
% Chapter:  How to User This Manual
%
\mylabel{c:howto}
\myfile{howto.tex}

This manual serves as a reference guide and cookbook for the
\wip\ interactive graphics package.
The manual is split into three parts.
The first part introduces you to
the \wip\ package and its general concepts.
The next part explains more advanced plotting techniques.
The last part of this manual
contains appendices with a variety of details
as well as some plotting examples.

Part~\ref{p:intro} is a general introduction.
It contains
Chapter~\ref{c:iface} which describes
the interactive user interface and how to start up \wip.
Chapter~\ref{c:concepts}
describes some general plotting concepts.
Chapter~\ref{c:basic} describes some of the commands used to
generate simple plots and Chapter~\ref{c:images}
explains the commands associated with images\index{Images}.

Part~\ref{p:advanced} describes
more advanced plotting techniques.
Chapter~\ref{c:fit}
illustrates how to perform various fits on data and then display the results.
Chapter~\ref{c:macros}
describes how to define and edit macros\index{Macros}.
Chapter~\ref{c:uservar}
explains user variables\index{User Variables}\index{String Variables}
and how to use them.
Finally, Chapter~\ref{c:flow}
shows how to have more control over which
commands are executed.

Lastly, Part~\ref{p:appendix} contains the appendices.
Appendix~\ref{a:wipinit} describes how to set up a file which
can be used to customize your \wip\ environment.
Appendix~\ref{a:cmdline} discusses the
options that may be specified on the command line\index{Command Line}
when starting \wip.
Appendix~\ref{a:glossary} is a glossary which 
describes and defines often-used terms for easy reference.
Appendix~\ref{a:cmdname} lists, in alphabetical order,
the commands \wip\ currently understands.
Next, Appendix~\ref{a:samples} contains
several examples of command files used to create the
accompanying plots.
And, finally, Appendix~\ref{a:faq} presents a list of frequently asked
questions along with explanations and solutions.

\section*{Notation:}
In the examples throughout this manual the
prompt from \wip\ is shown.
In most examples this is
done symbolically by the string \wipp followed by the
text typed by the user.
For example:
\begin{wiplist}%
  \index{Commands!{\tt device}}
  \item {\tt device myfile.ps/vps}
\end{wiplist}
illustrates the use of the \wip\ prompt followed by the
\wip\ command to define a new device.
When trying this example out, the user should not type the characters of
the prompt; just the text that follows it.
When the example is illustrating a command issued at the operating
system level, it will always be done with the percent sign.
For example, the command:
\begin{wiplist}%
  \item [\%] {\tt wip -d /tek}
\end{wiplist}
shows how to start \wip\ with a device other than the default.

In addition, almost every example illustrates one important feature of \wip:
the use of comments\index{Comments}.
While not required, comments are quite useful in \wip\ plot files,
especially several weeks or months later when you return to adopt a plot
file for use in another plot.
Comments begin with the comment character
(\#)\index{Comments!comment character(\#)}
and continue to the end of the current line.
Comments may appear anywhere except on command lines that require a text
string as an argument.
In this case, comments should not be included as they will be
incorporated into the text string.

\section*{There's always {\tt help}:}
As mentioned previously,\index{Help}
Appendix~\ref{a:cmdname} lists all the commands currently understood by \wip.
This listing is the same as the listing generated on-line by the command
{\tt help xxx}\index{Commands!{\tt help}}%
\index{Help!commands!{\tt help}}
where {\tt xxx} is any \wip\ command
name (see Section~\ref{s:help}).
Many times in this manual, a list of commands
will appear relevant to a particular style of plotting.
Rather than go into great detail in these sections
about command arguments and options,
the reader should investigate the command syntax presented
in Appendix~\ref{a:cmdname}.
Additionally, command arguments or options are more likely to change
with improvements to \wip\ while their general concepts will not.
For this reason, the discussions about individual commands will, in
general, be postponed until Appendix~\ref{a:cmdname} where individual
commands will be kept up to date.
If ever a discrepancy exists between this manual and the on-line
{\tt help} command, the on-line version is to be taken as correct.

Other useful sources of help in this manual include examples of
actual \wip\ commands accompanied by the generated plot and
a list of frequently asked questions about \wip.
The examples can be found in Appendix~\ref{a:samples}.
The frequently asked questions, as well as answers and examples,
can be found in Appendix~\ref{a:faq}.

\section*{READLINE:}
The \wip\ user interface has been written in such a way that,
if it exists,
the Readline library\index{Readline Library}%
\index{Commands!readline!editing}\index{Commands!readline!recall}
of routines can be linked into \wip\ when it is compiled.
Doing this gives \wip\ the command line recall and editing
characteristic of {\tt Emacs}.
By default, this capability is not present.
However, if it is present, a message will appear each time \wip\ is
started identifying which command listing file is being read.
These commands will not be executed but will be available to be recalled
and edited.
This name is also the file which \wip\ will, on exiting, write the
history of commands executed.
Even if this option is added when \wip\ is compiled,
it can be dynamically ignored via command line\index{Command Line} options
(see Appendix~\ref{a:cmdline}).

\section*{PGPLOT:}
The lower level work of \wip\ is done primarily by a collection of graphical
subroutines (developed by Tim Pearson) called PGPLOT\index{PGPLOT}.
\htmladdnormallink{The PGPLOT subroutines}{http://astro.caltech.edu/~tjp/pgplot/}
may be obtained free--of--charge because the software was written
under Unites States government support, although it is copyrighted by
California Institute of Technology.
PGPLOT is provided ``as is'' and carries no promise that it
will work in your configuration, but all source code is included.
Information/questions about PGPLOT should be addressed care of:
\begin{verbatim}
Tim Pearson
Astronomy Dept 105-24, Caltech, Pasadena, California 91125, USA
Internet:          tjp@astro.caltech.edu or Pearson_T@caltech.edu
NSI-Decnet (SPAN): Deimos::TJP or 15237::TJP
Telephone:         +1 818 395-4980
WWW:               "http://astro.caltech.edu/~tjp/pgplot/"
\end{verbatim}

\section*{World Wide Web:}
The
\htmladdnormallinkfoot{\wip\ User Manual}{http://bima.astro.umd.edu/wip/manual/wip.html}
can also be viewed via the World Wide Web (WWW).
There is also
\htmladdnormallinkfoot{the \wip\ home page}{http://bima.astro.umd.edu/wip/}
which contains general information about \wip\ as well as information on
how to obtain the latest \wip\ distribution.

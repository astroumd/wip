%
% Image (2-d) graphics.
%
\mylabel{c:images}
\myfile{images.tex}

This chapter will describe the image capabilities of \wip.
The first section focuses on the image files and file formats.
The next section details the commands \wip\ uses to select
or exclude parts of the image.
The following section identifies some techniques commonly used to get
the image in the right location and at the right aspect ratio. 
The last section concentrates on the \wip\ commands that generate plots
using image data and the \wip\ commands that can be used to enhance
these types of plots.

\section		{Image Files}
\mylabel{s:imfile}
\index{Images!image files}

Currently, there are a number of image file formats that \wip\ can recognize.
The number will change as more image file format drivers become available to
\wip\ and are included in the distribution.
The currently acceptable formats are:
\begin{itemize}
  \item Miriad image data sets.
  \item FITS image files.
  \item Basic image files.
\end{itemize}
A brief description of each type will appear later in this section.

In the current configuration of \wip, only
one active image may be present at any one time.
This is primarily because of the current way that \wip\ reserves
memory for the image and also because of the limitation on memory space
that many images can impose.

The
{\tt image}\index{Commands!{\tt image}}%
\index{Images!commands!{\tt image}}
command is used to ``load'' an image into \wip.
The command has three arguments but only the first one is required.
The required first argument is the image file name.
How the image file name is specified may be dependent on the format of the
image (this will be explained in the section that follows).
The second argument is optional and selects the plane number to read.
This defaults to the first plane if this argument is not present or a
non-existent plane is requested.
The third argument is also optional and specifies a mask value.
If the image has masked (undefined) data,
\wip\ will set the masked data (\eg magic numbers)
to a value of -99 unless the third argument is present.
If the mask value is present, then all masked data will be set to the
requested value.

When the {\tt image} command is called,
\wip\ attempts to open the image file based
on the different types of drivers available.
If \wip\ can identify the image type, it will attempt to read the header
information (if present) and then load the image into memory.

When an image has more than 3 dimensions,
the plane number argument ($\cal P$) may take on any value in the range:
\[ 1 \leq {\cal P} \leq \prod_{i=3}^{\rm NAXIS} {\rm NAXIS}_{i}, \]
where ${\rm NAXIS}$ is the total number of dimensions in the data set
and ${\rm NAXIS}_{i}$ is the dimension size of the $i$th axis.\footnote{If
${\rm NAXIS}$ only has a value of 2, the plane number may only be 1.}
Hence, \wip\ will be able to correctly assign the plane number
if a plane from a higher dimension is desired.

\subsection*		{The Miriad Image Format}
\mylabel{ss:Miriad images}

Miriad\index{Images!image formats!Miriad}
images are a subset of the Miriad data set format.
Currently, the Miriad image is one ``item'' of the full data set and the
header information is also stored as individual ``items.''
Refer to
\htmladdnormallink{the BIMA Project manuals}{http://bima.astro.umd.edu/programmer.html}
for a more detailed description of
\htmladdnormallink{Miriad}{http://bima.astro.umd.edu/}
and its data structure.

\subsection*		{The FITS Image Format}
\mylabel{ss:FITS images}

The \htmladdnormallink{FITS file format}{http://fits.cv.nrao.edu/}%
\index{FITS}\index{Images!image formats!FITS}
is a standard data transfer format used to pass data
between different computers.
In general, FITS images will contain a header section followed by the
image data (in binary form).
\wip\ expects that the header will be in standard 80 character format and
{\em must} contain the standard keywords: SIMPLE, BITPIX, NAXIS,
NAXIS{\it i} (for $i = 1, \dots$, NAXIS), and END.
\wip\ will {\em only} recognize FITS files with
BITPIX set to 16 (16 bit integers), 32 (32 bit integers),
or -32 (32 bit floating point).
In addition to the standard keywords,
\wip\ requests the following keywords
(but does not require them to be present):
BSCALE, BZERO, BLANK,\footnote{According to the FITS guidelines,
the BLANK keyword should be used only with data with positive BITPIX values.}
DATAMAX, and DATAMIN.
Also, the keywords
CRVAL{\it i},
CRPIX{\it i},
CDELT{\it i},
and
CTYPE{\it i}, (where $i$ is either 1 or 2)
are searched to help set up the transformation matrix.\footnote{Astronomical
images displayed in right ascension and
declination should provide CRVAL{\it i} and CDELT{\it i} values in
units of degrees and degrees per pixel, respectively.}
If the optional keywords are not found,
then reasonable default values are assigned.

\subsection*		{The Basic Image Format}
\mylabel{ss:Basic images}

Basic images\index{Images!image formats!Basic}
are simple binary files that contain a
multi-dimensional array along with (possibly) optional
header information at the beginning of the file.
Most image file formats will be structured something like this,
so the Basic file format should suffice if a particular image driver has not
yet been written.\footnote{Because of the way
Basic Image files are loaded, header information may not be extracted
automatically, or, for that matter, even accessible.}
The necessary information needed to load the Basic file is contained in
the file name argument to the
{\tt image}\index{Commands!{\tt image}}%
\index{Images!commands!{\tt image}}
command.
The name syntax is:
\begin{center}
  {\tt filename`columnsXrows[`offset]}
\end{center}
where {\tt filename} is the name of the file on the disk,
{\tt columns} is the number of columns (in pixels), and
{\tt rows} is the number of rows (in pixels).
The single character {\tt X} defines the pixel data type and
separates the {\tt columns} value from the {\tt rows} value.
The character {\tt X} may be only one of the following characters:
\begin{wiplist}%
  \item [{\tt b:}] unsigned bytes
\samepage
  \item [{\tt s:}] signed 2 byte integers
  \item [{\tt l:}] signed 4 byte integers
  \item [{\tt r:}] 4 byte floating point
  \item [{\tt d:}] 8 byte double precision floating point
\end{wiplist}
Finally, {\tt offset} represents the offset from the beginning of
the file (in units of bytes).
If this part of the name is not present, it defaults to 0 bytes.
The first back quote in the name syntax is required.
The second back quote should not be present unless an offset value is given.
If a Basic file has any header information at the beginning of the file,
this needs to be skipped;
this is provided for with the {\tt offset} parameter.

Some machines byte-swap\index{Images!byte-swapped}\index{Byte-swapped Images}
their data.
For all image types except Basic, the data is byte-swapped as needed.
However, since Basic images are usually created and stored locally,
they are probably already stored in the machine's native format.
Hence, by default, Basic images are not byte-swapped.
If however, on those machines, there are images that need to be
byte-swapped when loaded into \wip,
byte-swapping can be forced by using an upper case character
to identify the pixel data type.
For example, use `{\tt L}' to force byte-swapped a signed 4 byte integer array.
{\sc Note}: Machines that are not byte-swapped will always ignore this
byte-swapping option.

\section		{Image Characteristics}
\mylabel{s:imfunc}
\index{Images!image characteristics}

This section describes the methods \wip\ uses to associate a pixel
in an image with a world coordinate position, how to select a
subsection of the image, how to specify contour levels, and
how to access the image header file.

\subsection*		{The Coordinate Transfer Function}
\mylabel{ss:transfer}

All of the plotting commands discussed in Section~\ref{s:implot}
require a transfer function\index{Images!coordinate transfer function}.
Put simply, the transfer function tells \wip\ where a given pixel should
be located relative to the
world coordinates\index{Coordinate System!World}.
The command to specify the transfer function is
{\tt transfer}\index{Commands!{\tt transfer}}%
\index{Images!commands!{\tt transfer}}.
This command has six arguments that define the mapping from image pixel
to world coordinate.
If ({\tt I, J}) corresponds to a position in the image array
(a particular pixel),
then ($x$, $y$), the corresponding position in world coordinates,
is related by:
\begin{eqnarray*}
\mylabel{eq:transfer}
  x & = & a + b \mbox{I} + c \mbox{J}, \\
  y & = & d + e \mbox{I} + f \mbox{J}.
\end{eqnarray*}
The constants $a$, $b$, $c$, $d$, $e$, and $f$
are the six arguments to the {\tt transfer} command.
As an example, if the limits\index{Commands!{\tt limits}} of a plot
are the same as the number of pixels in an image
(\ie {\tt limits}\index{Coordinate System!commands!{\tt limits}}%
\index{Coordinate System!World}\index{World Units}
1 $Nx$ 1 $Ny$),
then the default transfer function constants of (0, 1, 0, 0, 0, 1)
will correctly map the image onto the display.
This configuration is particularly useful for images with large limits
and very small step sizes (\eg very high resolution astronomical images).

Another simple example deals with images that are flipped in one
direction.
For an image with the $y$-axis flipped and the world coordinates set as
in the example above,
a transfer function with constants (0, 1, 0, ($Ny + 1$), 0, -1)
will dynamically orient the image correctly.

If, however, you have loaded an astronomical image with the world
coordinate limits set by the image header variables {\tt CRVAL},
{\tt CRPIX}, and {\tt CDELT}, then a more useful set of values for
the transfer function would be:
\begin{eqnarray*}
\mylabel{eq:radectransfer}
  a & = & \mbox{CRVALX} - (\mbox{CRPIXX} \times \mbox{CDELTX}), \\
  b & = & \mbox{CDELTX}, \\
  c & = & 0, \\
  d & = & \mbox{CRVALY} - (\mbox{CRPIXY} \times \mbox{CDELTY}), \\
  e & = & 0, \\
  f & = & \mbox{CDELTY}.
\end{eqnarray*}
This transfer function can be, in fact, set up for the user via the command
{\tt header}\index{Commands!{\tt header}}%
\index{Images!commands!{\tt header}}
when these header variables are present in the image file.
If these header variables are not present in the image header,
they may be set manually by the user
(since they are standard user variables\index{User Variables}).
The user must set these user variables {\em after}
the image is read into \wip\ but {\em before} the {\tt header} command.

Using the command {\tt transfer}\index{Commands!{\tt transfer}}%
\index{Images!commands!{\tt transfer}} without any arguments
will automatically reset the transfer function
to the default value of (0, 1, 0, 0, 0, 1).

\subsubsection*		{Rotated Transfer Functions}
\mylabel{ss:cdtransfer}
\index{Images!transfer function}
\index{Images!CD matrix}

Some images do not have the {\tt cdelt} keywords but rather specify the
pixel step sizes in a matrix.
An example of that is a FITS image file that contains the keywords
{\tt CD1\_1}, {\tt CD1\_2}, {\tt CD2\_1}, and {\tt CD2\_2}.
To set up the transfer function (and the limits) for this type of FITS
image like the {\tt header rd} command would,
a macro could be written as follows:
\begin{wiplist}%
  \index{Commands!{\tt define}}
  \index{Commands!{\tt new}}
  \index{Commands!{\tt set}}
  \index{Commands!{\tt transfer}}
  \index{Commands!{\tt limits}}
  \index{Commands!{\tt subimage}}
  \index{Commands!{\tt free}}
  \index{Commands!{\tt end}}
  \index{User Variables!commands!{\tt new}}
  \index{User Variables!commands!{\tt free}}
  \item {\tt define doheader}
\samepage
  \item {\tt new cd11 cd21 cd12 cd22}
  \item {\tt new xconvert yconvert}
  \item {\tt set xconvert 3600.0 / 15.0}
  \item {\tt set yconvert 3600.0}
  \item {\tt set cd11 CD1\_1 * xconvert / cosd(crvaly)}
  \item {\tt set cd21 CD2\_1 * yconvert}
  \item {\tt set cd12 CD1\_2 * xconvert / cosd(crvaly)}
  \item {\tt set cd22 CD2\_2 * yconvert}
  \item {\tt set \esc{1} (xconvert * crvalx) - (crpixx * cd11) - (crpixy * cd12)}
  \item {\tt set \esc{2} (yconvert * crvaly) - (crpixx * cd21) - (crpixy * cd22)}
  \item {\tt transfer \esc{1} cd11 cd12 \esc{2} cd21 cd22}
  \item {\tt set \esc{11} \esc{1} + (cd11 * (subx1 - 0.5)) + (cd12 * (suby1 - 0.5) )}
  \item {\tt set \esc{12} \esc{1} + (cd11 * (subx2 + 0.5)) + (cd12 * (suby2 + 0.5) )}
  \item {\tt set \esc{13} \esc{2} + (cd21 * (subx1 - 0.5)) + (cd22 * (suby1 - 0.5) )}
  \item {\tt set \esc{14} \esc{2} + (cd21 * (subx2 + 0.5)) + (cd22 * (suby2 + 0.5) )}
  \item {\tt limits \esc{11} \esc{12} \esc{13} \esc{14}}
  \item {\tt subimage 1 nx 1 ny}
  \item {\tt free xconvert yconvert}
  \item {\tt free cd11 cd21 cd12 cd22}
  \item {\tt end}
\end{wiplist}
Although this is only one example,
the others probably can be solved in a similar fashion.

\subsection*		{Selecting subsections of the Image}
\mylabel{ss:subsect}
\index{Images!selecting subregions}

Two commands exist that allow easy selection of a sub-portion of an image.
The command
{\tt subimage}\index{Commands!{\tt subimage}}%
\index{Images!commands!{\tt subimage}}
sets the range of the indices such that only a subimage is displayed
(regardless of the range of the transfer function or the limits).
The other command,
{\tt quarter}\index{Commands!{\tt quarter}}%
\index{Images!commands!{\tt quarter}}
(which should be called right after the {\tt image} command)
is used to select a
particular quadrant of the current image.
By default, the {\tt quarter} command selects the inner quarter of the image,
but may be requested to select other quadrants.

If the user would like as much of the image as possible to be displayed,
but only the part that fits within a particular region, then the user
should not specify the sub portion of the image (\ie do not use the
two commands {\tt subimage} or {\tt quarter}) but rather fix
the {\em world} coordinates\index{Coordinate System!World}
to the desired range.
This method of fixing the {\em world} coordinates to clip the image is very
powerful; especially when overlaying several image plots
(see Section~\ref{ss:imoverlay} below and
the example in Section~\ref{p:overlay}).

\subsection*		{Setting Contour Levels}
\mylabel{ss:levels}
\index{Images!setting contour levels}

Currently, there are two commands used to specify the contour levels and
one command that can be used to scale the contour levels.
The commands to specify contour levels are
{\tt levels}\index{Commands!{\tt levels}}%
\index{Images!commands!{\tt levels}}%
\index{Contour Plots!setting levels}%
\index{Contour Plots!setting levels!{\tt levels}}
and
{\tt autolevs}.\index{Commands!{\tt autolevs}}%
\index{Images!commands!{\tt autolevs}}%
\index{Contour Plots!setting levels}%
\index{Contour Plots!setting levels!{\tt autolevs}}
The arguments to {\tt levels} are the individual contour level values.
The arguments to {\tt autolevs} are the number of levels desired, the
type of contour levels ({\tt linear} or {\tt logarithmic}), and,
optionally, the minimum and maximum over which to range the computed levels.
If the minimum and maximum are not specified, they default to the
minimum and maximum of the current image.

The command
{\tt slev}\index{Commands!{\tt slev}}%
\index{Images!commands!{\tt slev}}%
\index{Contour Plots!setting levels!{\tt slev}}
permits the user to scale the contour level array by a fixed amount.
There are two arguments to {\tt slev};
the first required and the second optional.
The first argument is a string which specifies the type of scaling
to perform and may be either {\tt absolute} or {\tt percent}.
The type {\tt absolute} is the default and specifies that the
scaling value be multiplied directly;
type {\tt percent} means to use the scaling value as a percentage
of the peak intensity of the plot.
The second (optional) argument defines the scaling value and defaults to 1
(no scaling).
As an example, the commands
\begin{wiplist}%
  \index{Commands!{\tt levels}}
  \index{Commands!{\tt slev}}
  \item {\tt levels 1 2 3 4 5 6 7 8 9 10}
\samepage
  \item {\tt slev P 10}
\end{wiplist}
will set the contour levels in intervals of 10\%
of the image peak intensity starting
at 10\% of the image peak intensity.
Likewise, the commands
\begin{wiplist}%
  \index{Commands!{\tt slev}}
  \index{Commands!{\tt autolevs}}
  \item {\tt autolevs 10 LIN 1 10}
\samepage
  \item {\tt slev A 1.0E-4}
\end{wiplist}
will set the contour levels to 1.0E-4, 2.0E-4, \ldots, 9.0E-4, and 1.0E-3.

\subsection*		{The Header Command}
\mylabel{ss:header}
\index{Images!header files}

The command
{\tt header}\index{Commands!{\tt header}}%
\index{Images!commands!{\tt header}}
allows the user to read in certain header variables and set up
particular defaults for the current image.
The defaults established by a call to {\tt header}
are the transfer function\index{Images!coordinate transfer function}
and the {\em world} coordinates.
\index{Coordinate System!World}\index{World Units}
The transfer function is set relative to
the user variables\index{User Variables}
{\tt CRVALX}, {\tt CRVALY}, {\tt CRPIXX}, {\tt CRPIXY}, {\tt CDELTX},
and {\tt CDELTY} (see the examples above).
The limits (or {\em world} coordinates) are set by the header variables and the
range of the indices specified by either
{\tt subimage}\index{Commands!{\tt subimage}}
\index{Images!commands!{\tt subimage}}
or
{\tt quarter}.\index{Commands!{\tt quarter}}
\index{Images!commands!{\tt quarter}}

If the image header does not contain the proper header values or the
image type does not allow them to be retrieved,
the user may specify them using the
{\tt set}\index{Commands!{\tt set}}%
\index{User Variables!commands!{\tt set}}
command.
This, however, should be done {\em after} the image is loaded and
{\em before} the {\tt header} command is called.

\subsection*		{Wedge Displays}
\mylabel{ss:wedges}
\index{Images!wedge displays}

Sometimes, it will be helpful to be able to annotate
a halftone plot with a wedge.
These are useful in that they help to delineate the gray or color
range displayed.
This is done with the
{\tt wedge}\index{Gray-Scale Plots!commands!{\tt wedge}}%
\index{Images!commands!{\tt wedge}}
command.
The arguments to {\tt wedge} command are similar to the
{\tt mtext}\index{Commands!{\tt mtext}}%
\index{Labels!commands!{\tt mtext}}
command.
There are required arguments which specify which side of the current
viewport to display the wedge (this also fixes the orientation);
the displacement from the viewport edge (in character height units); and
the thickness of the wedge (also in character height units).
If the displacement is negative, the wedge will appear inside the viewport;
if it is positive, the wedge will be outside.
Optional arguments include the minimum and maximum values to be used to
display as the extremes of the wedge and a label which will be displayed
on the outside of the wedge.
If a label is desired, the minimum and maximum must be specified.
If the minimum and maximum are not given,
the {\tt wedge} command will display the wedge
in such a way that the entire range of values displayed by the most
recent {\tt halftone}\index{Gray-Scale Plots!commands!{\tt halftone}}%
\index{Images!commands!{\tt halftone}}%
\index{Commands!{\tt halftone}}
command will be visible.

\subsection*		{The Image Transfer Function}
\mylabel{ss:itf}
\index{Images!image transfer function}

Sometimes the raw image is not desired to be displayed;
the image needs to be mapped differently.
For example, often it is useful to display the logarithm of the image.
The command
{\tt itf}\index{Gray-Scale Plots!commands!{\tt itf}}%
\index{Images!commands!{\tt itf}}\index{Commands!{\tt itf}}
facilitates this.
There are currently three types of image transfer functions available:
\begin{itemize}
  \item linear
  \item logarithmic
  \item square-root
\end{itemize}
Remember that the logarithm is a base-10 logarithm.
Initially, the image transfer function is set to linear and is
reset to linear whenever a new device is selected.

\subsection*		{The Image Palette}
\mylabel{ss:palette}
\index{Images!color palette}

For halftone plots (see below), the default is to use a gray scale ramp.
On devices that permit the use of colors beyond the color index 15,
different color lookup tables may be used.
The
{\tt palette}\index{Gray-Scale Plots!commands!{\tt palette}}%
\index{Images!commands!{\tt palette}}%
\index{Commands!{\tt palette}}
command loads a new palette.
On some interactive devices, the change is immediately displayed.
However, the user must be careful to be sure to load the desired palette
{\em prior} to displaying a halftone on a hardcopy device.
There are several pre-defined palettes available.
The {\tt palette} command has one required argument which is the number
of the palette desired.
The list below identifies the current list of pre-defined palettes.
\begin{enumerate}
  \item Gray scale.
  \item A rainbow.
  \item A heat scale.
  \item The IRAF scale.
  \item The AIPS scale.
  \item The PGPLOT scale.
  \item A Saoimage A scale.
  \item A Saoimage BB scale.
  \item A Saoimage HE scale.
  \item A Saoimage I8 scale.
  \item A DS scale.
  \item A cyclic scale (Red to Green to Blue to Red\dots).
\end{enumerate}
If the required argument is negative, the color palette is flipped.
%
\begin{htmlonly}
These palettes are shown in the example in Section~\ref{p:palettes}.
\end{htmlonly}

If a pre-defined palette is not acceptable, one may be defined by the
user using the
{\tt lookup}\index{Gray-Scale Plots!commands!{\tt lookup}}%
\index{Images!commands!{\tt lookup}}\index{Commands!{\tt lookup}}
command.
The {\tt lookup} command maps
the values in the {\bf X} array to the red colors, the {\bf Y} array to
the green colors, and the {\bf ERR} array to the blue colors.
The data in the {\bf PSTYLE} array identifies the ramp intensity level
for the corresponding RGB values.
The value in each of the four arrays must be between 0 and 1.
Colors on the ramp are linearly interpolated from neighboring levels.
An optional argument to {\tt lookup} may be used to flip the resulting
color table.
For example, a trivial gray scale ramp is defined below:
\begin{wiplist}%
  \index{Commands!{\tt data}}
  \index{Commands!{\tt xcolumn}}
  \index{Commands!{\tt ycolumn}}
  \index{Commands!{\tt ecolumn}}
  \index{Commands!{\tt pcolumn}}
  \index{Commands!{\tt lookup}}
  \item {\tt data stdin}\hfill \# Read data from the command line.
\samepage
  \item [{\tt Data Input> }] {\tt 0}\hfill \# Enter the data.
  \item [{\tt Data Input> }] {\tt 1}
  \item [{\tt Data Input> }] {\tt enddata}\hfill \# Finish entering data.
  \item {\tt xcolumn 1}\hfill \# Load the red colors.
  \item {\tt ycolumn 1}\hfill \# Load the green colors.
  \item {\tt ecolumn 1}\hfill \# Load the blue colors.
  \item {\tt pcolumn 1}\hfill \# Load the intensity level values.
  \item {\tt lookup}\hfill \# Load the lookup table.
\end{wiplist}

\section		{Window Adjustments for Images}
\mylabel{s:imwinadj}
\index{Images!window dimensions}

Sometimes the hardest job in dealing with images is getting it to fit
properly with the rest of the plot.
In the previous section, many commands were identified that help to
specify how the image pixel maps onto the world coordinate
system\index{Coordinate System!World}\index{World Units}.
This section goes one step further by dealing with how a
particular image should map onto the
viewport\index{Coordinate System!Viewport}\index{Viewport}
of a particular window\index{Coordinate System!Window}\index{Window}.
In other words, this section will discuss how to correctly set the
aspect ratio of the figure.

At first glance, it might be thought that the best place for \wip\ to
control the aspect ratio of an image is within the
{\tt header}\index{Commands!{\tt header}}%
\index{Images!commands!{\tt header}}
command.
After all, it loads the coordinate transfer function and sets the world
coordinate limits based on the information from the image header.
This would be fine if only one image was ever dealt with on a plot.
But this will fail whenever one image is to be overlaid on
top of another image and the two images have different dimensions.
Hence, it was decided to keep this task under the user's control.
Unfortunately, this means more work for the user.

The rest of this section will introduce the commands necessary to set
the aspect ratio for one image and try to illustrate how to properly
set the aspect ratio for multiple (overlaid) images.
Additional examples of these techniques will be found
in Appendix~\ref{a:samples}.

\subsection*		{The Window Adjustment Commands}
\mylabel{ss:winadj}
\index{Images!window adjustment}

In general, the image plot should be displayed in such a way that
the size of a pixel in the $x$-direction is the same as the size
of a pixel in the $y$-direction.
The {\tt winadj}\index{Commands!{\tt winadj}}%
\index{Coordinate System!commands!{\tt winadj}}
command permits the user to reconfigure the viewport such that the
scale in the $x$- and $y$-directions are equal.
This command must be used {\em after} the image is loaded and
{\em before} the
{\tt header}\index{Commands!{\tt header}}%
\index{Images!commands!{\tt header}}
command is called (\ie before the world coordinates of the plot window are set).
To illustrate, suppose we have an image named ``myimage''
which is dimensioned 80 by 33.
The following commands constrain the viewport so that the halftone
plot fits properly (\ie has an aspect ratio of 80:33):
\begin{wiplist}%
  \index{Commands!{\tt image}}
  \index{Commands!{\tt winadj}}
  \index{Commands!{\tt header}}
  \index{Commands!{\tt halftone}}
  \index{Images!commands!{\tt header}}
  \item {\tt image myimage}\hfill\# Load the image and header information.
\samepage
  \item {\tt winadj 0 nx 0 ny}\hfill\# Set viewport to an 80:33 aspect ratio.
  \item {\tt header}\hfill\# Use the header info to set transfer and limits.
  \item {\tt halftone}\hfill\# Draw the halftone.
\end{wiplist}
Note the use of the user variables\index{User Variables} {\tt nx} and {\tt ny}
to specify the dimensions of the plot.
Also, note that because pixels have a `size' associated with them,
the width of the window (\ie the number of pixels)
must have room for $\mbox{\tt nx} + 1$ pixels.
In order for this to be true,
the range of values in the $x$-direction
must be from 0 to {\tt nx}.
(likewise for the $y$-direction).

One of the most common mistakes in generating image plots is that user
will (a) forget to adjust the window aspect ratio; or (b) call the
{\tt winadj} command {\em after} the
{\tt header}\index{Commands!{\tt header}}%
\index{Images!commands!{\tt header}},
{\tt transfer}\index{Commands!{\tt transfer}}%
\index{Images!commands!{\tt transfer}},
or
{\tt limits}\index{Commands!{\tt limits}}%
\index{Images!commands!{\tt limits}}
commands.
Because the {\tt winadj} command changes both the
viewport\index{Coordinate System!Viewport}\index{Viewport}
{\em and}
world\index{Coordinate System!World}\index{World Units}
coordinates,
it is important that the limits (world coordinates) of the
image be established {\em after} this command is called.

\subsection*            {Panel Movements}
\mylabel{ss:paneladjust}
\index{Images!window adjustment!with panels}

To illustrate how to correctly use the {\tt winadj} command with the
{\tt panel}\index{Commands!{\tt panel}} command,
consider an example where two images should be displayed side by side.
Because the two images have the same dimensional range,
and for this example, it is desired that
there should not be any gap between the images (\ie the boxes should
butt up against each other).
The following macro definition and list of commands might be used:
\begin{wiplist}%
  \index{Commands!{\tt define}}
  \index{Commands!{\tt panel}}
  \index{Commands!{\tt image}}
  \index{Commands!{\tt winadj}}
  \index{Commands!{\tt header}}
  \index{Commands!{\tt halftone}}
  \index{Commands!{\tt box}}
  \index{Commands!{\tt end}}
  \index{Commands!{\tt viewport}}
  \index{Commands!{\tt xlabel}}
  \index{Commands!{\tt mtext}}
  \item {\tt define mytask}\hfill \# \$1=Panel number; \$2=Image plane.
\samepage
  \item [\wipd] {\tt panel -2 1 \$1}\hfill \# Set the requested panel.
  \item [\wipd] {\tt image myimage.fits \$2}\hfill \# Load the image.
  \item [\wipd] {\tt winadj 0 nx 0 ny}\hfill \# Set the aspect ratio.
  \item [\wipd] {\tt header rd}\hfill \# Set the transfer function and limits.
  \item [\wipd] {\tt halftone}\hfill \# Draw the halftone.
  \item [\wipd] {\tt box bcstz bcstz}\hfill \# Draw a box with tick marks.
  \item [\wipd] {\tt if (\$1 $<$ 2) box nhz nvdzy}\hfill \# Label the left box.
  \item [\wipd] {\tt if (\$1 $>$ 1) box nhfz 0}\hfill \# Label bottom of right box.
  \item [\wipd] {\tt end}\hfill \# Finish the macro definition.
  \item {\tt viewport 0.1 0.9 0.1 0.9}\hfill \# Set an initial viewport.
  \item {\tt mytask 1 3}\hfill \# Load the first image.
  \item {\tt mytask 2 4}\hfill \# Load the second image.
  \item {\tt panel 1 1 1}\hfill \# Reset the panel for common RA DEC labels.
  \item {\tt xlabel \esc{ga} (1950)}
  \item {\tt mtext L 3.5 0.5 0.5 \esc{gd} (1950)}
\end{wiplist}
These commands will actually
generate a plot with a gap between the images!
The gap exists even though
the {\tt panel}\index{Commands!{\tt panel}}
command was called with a negative argument.
The gap would remain regardless of the value of
the user variable {\em xsubmar}.

The reason for the gap is because of the call to {\tt winadj}.
Remember that the {\tt winadj} command sets the
world\index{Coordinate System!World}
coordinate system limits and then attempts to adjust
the viewport\index{Coordinate System!Viewport}
coordinate system to have the same aspect ratio.
The {\tt panel} command has established a viewport (which, for this
example, is split into two vertical halves).
Next, the {\tt winadj} command tries to
adjust the aspect ratio of this viewport to match the image.
In general, the resulting viewport from {\tt panel} will not match the
aspect ratio desired by the {\tt winadj} command and the viewport will
adjust to obtain the proper ratio.
As a result, either the top or the right edge (or both) will
need to move inward.

The correct solution to the problem above is to find the aspect
ratio desired for one image and then to scale it by the number of images
in each direction.
Adjusting the initial viewport to this range will not only correct the
problem seen in the example above but also remove the need for the call
to the {\tt winadj} command inside the macro.
In this example, there are two panels in the $x$-direction and only one
panel in the $y$-direction.
Hence, the commands
that will produce the properly scaled plot with the boxes butted together
is shown in the listing that follows
(with the {\tt winadj} command removed from the definition of
the {\tt mytask} macro above):
\begin{wiplist}%
  \index{Commands!{\tt viewport}}
  \index{Commands!{\tt image}}
  \index{Commands!{\tt set}}
  \index{Commands!{\tt winadj}}
  \index{Commands!{\tt panel}}
  \index{Commands!{\tt xlabel}}
  \index{Commands!{\tt mtext}}
  \item {\tt viewport 0.1 0.9 0.1 0.9}\hfill \# Set an initial viewport.
\samepage
  \item {\tt image myimage.fits 1}\hfill \# Do this to get Nx and Ny.
  \item {\tt set \esc{0} 2 * nx}\hfill \# Compute the $x$-direction width.
  \item {\tt winadj 0 \esc{0} 0 ny}\hfill \# Adjust the viewport.
  \item {\tt mytask 1 3}\hfill \# Display the first image.
  \item {\tt mytask 2 4}\hfill \# Display the second image.
  \item {\tt panel 1 1 1}\hfill \# Reset the panel for common RA DEC labels.
  \item {\tt xlabel \esc{ga} (1950)}
  \item {\tt mtext L 3.5 0.5 0.5 \esc{gd} (1950)}
\end{wiplist}
The variable \esc{0} is a user variable and is discussed in
Chapter~\ref{c:uservar}.

\subsection*            {Overlaid Images}
\mylabel{ss:imoverlay}
\index{Images!overlaid images}

A common plotting task is to overlay one image graph on top of another.
Many times, the two (or more) images will have the same dimensions and this
will present no problem.
However, there are many examples where a smaller image is overlaid on
top of a larger image (or vice-versa) or one image has a completely
different scaling (resolution) than another.
Making sure that the aspect ratio of these two images is maintained
does not have to be difficult.

The simple case is when one image is used to define the limits of the
world\index{Coordinate System!World}\index{World Units}
coordinates.
Since the aspect ratio is defined by one image and its
limits specify the world coordinate range of the plot,
there is no need to compute the aspect ratio for the other image.
The world coordinate limits need to be saved, however,
and then loaded after the
second image's
header variables are read and transfer function defined.
The following is an example of this method
(the user variables \esc{1}--\esc{4} are discussed in Chapter~\ref{c:uservar}):
\begin{wiplist}%
  \index{Commands!{\tt viewport}}
  \index{Commands!{\tt image}}
  \index{Commands!{\tt winadj}}
  \index{Commands!{\tt header}}
  \index{Commands!{\tt set}}
  \index{Commands!{\tt halftone}}
  \index{Commands!{\tt box}}
  \index{Commands!{\tt limits}}
  \index{Commands!{\tt levels}}
  \index{Commands!{\tt contour}}
  \item {\tt viewport 0.1 0.9 0.1 0.9}\hfill \# Set an initial viewport.
%\samepage
  \item {\tt image first.fits 1}\hfill \# Do this to get Nx and Ny.
  \item {\tt winadj 0 nx 0 ny}\hfill \# Adjust the viewport.
  \item {\tt header rd}\hfill \# Set the transfer function.
  \item {\tt set \esc{1} x1}\hfill \# Save the current $x$\dots
  \item {\tt set \esc{2} x2}\hfill \# \dots world coordinates.
  \item {\tt set \esc{3} y1}\hfill \# Save the current $y$\dots
  \item {\tt set \esc{4} y2}\hfill \# \dots world coordinates.
  \item {\tt halftone}\hfill \# Display the first image.
  \item {\tt box bcnsthz bcnstvdyz}\hfill \# Draw and label a box.
  \item {\tt image second.fits 2}\hfill \# Load the second image.
  \item {\tt header rd}\hfill \# Load it's transfer function.
  \item {\tt limits \esc{1} \esc{2} \esc{3} \esc{4}}\hfill \# Reset the world limits.
  \item {\tt levels 2 3.5 4 5.5}\hfill \# Set the contour level values.
  \item {\tt contour}\hfill \# Contour the second image.
\end{wiplist}

The more complicated case is when none of the images define the limits of the
world\index{Coordinate System!World}\index{World Units}
coordinates of the plot.
In this case, the limits need to be established by the user.
If these limits are simple (\ie not astronomical coordinates),
then the {\tt winadj} command can be used to establish the world coordinates
and the aspect ratio.
When astronomical coordinates are used, the aspect ratio must be
calculated based on the plot range.
In other words, determine the aspect ratio range using similar
coordinate systems (\eg in arcsec offsets) and then set the limits in
the absolute coordinate system.
The technique shown in the example above of saving the world coordinates
and then reloading them can be well utilized for this case.

\section		{Image Plots}
\mylabel{s:implot}
\index{Images!image plots}

This section presents the various ways of generating a plot with an image.
Each of the commands below requires that certain characteristics
are defined prior to being called.
Most of the image characteristics are described in Section~\ref{s:imfunc}.

\subsection*		{Contour Plots}
\mylabel{ss:contour}
\index{Contour Plots}
\index{Images!image plots!contour}

Before any contour plots may be generated, the user must specify the
contour levels using one or more of the techniques discuss in
Section~\ref{s:imfunc}.

There is only one command in \wip\ to generate a contour plot:
{\tt contour}.\index{Commands!{\tt contour}}%
\index{Images!commands!{\tt contour}}%
\index{Contour Plots!commands!{\tt contour}}
This command, however, permits access to many different methods to use
for drawing the contour plots.
There are up to three optional arguments to the {\tt contour} command.

The first (optional) argument specifies the method used to
draw the contour plot.
By default (or if the method {\tt s} is selected),
\wip\ draws a very simple contour plot.
Selecting contour method {\tt t} means the plot will be drawn with continuous
contour lines (\ie it starts a contour line and follows it until it ends).
Contour method {\tt b} means that the contour should be drawn (just like
contour method {\tt s}) but also allows a blanking value to be specified.
Contour method {\tt l} draws the same contour plot as method {\tt t} except
that it includes contour level labeling.
The second and third optional arguments are ignored
by the contour methods {\tt s} and {\tt t}.

The contour method {\tt t} will draw positive contours values with
solid lines ({\tt lstyle 1}\index{Commands!{\tt lstyle}}) and
negative contour values with dashed lines ({\tt lstyle 2}).
This change of line styles may be overridden by specifying the contour
method as {\tt -t}.
Using the method {\tt -t} will mean that the {\tt contour} command will draw
both positive and negative contour values with the same (\ie the current)
line style.
Note that the
other two methods of drawing contour plots always uses the current line
style regardless of the value of the contour level.

If contour method {\tt b} is selected, an additional optional argument may be
given that specifies the blanking value (image array values equal to this
blanking value are ignored in the plot).
The blanking value, if omitted, defaults to 0.

If contour method {\tt l} is selected, the second (optional) argument may be
provided to specify the spacing along the contour between labels
(in grid cells) and the third (optional) argument is used to specify
the minimum number of cells that must be crossed before drawing
the first contour label.
If not present, the second argument defaults to 16 and the third to 8.

\subsection*		{Gray-Scale Plots}
\mylabel{ss:gray}
\index{Gray-Scale Plots}
\index{Images!image plots!gray-scale}

Gray scale and color image plots are generated in \wip\ using the
{\tt halftone}\index{Commands!{\tt halftone}}%
\index{Images!commands!{\tt halftone}}%
\index{Gray-Scale Plots!commands!{\tt halftone}}
command.
By default, a gray scale halftone is generated.
To create color, the display or hardcopy device must be capable of
handling color maps and one must be loaded either with the
{\tt palette}\index{Commands!{\tt palette}}%
\index{Images!commands!{\tt palette}}%
\index{Gray-Scale Plots!commands!{\tt palette}}
or
{\tt lookup}\index{Commands!{\tt lookup}}%
\index{Images!commands!{\tt lookup}}%
\index{Gray-Scale Plots!commands!{\tt lookup}}
command (see Section~\ref{ss:palette}).
The {\tt halftone} command has two optional arguments that
may be used to delimit the minimum and maximum intensity range
of the gray scale.
By default, the minimum and maximum of the image are used.

Currently, \wip\ uses the most recently loaded lookup table
when displaying a halftone.
Some devices permit this to be changed after the halftone
is displayed but since this is not the case for hardcopy devices,
the user should be careful to insert the desired palette prior to
calling {\tt halftone} when generating a hardcopy plot.

Halftones will only be drawn if the device
is capable of accessing color indices
larger than 15.
Use the command {\tt `echo cmin cmax'}\index{Commands!{\tt echo}}%
\index{Colors!color index range}
to display the current
values of {\em CMIN}\index{Colors!CMIN} and {\em CMAX}\index{Colors!CMAX}).

\subsection*		{Cross Section Plots}
\mylabel{ss:xsectn}
\index{Cross Section Plots}
\index{Images!image plots!cross section}

There is a crude cross section plot available in \wip.
It is the
{\tt hi2d}\index{Commands!{\tt hi2d}}%
\index{Images!commands!{\tt hi2d}}%
\index{Gray-Scale Plots!commands!{\tt hi2d}}
command.
Each cross section slice consists of a hidden line histogram.
A bias is applied to each successive slice in order to raise it ``above''
(the $y$-direction) the previous slice.
This bias is the only required argument and is entered
in the same units as the image data.

Two other optional arguments may be specified to control the display of
the cross section plot.
The first optional argument controls the slant of the plot
by acting as an offset to successive cross sections in the $x$-direction.
If this argument is positive ($>$ 0), it creates a plot that appears to
slant to the right; if negative ($<$ 0), slant left.
The default value for this argument is no slant effect (argument is 0).
The last (optional) argument is a flag to specify whether the indices in
the $x$-direction are centered or represent the left edges of the bins.
The default value for this argument is to center the bins (argument is 0).

\subsection*		{Drawing A Beam}
\mylabel{ss:beam}
\index{Images!beam}

The {\tt beam}\index{Commands!{\tt beam}}%
\index{Images!commands!{\tt beam}}
command was created because many astronomical image files contain information
about the beam size and orientation used to observe the source.
This information can be used to display the beam providing a graphical
presentation of the resolution of the image.
The {\tt beam}\index{Commands!{\tt beam}} command
draws a beam at the current cursor position (with an optional offset).
This beam is outlined in the current color using the current line properties.
The required arguments are the beam major axis, the minor axis, and the
position angle.
As in all astronomical images, the position angle is defined as 0 degrees
towards the north and increases counter-clockwise.

The optional arguments include offsets, a scaling factor, a fill color
index, and a background rectangle color index.
The offsets permit the center of the beam to be shifted relative to the
current cursor position.
The scaling factor provides greater control over the aspect ratio.
By default, the scaling factor (or if it is provided and is negative) is 
set to $15 \times \cos(\mbox{declination})$
(which is proper for RA-Dec type scaling).
The fill color index is used to draw the beam with a filled color.
The background color index (if greater than or equal to zero)
is used as the color to draw the (filled) bounding box of the beam.

The optional offsets are provided in units of the width and height of the beam.
This permits the user to easily shift the beam location.
As an example of how to use the optional offsets, consider drawing
a beam in the lower right corner of the window.
Rather than computing the position and then guessing how much to move it
so it is slightly offset from the box (and then having to redo it all
again if the coordinates need to be changed!), the offsets permit an
easy alternative:
\begin{wiplist}%
  \index{Commands!{\tt move}}
  \index{Commands!{\tt beam}}
  \index{Commands!{\tt image}}
  \index{Commands!{\tt header}}
  \index{Images!commands!{\tt image}}
  \index{Images!commands!{\tt header}}
  \item {\tt image myimage}\hfill\# Load the image and header information.
\samepage
  \item {\tt header rd}\hfill\# Set transfer fn and limits.
  \item {\tt halftone}\hfill\# Draw the halftone.
  \item {\tt move x2 y1}\hfill\# Move to the lower right corner.
  \item {\tt beam bmaj bmin bpa -0.8 0.8}\hfill\# Draw the beam.
\end{wiplist}
In the example above, the parameters {\tt bmaj bmin bpa} are dynamically
read from the image header.
It is possible some conversion of these terms may be necessary.
In this example it was assumed
that the major and minor sizes were in the same units as the Y-axis and
the position angle was in degrees.

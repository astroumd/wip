%% This file is to be included by latex in wip.tex
%
% Macros.
%
\mylabel{c:macros}
\myfile{macros.tex}

\section		{General Concepts}
\mylabel{s:macgeneral}
\index{Macros!general concepts}

Macros provide a way to execute a list of \wip\ commands
repetitively.
In much the same way,
a programmer defines a subroutine
which will do a particular task for different programs or different
aspects of the same program.
Hence, macros provide an easy way to tie together several plotting
instructions under one name.
Whenever this defined name is called,
the commands associated with it are executed.
Macros in \wip\ may be defined, edited, executed, and deleted.
In fact, there always exists at least one defined macro:
the command buffer\index{Macros!buffer}.

Most of the commands that the user types are saved in a macro called the
command buffer.
There are a few commands, however, which do not typically need to be saved
(\eg {\tt echo}\index{Commands!{\tt echo}},
{\tt show}\index{Commands!{\tt show}},
and {\tt help}\index{Commands!{\tt help}}%
\index{Help!commands!{\tt help}}).
These commands are generally used diagnostically and not graphically.
For the most part, \wip\ ``knows'' which commands to save and which to
not save in the command buffer.
If a command is not saved, it may still be stored in the command
buffer by explicitly entering it using the
{\tt insert}\index{Commands!{\tt insert}}%
\index{Macros!commands!{\tt insert}}
command (see Section~\ref{s:macedit}).

As an example of which commands are typically saved and which commands are not,
suppose that during a typical \wip\ interactive session a user typed
the following commands:
\begin{wiplist}%
  \index{Commands!{\tt device}}
  \index{Commands!{\tt limits}}
  \index{Commands!{\tt box}}
  \index{Commands!{\tt show}}
  \index{Commands!{\tt xlabel}}
  \index{Commands!{\tt ylabel}}
  \index{Commands!{\tt erase}}
  \index{Commands!{\tt move}}
  \index{Commands!{\tt draw}}
  \item {\tt device /xwindow}
  \item {\tt limits 0 10 0 5}
  \item {\tt box bcnst bcnstv}
  \item {\tt show c}
  \item {\tt xlabel time (sec)}
  \item {\tt ylabel Number}
  \item {\tt erase}
  \item {\tt box bcnst bcnstv}
  \item {\tt move 0 2}
  \item {\tt draw 0 3}
  \item {\tt draw 2 3}
\end{wiplist}
At this point, if the user issued the \wip\ command
{\tt list},\index{Commands!{\tt list}}%
\index{Macros!commands!{\tt list}}
only the following commands would appear:
\begin{wiplist}%
  \index{Commands!{\tt list}}
  \index{Commands!{\tt limits}}
  \index{Commands!{\tt box}}
  \index{Commands!{\tt xlabel}}
  \index{Commands!{\tt ylabel}}
  \index{Commands!{\tt move}}
  \index{Commands!{\tt draw}}
  \item [\wipp] {\tt list}
\samepage
  \begin{enumerate}\itemsep=0pt
  \item {\tt limits 0 10 0 5}
  \item {\tt box bcnst bcnstv}
  \item {\tt xlabel time (sec)}
  \item {\tt ylabel Number}
  \item {\tt box bcnst bcnstv}
  \item {\tt move 0 2}
  \item {\tt draw 0 3}
  \item {\tt draw 2 3}
  \end{enumerate}
  \item [\wipp] \ %
\end{wiplist}
Note that the {\tt device}, {\tt show}, and {\tt erase} commands are
not present in the listing.
This ``listing'' actually holds the contents of the command buffer or, more
precisely, the macro named {\em buffer}\index{Macros!buffer}.

The sections that follow will explain 
how to define macros; pass arguments into the macros; edit,
execute, and remove macros.
In Chapter~\ref{c:flow}, more advanced use of macros will be illustrated.
Appendix~\ref{a:wipinit} presents how to have macros automatically
defined whenever \wip\ is started.
Finally, Appendix~\ref{a:samples} illustrates several
examples of macro definitions and their uses.

\section		{Creating Macros}
\mylabel{s:macdefine}
\index{Macros!how to create}

Macros are created using the
{\tt define}\index{Commands!{\tt define}}%
\index{Macros!commands!{\tt define}}
command.
The {\tt define} command has only one required argument: the macro name.
Macros may have any name as long as it does not conflict with any \wip\
command or any other previously defined macro.
Macro names
are case insensitive (\ie {\tt MyLabel} is considered the same macro
name as {\tt mylabel}).
All valid commands entered after the {\tt define} command are
inserted into the named macro.
When \wip\ receives a proper {\tt define} command, the prompt changes to
{\rm DEFINE\verb+>+} to remind the user they are in definition mode,
and this prompt remains until the next
{\tt end}\index{Commands!{\tt end}}\index{Macros!commands!{\tt end}}
command terminates the definition.

Each command entered as part of the definition of a macro
is checked that it is either a valid command or another defined macro
and is then stored for later use.
For this reason, macros may not reference other macros that have not yet
been defined.\footnote{This is to prevent macros from calling themselves.}
Of course, previously defined macros may be used within
another macro definition.

If a macro already exists with the same name as the macro being defined
a warning is issued to the user and the previous definition is removed.
The warning is issued to indicate to the user that a conflict of names
exists and that the user should be careful that the proper macro is
going to be used in the plot file.

\section		{Macro Arguments}
\mylabel{s:macarguments}
\index{Macros!passing arguments}

There are 99 special arguments that may be used within the definition of
a macro: \$1, \$2, \ldots, \$99, and \$0 has a historic meaning
which is the same as \$10. Version 2.3 and below of WIP only
supported 10 arguments.
When a macro is executed, any parameters following the macro name are
assigned, in order, to the (up to 99) variables. 
Whenever a command in the macro references one of these variables,
the command line parameter is substituted in its place.
The input arguments to a macro are specified in the same fashion as when
providing arguments to a command:
tokens separated by white space or commas.
And, as usual, arguments may be either numbers or text strings.
If a requested argument is not found, the special argument is skipped.
As an example of how to define a macro with arguments, consider
the following example (remember the \# character denotes the beginning
of a comment that lasts until the end of the line):
\begin{wiplist}%
  \index{Commands!{\tt define}}
  \index{Commands!{\tt move}}
  \index{Commands!{\tt label}}
  \index{Commands!{\tt end}}
  \index{Commands!{\tt limits}}
  \item [\wipp] {\tt define mylabel}\hfill\# Use as mylabel x y label
\samepage
  \item [\wipd] {\tt move~~~~\$1~~~\$2}\hfill\# Move to position (x, y).
  \item [\wipd] {\tt label \$3 \$4 \$5 \$6}
  \item [\wipd] {\tt end}\hfill\# End of macro definition.
  \item [\wipp] {\tt limits 0 10 0 10}\hfill\# Set the world coordinates.
  \item [\wipp] {\tt mylabel 1~~~4 A~~~ string\ \ to~~~present.}\hfill
    \# Call the macro.
  \item [\wipp] {\tt mylabel 1 8 6th~~~~argument omitted.}\hfill\# Pass
    only 5 args.
  \item [\wipp] {\tt mylabel 4 5 A string of 5 elements.}\hfill\# Pass
    too many args.
\end{wiplist}
In short, {\tt mylabel} moves to the specified position and writes a label
consisting of up to four words separated by single spaces.
Notice that the second time the {\tt mylabel} macro is called,
it is only called with five arguments.
In this case, the sixth argument is being ignored and has no effect
on the text labeled.
When the macro is executed, the command parser will substitute the
command line arguments.
When the parser does not find a requested argument (in this example, the
sixth), it silently ignores it.
The last time the macro is called, too many arguments are passed into it.
Unneeded arguments to macros are ignored.
So, in this example, the string that will actually appear will only
consist of ``A string of 5''; the word ``elements'' will be discarded.

Note also that, in the above example,
the number of spaces separating the arguments
in the first two calls to the macro does not matter.
This is also true of the {\tt move} command in the macro definition.
However, in the {\tt label} command, the spacing and location of the
arguments is important!
The macro command
\begin{wiplist}%
  \index{Commands!{\tt label}}
  \item [\wipd] {\tt label \$3\ \$4\ \$5\ \$6}
\end{wiplist}
will produce quite different results from the statement
\begin{wiplist}%
  \index{Commands!{\tt label}}
  \item [\wipd] {\tt label \$3\$4\$5\$6}
\end{wiplist}
This distinction may be quite useful when defining a macro.
In particular, a macro command like
\begin{wiplist}%
  \index{Commands!{\tt set}}
  \item [\wipd] {\tt set \esc{0} 0\$1}
\end{wiplist}
will set the user variable\index{User Variables} \esc{0} to 0 if the
macro is called without any arguments.
This is because the macro argument {\tt \$1} will be skipped if no arguments
are present and the resulting command will be the same as if the command
typed was:
\begin{wiplist}%
  \index{Commands!{\tt set}}
  \item [\wipd] {\tt set \esc{0} 0}
\end{wiplist}

An advanced feature of macro arguments is that of returned values.
If used properly, a macro argument may be set in a macro and the new
value ``returned'' at the end.
In this way, internal counters for the
{\tt loop}\index{Commands!{\tt loop}} command (see section~\ref{s:loop})
can be updated.
Several examples of this technique are shown in the section
containing sample plot files and figures (Appendix~\ref{a:samples}).
It is important to
note that the macro argument that is used (and returned) is always
a user variable (see Chapter~\ref{c:uservar})
and never a literal value or string.

Notice that comments\index{Comments} are used whenever possible
in the macro definition,
but are not present at the end of any text labeling command.
A useful convention to follow is to identify the arguments to a macro
in the definition of the macro and identify which
user variables\index{User Variables},
if any, are changed or used by the macro.
When working interactively, comments are ignored and are not saved in
the command buffer.
Working in {\rm INSERT}\index{Macros!commands!{\tt insert}} mode,
as well as when plot files are created
external to \wip\ in an editor, will, however, retain what comments are typed.
Hence if you define a macro externally or modify the plot file in an editor,
you can place comments and label the inputs to the macro.
This will greatly help you at a later time when you return to this macro
and try to figure out what it does and what the inputs should be when
calling it.
Neither comments or identification of arguments are required, but like
good programming, they are extremely valuable.

\section		{Editing Macros}
\mylabel{s:macedit}
\index{Macros!editing}

All macros, including the command buffer\index{Macros!buffer}
may be edited.
The three commands used to edit a macro and the command buffer are
described in the sections that follow.
It should be understood that whenever the word macro is used in the
discussions that follow, it includes all user defined macros {\em and}
the command buffer.

\subsection*		{List}
\mylabel{ss:maclist}
\index{Commands!{\tt list}}\index{Macros!commands!{\tt list}}

Whenever a user edits a macro,
the edit command usually requires a line number as an argument.
The line number tells the edit command where in the macro to begin to edit.
These line numbers are associated with the numbers presented
whenever the macro is listed.
The {\tt list} command provides a way to display the contents of a macro.

Without any arguments, {\tt list} will display the entire contents of
the command buffer.
Each command will be proceeded by a number representing the order it
is executed.
As the number of commands in a macro increases,
this may make it difficult to ``see'' all
the commands on your display.

If only one argument is given to the {\tt list} command,
it is assumed to be the starting command line number.
All commands, including the one specified, are displayed until the end
of the buffer.
So, the command
\begin{wiplist}%
  \index{Commands!{\tt list}}
  \item [\wipp] {\tt list 5}
\end{wiplist}
will list the fifth command in the command buffer, then the sixth, and
so on, until the end of the command buffer.

If a second argument is present, it is the line number at which the listing
will stop.
This is useful when the user only wants to display the first few lines
of a macro.
If this second number is greater than the number of commands in the buffer,
then the {\tt list} command will present all commands from the starting
line number to the end of the buffer.
In this way, the command
\begin{wiplist}%
  \index{Commands!{\tt list}}
  \item [\wipp] {\tt list 5 10000}
\end{wiplist}
will present every command in the command buffer starting with the fifth.
(This is identical to the command shown before.)

Finally, a third argument, if present, will inform the {\tt list}
command which macro to list.
If the user wishes to display the entire macro, then the command
\begin{wiplist}%
  \index{Commands!{\tt list}}
  \item [\wipp] {\tt list 1 10000 mymacro}
\end{wiplist}
will surely be sufficient to cover the entire listing
of the macro {\tt mymacro}.
Small sections of the macro may be specified by judicious selection of
the input line number arguments.

\subsection*		{Insert}
\mylabel{ss:macins}
\index{Commands!{\tt insert}}\index{Macros!commands!{\tt insert}}

Commands are never typed in the way we want them on the
first time through a plot.
Often, a color change or line style adjustment extends further
in the plot file than we first expected.
Another command is needed just a few lines prior to the current command.
Or perhaps we have, too hastly, finished a definition of
a macro and forgot to include an important command.
These situations can be quite easily fixed with the {\tt insert} command.
The {\tt insert} command may have up to two optional arguments
(discussed below).

When the {\tt insert} command is called, it places the user into insert mode.
This is demarked by the prompt changing to {\rm INSERT\verb+>+}.
All commands typed at this point are checked that they are valid
commands (if not, they are ignored) and are inserted in the
proper order (without being executed).
Insert mode is terminated by entering the {\tt end} command.

Without any arguments, {\tt insert} starts inserting commands at the end
of the command buffer.
This is the same as interactive mode except that the commands typed are
not executed.
If one argument is present, it is considered the insert line number and
all commands will be inserted {\em before} that line in the macro.
Hence, if the command buffer has three commands currently in it and the
user types
\begin{wiplist}%
  \index{Commands!{\tt insert}}
  \item [\wipp] {\tt insert 10000}
\end{wiplist}
the effect will be the same as if the user had typed
\begin{wiplist}%
  \index{Commands!{\tt insert}}
  \item [\wipp] {\tt insert 4}
\end{wiplist}
or just
\begin{wiplist}%
  \index{Commands!{\tt insert}}
  \item [\wipp] {\tt insert}
\end{wiplist}
If two arguments are present, then the first argument specifies
which line number to insert commands before (as above) and the second
argument is the name of the macro in which commands will be inserted.
As an example, if the user has a macro defined called {\tt mymacro},
then the command
\begin{wiplist}%
  \index{Commands!{\tt insert}}
  \item [\wipp] {\tt insert 1 mymacro}
\end{wiplist}
will allow the user to insert commands at the beginning of the named macro.

\subsection*		{Delete}
\mylabel{ss:macdel}
\index{Macros!removing}\index{Macros!commands!{\tt delete}}
\index{Commands!{\tt delete}}

Just as users may wish to insert commands in either the command buffer
or their own macros,
they will want to remove certain commands.
The {\tt delete} command permits the user to specify lines of commands
or even entire macros to be deleted.
A macro is automatically deleted whenever the number of commands
it contains reaches zero.\footnote{The command buffer is never removed;
even if there are no commands in it.}
Deleting an entire macro is sometimes useful if the user wishes to define
another macro with the same name.

The {\tt delete} command may have up to three optional arguments.
With no arguments present, it removes the last line in the command buffer.
The first two arguments are considered as command line numbers.
With only one argument, the specified command line in the command buffer
is removed.
If the line number is greater than the number of commands present in the
macro, the last command is removed.
If two arguments are present, then the inclusive range of commands are
removed from the command buffer.
Finally, if all three optional arguments are present, then the lines
removed are from the named macro.
Hence, the command
\begin{wiplist}%
  \index{Commands!{\tt delete}}
  \item [\wipp] {\tt delete 4 4 mymacro}
\end{wiplist}
will remove only the forth line from the macro {\tt mymacro}.
The command
\begin{wiplist}%
  \index{Commands!{\tt delete}}
  \item [\wipp] {\tt delete 1000 1000 mymacro}
\end{wiplist}
will remove only the last command from the macro {\tt mymacro} (if there
are fewer than 1000 commands in the macro).

\section		{Executing Macros}
\mylabel{s:maccall}
\index{Macros!calling}
\index{Macros!executing}

Once a macro has been defined, it is executed just like any
of the other \wip\ commands.
To execute a macro, just call it by name.
The macro name may also be followed by any required or
optional arguments as is necessary.

In addition, a macro may be executed without being
saved in the command buffer.
This is accomplished with the
{\tt playback}\index{Commands!{\tt playback}}%
\index{Macros!commands!{\tt playback}}
command.
Without any arguments,
{\tt playback} executes the commands in the command buffer.
This is an extremely useful command when the constructed plot
begins to become quite complicated.
The optional arguments to {\tt playback} is the name of the macro
to be executed and any arguments needed by the macro.
Since the {\tt playback} command is not saved in the command buffer,
this permits a user the opportunity to test their macros.
However, the real usefulness of the {\tt playback} command is in
repeating the plot file constructed.

\section		{Saving and Recalling Macros}
\mylabel{s:macwrite}
\index{Macros!saving}

Once a user has defined a useful macro, they will undoubtedly wish to
save them to a file for use again.
This is accomplished with the
{\tt write}\index{Commands!{\tt write}}
\index{Macros!commands!{\tt write}}
command.
The {\tt write} command permits named macros and the command buffer to
be written to an ASCII\index{ASCII} text file.
The only required argument to the {\tt write} command
is the name of the output file.
With no other arguments, the command buffer is written to the named file.
If any other arguments are present after the file name,
they are considered to be the macros to be saved and these named macros
are written to the specified file.
Multiple macros may be written to the same file with only one
{\tt write} command.
If the macro name listed is ``all'', then all macros currently defined
{\em and} the command buffer will be written to the named file.

To retrieve macro definitions, \wip\ provides three
commands\index{Macros!recalling}.
Each command treats the input file differently and are explained in
brief detail in the following list.
\begin{description}
  \item [{macro {\em filename} --}]
    \index{Commands!{\tt macro}}\index{Macros!commands!{\tt macro}}
    Used to define macros.
    Only macro definitions are read in; any other commands are ignored.
    The macro is not executed but is now available for further execution.
  \item [{read {\em filename} --}]
    \index{Commands!{\tt read}}\index{Macros!commands!{\tt read}}
    Used to read in plot commands into the command buffer\index{Macros!buffer}.
    Commands are placed into the command buffer without executing them
    (this is similar to the {\tt insert}
    \index{Commands!{\tt insert}}%
    \index{Macros!commands!{\tt insert}}%
    command).
    If the file contains macro definitions,
    they will be not placed into the command buffer or
    defined; they must be entered using the {\tt macro} command.
  \item [{input {\em filename} --}]
    \index{Commands!{\tt input}}\index{Macros!commands!{\tt input}}
    Commands are executed just as if they were typed in interactive mode
    but are not saved in the command buffer.
    Macro definitions are acceptable but will overwrite (redefine)
    the macro.  A message is printed when a macro is redefined.
    Only exact matches to the macro name are overwritten.
    Because of this, care should be taken to insure that
    the proper macro is being executed.
\end{description}

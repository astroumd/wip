%% This file is to be included by latex in wip.tex
%
% Sample Plots.
%
\mylabel{a:samples}
\myfile{samples.tex}

This appendix contains several examples of plot files generated using
many of the \wip\ commands described in this manual.
Because of the obvious limitations of print, commands like {\tt color} are
omitted from this appendix.
Wherever possible, each command is followed by a comment statement.
While these comments, marked by the comment
character\index{Comments} (\#),\index{Comments!comment character(\#)}
are not required, they do provide information as to why certain commands
are included or are necessary to generate a certain aspect of a plot.
The user is strongly encouraged to include comments whenever possible;
especially in macro definitions.
All plots in this appendix may be reproduced by typing the commands
listed in the accompanying plot files.

\begin{description}
  \item[Figure~\ref{p:basic}] is a very basic plot demonstrating many of the
    commands that control the simple graphical attributes described
    in Chapter~\ref{c:concepts}.

  \item[Figure~\ref{p:simple}] demonstrates how to read data in from
    an external file.

  \item[Figure~\ref{p:text}] shows several examples of how to annotate a plot
    with text and also illustrates the various fonts available in \wip.

  \item[Figure~\ref{p:greek}] displays the lower and upper case Greek letters
    and the Roman characters needed to generate them.

  \item[Figure~\ref{p:panel}] presents a simple example of how to partition
    a single plot surface into sub-regions or panels;
    illustrates the various ways that the
    {\tt box}\index{Commands!{\tt box}}\index{Boxes!commands!{\tt box}}
    command may be called;
    and demonstrates how to define and pass arguments to a macro\index{Macros}.

  \item[Figure~\ref{p:cos}] presents a more complicated example of how to use
    macros and how to generate array entries.
    This example also illustrates the use of the {\tt set} and {\tt loop}
    commands.

  \item[Figure~\ref{p:symbol}]\index{Symbols!markers} plots the
    standard graph markers used by the
    {\tt symbol}\index{Commands!{\tt symbol}}\index{Symbols} command.

  \item[Figure~\ref{p:hershey}]\index{Symbols!Hershey symbols} continues
    the previous example by illustrating how to display the Hershey
    symbols used by \wip.  The macro, as listed, will display all
    currently available Hershey symbols, but only the first page is shown
    in this example.
    This example also makes use of macros\index{Macros},
    panels\index{Commands!{\tt panel}}, and the
    {\tt loop}\index{Commands!{\tt loop}} command.

  \item[Figure~\ref{p:fit}] is an example of how to generate a fit
    and display it.  The different methods of fits are illustrated.

  \item[Figure~\ref{p:image}] is a simple example of how to generate an
    image halftone and wedge and also lists a typical header for a
    FITS\index{FITS}\index{Images!image formats!FITS} image file.

  \item[Figure~\ref{p:image2}] expands on the previous example by
    demonstrating how to extract a subimage and display it in a
    different region on the view screen.  In addition, it illustrates
    the power of user variables\index{User Variables} and macros in
    moving to various regions of the display.

  \item[Figure~\ref{p:overlay}] expands on the previous two image examples by
    showing how to overlay two images that have different resolutions.

\begin{htmlonly}
  \item[Figure~\ref{p:cimage}] is the same as Figure~\ref{p:image}
    except that it is done in color.

  \item[Figure~\ref{p:palettes}] displays a collection of color palettes.
    This example makes use of the {\tt loop} and {\tt wedge} commands.
\end{htmlonly}

\end{description}

\clearevenpage
\insertplot{Basic graphics commands}{basic}{A simple plot.}{}{}

\clearpage
\insertplot{Reading data from an external file}{simple}%
{A plot using an external data file.}%
{Contents of the external file {\tt data.dat}}{Examples/data.dat}

\clearpage
\insertplot{Fonts and different ways to present text}{text}%
{A plot demonstrating various text capabilities.}{}{}

\clearpage
\insertplot{Roman to Greek character mapping}{greek}%
{A plot demonstrating how to generate Greek characters.}{}{}

\clearpage
\insertplot{Macros and subdivision of a view surface}{panel}%
{A plot demonstrating different panels.}{}{}

\clearpage
\insertplot{Generating data internally with macros}{cos}%
{A plot with data generated internally.}{}{}

\clearpage
\insertplot{Symbols used to plot points}{symbol}%
{A plot of the standard symbol markers.\index{Symbols}}{}{}

\clearpage
\insertplot{Hershey Symbols}{hershey}%
{A plot of a few Hershey symbols.\index{Symbols!Hershey symbols}}{}{}

\clearpage
\insertplot{Fitting}{fit}%
{A plot showing different ways to fit data.\index{Fitting!an example}}{}{}

\clearpage
\insertplot{A simple image}{image}{A basic image, halftone plot, and wedge.}%
{Contents of the image header file}{Examples/orion.hed}

\clearpage
\insertplot{Multiple image layout and more macros}{image2}%
{A plot demonstrating the use of subimages.}{}{}

\clearpage
\insertplot{Overlay of multiple images}{overlay}%
{A plot demonstrating how to overlay different images.}{}{}

\begin{htmlonly}
\clearpage
\insertplot{A simple color image}{cimage}%
{A basic color image plot.  The contents of the image
header file are the same as those seen in Example~\ref{p:image}.}{}{}

\clearpage
\insertplot{Color Palettes}{palettes}%
{A collection of the pre-defined color palettes.}{}{}
\end{htmlonly}

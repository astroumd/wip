%% This file is to be included by latex in wip.tex
%
% Uservars.
%
\mylabel{c:uservar}
\myfile{uservar.tex}

\section		{General Concepts}
\mylabel{s:vargeneral}
\index{User Variables!general concepts}

Available to the \wip\ user are a host of variables.
Many of these variables are associated with the general
graphical attributes discussed in Chapter~\ref{c:concepts};
some with the current image (Section~\ref{c:images});
and some are available to perform simple to complex calculations.
Whenever the user accesses a graphics tool in \wip,
they are probably setting the value of an internal variable.
In general, these variables need not (and should not) be accessed directly.
These variables may, however,
be used to probe the current values and may also
be used as arguments to their respective commands.
Some of the variables, however, can only be accessed
directly and are designed for that use.
The sections that follow will describe the variables that are currently
available to the user (Section~\ref{s:varpredef})
and the tools used to access them (Section~\ref{s:varset}).
In addition, there are ways the user can create and destroy
their own user variables (Section~\ref{s:varnew}).

User variables enable the \wip\ user to perform quite
complicated tasks easily.
In addition, it allows the user to perform commands without knowledge of
the actual current {\em value} of the user variable.
For example, as will be shown in Section~\ref{s:varpredef}, the user
variables that specify the limits of the current
world (or User coordinate) window are
{\tt x1}, {\tt x2}, {\tt y1}, and {\tt y2}.
Suppose an astronomer reads in a list of $x$ and $y$ values that
correspond to right ascension and declination positions, respectively.
If the limits are determined automatically with the
{\tt limits}\index{Commands!{\tt limits}} command
(\ie the command called with no arguments),
then the $x$ axis values will increase
to the right.
But right ascension increases to the left.
The following \wip\ listing shows how, with the use of user variables,
this can be corrected\ldots without knowing the actual values of
{\tt x1}, {\tt x2}, {\tt y1}, and {\tt y2}:
\begin{wiplist}%
  \index{Commands!{\tt data}}
  \index{Commands!{\tt xcolumn}}
  \index{Commands!{\tt ycolumn}}
  \index{Commands!{\tt limits}}
  \item {\tt data myfile}\hfill\# Access the data file.
\samepage
  \item {\tt xcolumn 1}\hfill\# Read the RA positions.
  \item {\tt ycolumn 2}\hfill\# Read the Dec.\ positions.
  \item {\tt limits}\hfill\# Autoscale the limits.
  \item {\tt limits x2 x1 y1 y2}\hfill\# Flip only the $x$ axis limits.
\end{wiplist}
Note that by calling {\tt limits} with the four user variables as
arguments, the user does not need to know the {\em actual} values or
contents of the user variables, just the names of the variables.
Furthermore, by switching the order of the $x$-axis arguments, the
orientation of the $x$-axis direction has been flipped.

This works because \wip\
checks for arguments and expects literal numeric values
whenever it is passed a command.
If \wip\ cannot ``understand'' the argument as a number it searches the
list of user variables for a match.
If a match is made, the current value of the user variable is
substituted internally for the corresponding command argument.

\section		{Pre-Defined Variables}
\mylabel{s:varpredef}
\index{User Variables!pre-defined variables}

There are two types of user variables available:
numeric variables and string variables.
These two types cannot be mixed at any time!
Attempting to mix these types of user variables will either create an
error message or produce unpredictable and undesirable results.

Tables~\ref{t:uservar1} -- \ref{t:uservar3}
identify the current predefined numeric user variable
names, a description,
and a (typical) command which may be used to set it
(the index at the end of the manual will direct the reader to a more
detailed description of each of the commands mentioned as well as
possibly a few more that are applicable).
The tables are broken up into areas where the user variable is most applicable.
\begin{table}
  \caption{User Variables -- Coordinate Systems}
  \mylabel{t:uservar1}
  \index{User Variables!table of}
  \index{User Variables!commands!{\tt set}}
  \index{Commands!{\tt set}}
  \index{Commands!{\tt limits}}
  \index{Commands!{\tt viewport}}
  \index{Commands!{\tt cursor}}
  \index{Commands!{\tt ticksize}}
  \index{Commands!{\tt show}}
  \index{Coordinate System!commands!{\tt limits}}
  \index{Coordinate System!commands!{\tt viewport}}
  \index{Cursor!commands!{\tt cursor}}
  \index{Boxes!commands!{\tt ticksize}}
  \centering
  \setlength{\tabentrylen}{\textwidth}
  \addtolength{\tabentrylen}{-1.6in}
  \begin{tabular}{|c|c|p{\tabentrylen}|} \hline\hline
    \multicolumn{1}{|c}{Variable}
    & \multicolumn{1}{|c|}{Command}
    & \multicolumn{1}{|c|}{Description} \\ \hline
    && \\
    x1     & {\tt limits}   & World Coordinate left $x$-axis limit value. \\
    x2     & {\tt limits}   & World Coordinate right $x$-axis limit value. \\
    y1     & {\tt limits}   & World Coordinate lower $y$-axis limit value. \\
    y2     & {\tt limits}   & World Coordinate upper $y$-axis limit value. \\
    vx1    & {\tt viewport} & Viewport Coordinate left $x$ axis limit. \\
    vx2    & {\tt viewport} & Viewport Coordinate right $x$ axis limit. \\
    vy1    & {\tt viewport} & Viewport Coordinate lower $y$ axis limit. \\
    vy2    & {\tt viewport} & Viewport Coordinate upper $y$ axis limit. \\
    cx     & {\tt cursor} & Current World Coordinate $x$ value (pen position).\\
    cy     & {\tt cursor} & Current World Coordinate $y$ value (pen position).\\
    xtick  & {\tt ticksize}
           & Major tick mark interval along $x$ axis (World units). \\
    ytick  & {\tt ticksize}
           & Major tick mark interval along $y$ axis (World units). \\
    nxsub  & {\tt ticksize} & Number of minor tick marks along the $x$ axis. \\
    nysub  & {\tt ticksize} & Number of minor tick marks along the $y$ axis. \\
    xsubmar & {\tt submargin}
           & Panel gap size in $x$-direction (char. height units). \\
    ysubmar & {\tt submargin}
           & Panel gap size in $y$-direction (char. height units). \\
    && \\ \hline\hline
  \end{tabular}
\end{table}
\begin{table}
  \caption{User Variables -- Images}
  \mylabel{t:uservar2}
  \index{User Variables!table of}
  \index{User Variables!commands!{\tt set}}
  \index{Commands!{\tt set}}
  \index{Commands!{\tt header}}
  \index{Commands!{\tt image}}
  \index{Commands!{\tt subimage}}
  \index{Commands!{\tt slev}}
  \index{Images!commands!{\tt header}}
  \index{Images!commands!{\tt image}}
  \index{Images!commands!{\tt slev}}
  \index{Images!commands!{\tt subimage}}
  \index{Images!user variables!table of}
  \index{Contour Plots!setting levels!{\tt slev}}
  \centering
  \setlength{\tabentrylen}{\textwidth}
  \addtolength{\tabentrylen}{-1.6in}
  \begin{tabular}{|c|c|p{\tabentrylen}|} \hline\hline
    \multicolumn{1}{|c}{Variable}
    & \multicolumn{1}{|c|}{Command}
    & \multicolumn{1}{|c|}{Description} \\ \hline
    && \\
    subx1  & {\tt subimage} & Left $x$ pixel number of current subimage. \\
    subx2  & {\tt subimage} & Right $x$ pixel number of current subimage. \\
    suby1  & {\tt subimage} & Lower $y$ pixel number of current subimage. \\
    suby2  & {\tt subimage} & Upper $y$ pixel number of current subimage. \\
    crvalx & {\tt header}
           & Reference pixel value in the $x$-direction (World units). \\
    crpixx & {\tt header}   & The reference pixel in the $x$-direction. \\
    cdeltx & {\tt header}
           & The $x$-direction step size between pixels (World units). \\
    crvaly & {\tt header}
           & Reference pixel value in the $y$-direction (World units). \\
    crpixy & {\tt header}   & The reference pixel in the $y$-direction. \\
    cdelty & {\tt header}
           & The $y$-direction step size between pixels (World units). \\
    slevel & {\tt slev}
           & The scaling factor applied to the contour level array. \\
    immin  & {\tt image}    & Minimum intensity value of the current image. \\
    immax  & {\tt image}    & Maximum intensity value of the current image. \\
    nx     & {\tt image}
           & Number of pixels in the current image's $x$ direction. \\
    ny     & {\tt image}
           & Number of pixels in the current image's $y$ direction. \\
    && \\ \hline\hline
  \end{tabular}
\end{table}
\begin{table}
  \caption{User Variables -- Graphical Attributes}
  \mylabel{t:uservar3}
  \index{User Variables!table of}
  \index{User Variables!commands!{\tt set}}
  \index{Commands!{\tt set}}
  \index{Commands!{\tt angle}}
  \index{Commands!{\tt bgci}}
  \index{Commands!{\tt expand}}
  \index{Commands!{\tt color}}
  \index{Commands!{\tt fill}}
  \index{Commands!{\tt font}}
  \index{Commands!{\tt itf}}
  \index{Commands!{\tt lstyle}}
  \index{Commands!{\tt lwidth}}
  \index{Commands!{\tt show}}
  \index{Graphical Attributes}
  \index{User Variables!pre-defined variables!{\tt maxarray}}
  \centering
  \setlength{\tabentrylen}{\textwidth}
  \addtolength{\tabentrylen}{-1.6in}
  \begin{tabular}{|c|c|p{\tabentrylen}|} \hline\hline
    \multicolumn{1}{|c}{Variable}
    & \multicolumn{1}{|c|}{Command}
    & \multicolumn{1}{|c|}{Description} \\ \hline
    && \\
    angle  & {\tt angle}    & Current rotation angle value (in degrees). \\
    expand & {\tt expand}
           & Current character expansion value (character height units). \\
    color  & {\tt color}    & Current color index value. \\
    fill   & {\tt fill}     & Current fill type value. \\
    font   & {\tt font}     & Current font type. \\
    lstyle & {\tt lstyle}   & Current line style. \\
    lwidth & {\tt lwidth}   & Current line width. \\
    cmin   & {\tt device}   & Current device's minimum color index. \\
    cmax   & {\tt device}   & Current device's maximum color index. \\
    bgci   & {\tt bgci}
           & Value of argument to the most recent call to {\tt bgci}. \\
    itf    & {\tt itf}
           & Value of argument to the most recent call to {\tt itf}. \\
    cursmode & {\tt set}    & Value determines type of cursor drawn. \\
    hardcopy & {\tt device}
           & 1 if current device is hardcopy; 0 otherwise. \\
    maxarray & {\tt set}    & Initializes the maximum size of the
             {\bf X, Y, ERR}, and {\bf PSTYLE} arrays. \\
    nsig   & {\tt set}
           & Number of significant digits used to print a variable. \\
    pi     & (none)         & The value of $\pi$. \\
    \esc{0} -- \esc{20} & {\tt set} & Storage registers 0 -- 20. \\
    && \\ \hline\hline
  \end{tabular}
\end{table}

In addition,
Table~\ref{t:uservararray}
identifies the current predefined numeric user variable array (vector) names,
the maximum defined size of the array, and the typical commands used to load
the array.
Some of these entries have the maximum size set by the user variable
{\tt maxarray}\index{User Variables!pre-defined variables!{\tt maxarray}}.
This is initialized to 20,000 by \wip\ but may be overridden by setting
it to a larger or smaller number.
This must, however, be done right when \wip\ starts.
The best way to insure this is done correctly is to specify a different
value in the user's \wipinit\index{wipinit@\wipinit} file
(see Appendix~\ref{a:wipinit} for further details).
For example, to set this value to 10,000, the following command would be
put in the user's \wipinit\index{wipinit@\wipinit} file:
\begin{wiplist}%
  \index{Commands!{\tt set}}
  \index{User Variables!pre-defined variables!{\tt maxarray}}
  \item {\tt set maxarray 10000}\hfill\# Limits array size.
\end{wiplist}
In general, though, the default value should be sufficient for almost all
needs.
\begin{table}
  \caption{User Variables -- Numeric Arrays}
  \mylabel{t:uservararray}
  \index{User Variables!table of}
  \index{Commands!{\tt xcolumn}}
  \index{Commands!{\tt ycolumn}}
  \index{Commands!{\tt ecolumn}}
  \index{Commands!{\tt pcolumn}}
  \index{Commands!{\tt transfer}}
  \index{Commands!{\tt header}}
  \index{Commands!{\tt levels}}
  \index{Commands!{\tt autolevs}}
  \index{Commands!{\tt slev}}
  \index{Data Files!commands!{\tt xcolumn}}
  \index{Data Files!commands!{\tt ycolumn}}
  \index{Data Files!commands!{\tt ecolumn}}
  \index{Data Files!commands!{\tt pcolumn}}
  \index{Images!commands!{\tt header}}
  \index{Images!commands!{\tt transfer}}
  \index{Images!commands!{\tt levels}}
  \index{Images!commands!{\tt autolevs}}
  \index{Images!commands!{\tt slev}}
  \index{Contour Plots!setting levels}
  \index{User Variables!pre-defined variables!{\tt maxarray}}
  \centering
  \begin{tabular}{|c|c|c|} \hline\hline
    \multicolumn{1}{|c}{Variable}
    & \multicolumn{1}{|c}{Maximum Size}
    & \multicolumn{1}{|c|}{Command} \\ \hline
    && \\
    x        & {\tt maxarray} & {\tt xcolumn} \\
    y        & {\tt maxarray} & {\tt ycolumn} \\
    err      & {\tt maxarray} & {\tt ecolumn} \\
    pstyle   & {\tt maxarray} & {\tt pcolumn} \\
    transfer &     6 & {\tt transfer} {\tt header} \\
    levels   &    40 & {\tt levels} {\tt autolevs} {\tt slev} \\
    && \\ \hline\hline
  \end{tabular}
\end{table}

The string user variables may be fewer in number but are just as
powerful as the numeric user variables.
The current pre-defined string user variables are
listed in Table~\ref{t:userstrvar}
along with the primary command used to set each,
a description of each variable,
and the corresponding default string value.
\begin{table}
  \caption{String User Variables}
  \mylabel{t:userstrvar}
  \index{User Variables!table of}
  \index{User Variables!commands!{\tt set}}
  \index{User Variables!commands!{\tt string}}
  \index{String Variables!table of}
  \index{Commands!{\tt set}}
  \index{Commands!{\tt string}}
  \index{Commands!{\tt device}}
  \index{Commands!{\tt data}}
  \index{Commands!{\tt image}}
  \index{Commands!{\tt box}}
  \index{Commands!{\tt header}}
  \index{Commands!{\tt slev}}
  \index{Help!on-line help file!\$WIPHELP}
  \index{Help!on-line help file!the {\em helpfile} string}
  \index{Devices!device-type!/xwindow}
  \index{Devices!the {\em device} string}
  \index{Data Files!commands!{\tt data}}
  \index{Images!commands!{\tt image}}
  \index{Images!commands!{\tt header}}
  \index{Images!commands!{\tt slev}}
  \index{Boxes!commands!{\tt box}}
  \index{Boxes!the {\em xbox} string}
  \index{Boxes!the {\em ybox} string}
  \index{Contour Plots!setting levels!{\tt slev}}
  \centering
  \begin{tabular}{|c|c|l|c|} \hline\hline
    \multicolumn{1}{|c}{Variable}
    & \multicolumn{1}{|c|}{Command}
    & \multicolumn{1}{|c|}{Description}
    & \multicolumn{1}{|c|}{Default} \\ \hline
    &&& \\
    device    & {\tt device} & Name of the current device. & /xwindow \\
    datafile  & {\tt data}   & Name of the current data file. & empty \\
    imagefile & {\tt image}  & Name of the current image file. & empty \\
    helpfile  & {\tt set string} & Name of the on-line help file. & \$WIPHELP \\
    xbox      & {\tt set string} & $x$-axis argument for {\tt box}. & bcnst \\
    ybox      & {\tt set string} & $y$-axis argument for {\tt box}. & bcnstv \\
    print     & {\tt set string} & The print plot file command. & lpr \\
    xheader   & {\tt set string} & $x$-axis argument for {\tt header}. & rd \\
    yheader   & {\tt set string} & $y$-axis argument for {\tt header}. & rd \\
    levtype   & {\tt slev}   & Type of scaling for contour levels. & absolute \\
    &&& \\ \hline\hline
  \end{tabular}
\end{table}

\section		{Accessing the User Variables}
\mylabel{s:varset}
\index{User Variables!accessing}

This section describes the \wip\ commands that access and change the
value of the user variables.
Remember that there are two types of user variables:
numeric and string.
The primary command used to access all User Variables
is {\tt set}\index{Commands!{\tt set}}.
In addition, another command used to set the string variables
is called {\tt string}\index{Commands!{\tt string}}.
The command {\tt echo}\index{Commands!{\tt echo}}
can be used to display the value of many of the user variables
(both numeric and string) on the command line.
Also of use is the {\tt show}\index{Commands!{\tt show}} command.
The {\tt show} command presents many of the user variables
along with their current values.
The user variables are listed by {\tt show} in upper case along
with their current value.

\subsection*		{The SET Command}
\mylabel{ss:set}
\index{Commands!{\tt set}}
\index{User Variables!commands!{\tt set}}

The command {\tt set} requires at least two arguments.
The first (required) argument is an acceptable user variable name.
This may be any predefined user variable
(see Tables~\ref{t:uservar1} -- Table~\ref{t:userstrvar})
or any user variable defined with the
{\tt new}\index{Commands!{\tt new}}\index{User Variables!commands!{\tt new}}
command (see below).
The remaining arguments are treated 
differently depending on the type of user variable being set.
The next two sections discuss how {\tt set} treats (i) numeric and variable
array (vector) user variables and (ii) string user variables.

\subsubsection*		{Numeric User Variables}
\mylabel{sss:numbers}
\index{Commands!{\tt set}}
\index{User Variables!commands!{\tt set}}

When evaluating numeric user variables (both simple numeric and vector),
the first argument will be set
to the result of the expression consisting of the tokens appearing in
the rest of the command.
The expression is used to
determine what value will be assigned to the user variable.
The syntax of the {\tt set} command expression in this case is just a standard
mathematical expression with the following exceptions:
\begin{enumerate}
  \item The inner most expression enclosed by parenthesis
        is always evaluated first.
\samepage
  \item All expressions are evaluated left to right.
  \item All functions must have a set of parenthesis delimiting their argument.
  \item All operators that act on two variables (binary operators)
        must appear between their arguments.
\end{enumerate}
In many ways, the {\tt set} command is similar to a basic calculator.

The most straightforward use is to set the user variable equal to
something.
For example (and note the absence of the = sign),
\begin{wiplist}%
  \index{Commands!{\tt set}}
  \item {\tt set x1 0.1}\hfill\# Sets the user variable
    {\tt x1} to the value of 0.1.
\end{wiplist}

Table~\ref{t:twoop} identifies the operations that act on two variables.
These operators allow user variables to be added, subtracted, and so on.
They also allow two user variables (or expressions) to be compared with
the validity of the comparison as the result.
As mentioned already, there is no precedence of the various operators
except that of being evaluated from left to right.
This may be explicitly specified by enclosing various
operations within a matching pair of parentheses.
\begin{table}
  \caption{Double Variable Operations}
  \mylabel{t:twoop}
  \index{User Variables!operations!table of two variable}
  \centering
  \setlength{\tabentrylen}{\textwidth}
  \addtolength{\tabentrylen}{-1.0in}
  \begin{tabular}{|c|p{\tabentrylen}|} \hline\hline
    \multicolumn{1}{|c}{Binary} & \multicolumn{1}{|c|}{Result} \\ \hline
    & \\
    $+$          & Addition of the two arguments. \\
    $-$          & Subtraction of second from first argument. \\
    $*$          & Multiplication of the two arguments. \\
    $/$          & Division of first argument by second argument. \\
    $\backslash$ & Integer result of division of first by second argument. \\
    \%           & Remainder of first argument divided by second argument. \\
    max          & Larger value of the first and second argument. \\
    min          & Smaller value of the first and second argument. \\
    $**$         & First argument raised to the power of the second argument. \\
    & \\ \hline\hline
    \multicolumn{1}{|c}{Logical} &
    \multicolumn{1}{|c|}{Result is 1 only if the relationship is true;
    0 otherwise}
    \\ \hline
    & \\
    $>$  & True if the first argument is greater than the second. \\
    $<$  & True if the first argument is less than the second. \\
    $>=$ & True if the first argument is greater than or equal to the second. \\
    $<=$ & True if the first argument is less than or equal to the second. \\
    $==$ & True if the first argument is equal to the second. \\
    $!=$ & True if the first argument is not equal to the second. \\
    $\mid\mid$
         & True if either the first or second argument is not equal to zero. \\
    \&\& & True if both the first and second arguments are not equal to zero. \\
    \verb+^+
      & True if first or second argument (but not both) is not equal to zero. \\
    & \\ \hline\hline
  \end{tabular}
\end{table}
Note that the logical operations will only return a value
of 1 if the expression evaluated is true or a value of 0 if it is false.
Only these two values are returned
regardless of the value of the two arguments used to test the conditional.

Table~\ref{t:oneop} lists functions that operate on single variables.
Functions are identified by \wip\ when it sees a recognized function
name followed by parenthesis which enclose the variable.
The argument within the parenthesis is evaluated and then passed to the
function.
Hence, the expression within the parenthesis may be any valid expression
(even another function) as long as it evaluates to a single variable.
Note that functions appearing in the top part of Table~\ref{t:oneop}
require a {\em numeric value} as the argument to the function and
functions listed in the bottom part require a {\em vector name}.
\begin{table}
  \caption{Single Variable Functions}
  \mylabel{t:oneop}
  \index{User Variables!operations!table of one variable}
  \centering
  \begin{tabular}{|c|c|l|} \hline\hline
    \multicolumn{1}{|c}{Function} & \multicolumn{1}{|c}{Numeric Argument}
      & \multicolumn{1}{|c|}{Result} \\ \hline
    && \\
    sin($\theta$)  & ($\theta$) -- angle in radians  & $\sin(\theta)$ \\
    sind($\theta$) & ($\theta$) -- angle in degrees  & $\sin(\theta)$ \\
    asin($x$)      & $-1 \leq (x = \sin(\theta)) \leq 1$
           & Angle ($- \pi / 2 \leq \theta \leq \pi / 2$) in radians. \\
    asind($x$)     & $-1 \leq (x = \sin(\theta)) \leq 1$
           & Angle ($-90 \leq \theta \leq 90$) in degrees.            \\
    cos($\theta$)  & ($\theta$) -- angle in radians  & $\cos(\theta)$ \\
    cosd($\theta$) & ($\theta$) -- angle in degrees  & $\cos(\theta)$ \\
    acos($x$)      & $-1 \leq (x = \cos(\theta)) \leq 1$
           & Angle ($0 \leq \theta \leq \pi$) in radians.             \\
    acosd($x$)     & $-1 \leq (x = \cos(\theta)) \leq 1$
           & Angle ($0 \leq \theta \leq 180$) in degrees.             \\
    tan($\theta$)  & ($\theta$) -- angle in radians  & $\tan(\theta)$ \\
    tand($\theta$) & ($\theta$) -- angle in degrees  & $\tan(\theta)$ \\
    atan($x$)      & $x = \tan(\theta)$
           & Angle ($- \pi / 2 \leq \theta \leq \pi / 2$) in radians. \\
    atand($x$)     & $x = \tan(\theta)$
           & Angle ($-90 \leq \theta \leq 90$) in degrees. \\
    sqrt($x$)      & Any argument $x > 0$  & Non-negative square root.\\
    ln($x$)        & Any argument $x > 0$  & Base $e$ logarithm.      \\
    log($x$)       & Any argument $x > 0$  & Base 10 logarithm.       \\
    log10($x$)     & Any argument $x > 0$  & Base 10 logarithm.       \\
    exp($x$)       & Any argument $x < 85$ & Base $e$ exponent function.     \\
    abs($x$)       & Any argument $x$      & Absolute value of the argument. \\
    int($x$)       & Any argument $x$      & Integer value of the argument.  \\
    nint($x$)      & Any argument $x$      & Nearest integer of the argument.\\
    rand($x$)      & An integer seed $x$   & Uniform random deviate [0, 1). \\
    gasdev($x$)    & An integer seed $x$
           & Normal (Gaussian) random deviate [0, 1). \\
    && \\ \hline\hline
    \multicolumn{1}{|c}{Function} & \multicolumn{1}{|c}{Vector Argument}
      & \multicolumn{1}{|c|}{Result} \\ \hline
    && \\
    npts($x$)      & Any vector name $x$   & Current size of the named vector.\\
    && \\ \hline\hline
  \end{tabular}
\end{table}

One note about two of the functions listed in Table~\ref{t:oneop}.
The functions rand($x$) and gasdev($x$) use a negative integer argument
to help seed the random number generator functions.
By default, a seed is pre-supplied by \wip.
Hence, the user only needs to supply an argument of zero each time one
of these functions is called.
If the user wishes to start with a different seed value, then
the first time one of these functions is called, the user should
supply an odd, negative integer value.
On subsequent calls, the user should still supply a value of zero.
This is because the seed is retained internally and it is used if any
positive argument is passed to the function.

Note that an argument in the expression of the
{\tt set}\index{Commands!{\tt set}} syntax
can be easily made negative by placing the minus sign
before the user variable name.
For example, the commands
\begin{wiplist}%
  \index{Commands!{\tt set}}
  \item {\tt set \esc{0} -\esc{0}}
  \item {\tt set \esc{1} tand(-angle)}
  \item {\tt set \esc{2} 2 ** -x1}
\end{wiplist}
will set, respectively,
the user variable \esc{0} to the negative of its current value;
the user variable \esc{1} to the tangent of the negative of the current
angle variable;
and \esc{2} to the value of two raised to the negative of the current
value of {\tt x1}.

\subsubsection*		{String User Variables}
\mylabel{sss:strings}
\index{Commands!{\tt string}}
\index{User Variables!commands!{\tt string}}
\index{String Variables|(}

When the user variable is a string variable, the first argument to {\tt set}
is the string variable name and
the remaining arguments are treated as a string to be assigned to that variable.
This is the easiest way to set a string variable.
For example,
\begin{wiplist}%
  \index{Commands!{\tt set}}
  \item {\tt set print ignore}
\end{wiplist}
will set the string variable
{\em print}\index{String Variables!variables!{\em print}}
to the string {\tt ignore}.
This overrides the default value such that when a plot file
is ready to be spooled to the printer,
the spool command will not send the file to the printer.

The syntax of the {\tt set}\index{String Variables!commands!{\tt set}}
command is such that it requires at least
one argument: the string variable name.
If no other arguments are present,
the string variable is set to an empty string.
If other arguments are present, then the remaining arguments are
copied, exactly as entered, into the string variable.
Hence, string variables may contain several tokens (words) in its
translation.\footnote{The current maximum length for the value
of a string variable is 256 characters.}

The string variables in Table~\ref{t:userstrvar}
not directly associated with the 
{\tt set} command ({\em device}, {\em datafile}, {\em imagefile}, etc.)
are set by the listed commands and generally should not be changed with
the {\tt set} command.
The other string variables listed in Table~\ref{t:userstrvar}
are specifically designed to be accessed by the user
to allow \wip\ to be customized according to their needs.
The {\tt set} command changes the value of the variable until either
the end of the plotting session, until another {\tt set} command,
or until it is reset by a command.
Additionally, {\tt set} commands may be placed in the user's
\wipinit\index{wipinit@\wipinit} file (see Appendix~\ref{a:wipinit} for more
details) to always override \wip's initial values.

The {\tt string}\index{String Variables!commands!{\tt string}} command
is another method used to set string user variables.
The {\tt string} command allows the user to set the string variable
using the contents of an external file.
Only one line is read by the command but the user may specify which
words of that line are actually loaded into the string variable.
This command, along with its arguments, is discussed further
in Section~\ref{s:data} and Appendix~\ref{a:cmdname}.%
\index{String Variables|)}

\section		{Creating and Removing User Variables}
\mylabel{s:varnew}
\index{User Variables!defining new variables}
\index{User Variables!removing user variables}

New user variables may easily be defined with the
{\tt new}\index{Commands!{\tt new}}%
\index{User Variables!user defined variables}%
\index{User Variables!commands!{\tt new}}
command.
Likewise, the user defined variables may be removed with the
{\tt free}\index{Commands!{\tt free}}%
\index{User Variables!user defined variables}%
\index{User Variables!commands!{\tt free}}
command.
The ability to define user variables allows the user
to create local variables in macros,
retrieve and assign image header variables,
and associate a value with a variable name more in keeping with its use.
The only limitation on user defined user variables is that they may not
have the same name as any other user variable, vector, string variable,
or function.\footnote{Just like predefined commands,
no predefined user variable, vector, string variable,
or function may be removed.}

As an example of local usage of user variables, consider the following macro:
\begin{wiplist}%
  \index{Commands!{\tt define}}
  \index{Commands!{\tt new}}
  \index{Commands!{\tt set}}
  \index{Commands!{\tt if}}
  \index{Commands!{\tt move}}
  \index{Commands!{\tt free}}
  \index{Commands!{\tt end}}
  \index{User Variables!commands!{\tt new}}
  \index{User Variables!commands!{\tt free}}
  \item [\wipp] {\tt define radec}\hfill\# \$1--\$6=rah ram ras decd decm decs
  \item [\wipd] {\tt new ra dec}\hfill\# Create new user variables:
    ``ra'' and ``dec''.
  \item [\wipd] {\tt set ra (((\$1 * 60.0) + \$2) * 60.0) + \$3}
  \item [\wipd] {\tt set dec (((abs(\$4) * 60.0) + \$5) * 60.0) + \$6}
  \item [\wipd] {\tt if \$4 < 0 set dec -dec}
  \item [\wipd] {\tt move ra dec}\hfill\# Move to the position (ra, dec).
  \item [\wipd] {\tt free ra dec}\hfill\# Remove the user variables ra and dec.
  \item [\wipd] {\tt end}\hfill\# Finish the definition.
\end{wiplist}
This macro will convert a Right Ascension and Declination position
into units of seconds of time and seconds of arc and moves the current
position to that location.
By creating the user variables {\tt ra} and {\tt dec}
within the macro and then removing them prior to leaving it,
the user has limited the scope of these variables.
In addition, the user has also removed the possibility of inadvertently
altering their value by another macro.

The user should be aware, however, that defining user variables is a
global operation.\index{User Variables!lifetime of new variables}
Just because a variable is defined within a macro does not mean that
it is only seen by that macro or is active only while in that macro.
On the contrary, user defined variables exist in same way as predefined
user variables and do so until they are removed.
For this reason, care is needed when defining local variables.
In the example above,
another macro could define these same variables and then call the macro
{\tt radec}.
This would cause the macro {\tt radec} to abort when
the {\tt new} command is reached.
Therefore, make sure your local names are ``really'' local (\ie unique).

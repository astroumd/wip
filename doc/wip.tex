% LATEX document.
%
% HISTORY:
%   xx-xxx-88   rjs Original version
%   xx-jul-90   mjs 
%   19-feb-91   pjt assembled together from previous versions
%   22-feb-91   pjt included cookbook stuff
%    1-mar-91   pjt glossary and many more fixes
%    6-mar-91   pjt various user suggested improvements - added \newif's
%   17-apr-91   pjt final version for the Spring 91 Release
%   10-jun-91    jm Converted for use with WIP Version 1.0 (July 1991).
%   11-feb-92    jm Updated for WIP Version 1.5 (March 1992).
%   01-oct-92    jm Updated for WIP Version 1.6 (October 1992).
%   08-nov-93    jm Updated for WIP Version 1.7 (November 1993).
%   12-oct-94    jm Updated for WIP Version 1.8 (October 1994).
%   12-jun-95    jm Updated for WIP Version 2.0 (June 1995).
%   10-oct-96    jm Updated for WIP Version 2.1 (October 1996).
%   10-oct-00   pjt Updated for WIP Version 2.4 (October 2000).
%
\nonstopmode	             % Keeps LaTeX running despite certain errors....
%
%% Use a few TeX tricks to conditionally compile the manual.
%% Define variables {var} needed in this document as: \{var}true or \{var}false
%% DEBUG:   If true, margins contain index, as well as label stuff.
%% DOPLOT:  If true, implicitly insert PostScript commands for plots.
%%          doplottrue should only be used with Radical Eye Software's dvips!
%%  For final copy, all should be true (except debug).
%% These variables are all handled by wip.sty.
%
\newif\ifdebug  \debugfalse
\newif\ifdoplot \doplottrue
%
%%%%%%%%%%%%%%%%%%%%%%%%%%%%%%%%%%%%%%%%%%%%%%%%%%%%%%%%%%%%%%%%%%%%%%%%
%%
%% To print this LaTeX document (wip.tex):
%%
%%	latex name		(2 or 3 times to get page numbers right)
%%	makeindex name.idx 	(convert index file)
%%	latex name		(2 or 3 times, includes the .ind file)
%%	dvips name -o		(create the PostScript version)
%%
%%      Note: If you don't have makeindex, created a zero-length file
%%            named wip.ind or remove the makeidx option and \makeindex
%%            and \printindex from this document.
%%
%%                              James Morgan
%%
%%%%%%%%%%%%%%%%%%%%%%%%%%%%%%%%%%%%%%%%%%%%%%%%%%%%%%%%%%%%%%%%%%%%%%%%
%
%  Optional commands and style files needed:
%    wip.sty     : The main collection of WIP style modifications
%                : needed for this manual (a provided sty file).
%    twoside     : This option triggers two-side mode inside report.
%    makeidx     : This option puts the index at the end of the file
%                : (a generic LaTeX sty file).
%    comment     : This option permits creation of comment environments
%                : that can be ignored (a generic LaTeX sty file).
%    epsf        : This option permits insertion of PostScript figures
%                : (a misc LaTeX sty file).
%    html        : This option permits insertion of HTML links
%                : (a local LaTeX sty file).
%    report      : A generic LaTeX document style file.
%
%%%%%%%%%%%%%%%%%%%%%%%%%%%%%%%%%%%%%%%%%%%%%%%%%%%%%%%%%%%%%%%%%%%%%%%%
%
\documentstyle[twoside,makeidx,comment,epsf,html,wip]{report}
\makeindex
%
\title{\wip\\ User's Manual}
%
\author{James A. Morgan\\
        and \\
   \and
        Radio Astronomy Laboratory, University of California at Berkeley \\
   \and
        Laboratory for Astronomical Imaging, University of Illinois \\
   \and
        Laboratory for Millimeter Astronomy, University of Maryland
}
%
\date{Version 2.4 \\
   October 2000 \\
   Printed: \today
}
%
%%%%%%%%%%%%%%%%%%%%%%%%%%%%%%%%%%%%%%%%%%%%%%%%%%%%%%%%%%%%%%%%%%%%%%%%%%%
%
\setlength{\parindent}{0pt}
\setlength{\parskip}{2.5mm}
%
\addtolength{\textwidth}{1.5in}
\addtolength{\evensidemargin}{-0.75in}
\addtolength{\oddsidemargin}{-0.75in}
%
%  The following macro provides a way to insert figures and
%  captions in a uniform way.  The macro insertplot requires
%  five arguments:
%  \insertplot{Section Title}{PS file root}{caption}%
%     {data file title}{verbatim data file name}
%  The last two arguments can be empty.  The WIP include file
%  should have the same name as the PostScript file except for
%  a different suffix.
%
%  I have included this definition here instead of inside wip.sty
%  because I also want latex2html to be able to access it early
%  (because of the \section commands) in the conversion process.
%
\newcommand{\insertplot}[5]{
  \section{#1}
  \small
  \fileverbatim{Examples/#2.wip}
  \normalsize
  \subsection*{#4}
  \footnotesize
  \fileverbatim{#5}
  \normalsize
  \begin{figure}[p]
    \wipplot{#2}
    \caption{#3}
    \mylabel{p:#2}
  \end{figure}
}

\begin{document}
\pagenumbering{roman}
%
\maketitle            % Make the title page.
%
%  Backside of cover page is a disclaimer (in smaller roman font).
%
\newpage
\pagestyle{empty}
\footnotesize\rm
\input disclaimer
\normalsize
%
%%%%%%%%%%%%%%%%%%%%%%%%%%%%%%%%%%%%%%%%%%%%%%%%%%%%%%%%%%%%%%%%%%%%%%%%
%  The various content pages.
\cleardoublepage
\pagestyle{headings}
\addcontentsline{toc}{chapter}{Table of Contents}
\tableofcontents
%
\newpage
\addcontentsline{toc}{chapter}{List of Tables}
\listoftables
%
\newpage
\addcontentsline{toc}{chapter}{List of Figures}
\listoffigures
%
%%%%%%%%%%%%%%%%%%%%%%%%%%%%%%%%%%%%%%%%%%%%%%%%%%%%%%%%%%%%%%%%%%%%%%%%
%  The start of the main part of the manual.
\cleardoublepage
\pagenumbering{arabic}
\part{General Concepts}\mylabel{p:intro}
%
\chapter{How to Use This Manual}
\input howto
%
\chapter{The User Interface}
\input iface
%
\chapter{General Graphics Concepts}
\input concepts
%
\chapter{Basic Plotting}
\input basic
%
\chapter{Images}
\index{Images|(}
\input images
\index{Images|)}
%
%%%%%%%%%%%%%%%%%%%%%%%%%%%%%%%%%%%%%%%%%%%%%%%%%%%%%%%%%%%%%%%%%%%%%%%%
\part{Advanced Concepts}\mylabel{p:advanced}
%
\chapter{Data Fitting}
\index{Fitting|(}
\input fit
\index{Fitting|)}
%
\chapter{Macros}
\index{Macros|(}
\input macros
\index{Macros|)}
%
\chapter{User Variables}
\index{User Variables|(}
\input uservar
\index{User Variables|)}
%
\chapter{Control Flow}
\input flow
%
%%%%%%%%%%%%%%%%%%%%%%%%%%%%%%%%%%%%%%%%%%%%%%%%%%%%%%%%%%%%%%%%%%%%%%%%
\appendix
\part{Appendices}\mylabel{p:appendix}
%
\chapter{The \wipinit File}
\index{wipinit@\wipinit|(}
\input wipinit
\index{wipinit@\wipinit|)}
%
\chapter{Command Line Interface}
\input cmdline
%
\chapter{Glossary}
\input glossary
%
\chapter{Table of Current Command Names}
\input cmdname
%
\prefigs
\chapter{Sample Plots}
\input samples
\postfigs
%
\chapter{Frequently Asked Questions}
\input faq
%
%%%%%%%%%%%%%%%%%%%%%%%%%%%%%%%%%%%%%%%%%%%%%%%%%%%%%%%%%%%%%%%%%%%%%%%%
%  End of the document.  The rest is index stuff.
%
\addcontentsline{toc}{chapter}{Index}
\printindex
%
\typeout{### Type: makeindex wip.idx}
\typeout{###   To create the .ind file from the .idx file}
\typeout{###   in case the index file needs to be regenerated}
%
\end{document}

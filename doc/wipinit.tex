%% This file is to be included by latex in wip.tex
%
\mylabel{a:wipinit}
\myfile{wipinit.tex}

\wip\ has set up most of the default attributes so that the user should
not have to ``worry'' about them until they want to alter them to
enhance their plot.
There may be a time, however, when a user will want to customize
characteristics a certain way.
For example, suppose a user typically works with
a tektronix device\index{Devices}\index{Devices!device-type!/tek}.
The hard coded device type is
the {\tt /xwindow}\index{Devices!device-type!/xwindow}.
To override this default device, the user could start \wip\ with
the command\index{Command Line}\index{Command Line!switches}
\begin{wiplist}%
  \item[\%] {\tt wip -d /tek}
\end{wiplist}
every time they use \wip.
Obviously, a simpler solution is desired.
Placing the line
\begin{wiplist}%
  \index{Commands!{\tt set}}
  \item[\ ] {\tt set device /tek}
\end{wiplist}
in the file \wipinit in the user's home directory
will cause \wip\ to always act as if the hard coded initial device was a
tektronix terminal.\footnote{This default device may still be overridden
with the command line option {\em -d~device-type}
(see Appendix~\ref{a:cmdline}).}

An another example, suppose a user typically displays images
with the $x$-axis representing Right Ascension and the $y$-axis Declination.
Then, this user might like to have a call to
{\tt box}\index{Commands!{\tt box}}\index{Boxes!commands!{\tt box}}
without any arguments to correctly label the $x$-axis in hours, minutes, and
seconds and the $y$-axis in degrees, minutes, and seconds.
By default, \wip\ defines
{\em xbox}\index{String Variables!variables!{\em xbox}}
as the string ``BCNST'' and
{\em ybox}\index{String Variables!variables!{\em ybox}}
as ``BCNSTV''
(see Table~\ref{t:userstrvar} for an explanation of these characters).
One alternative might be that
the user would specify the string\index{String Variables} variables
{\em xbox}\index{String Variables!variables!{\em xbox}} and
{\em ybox}\index{String Variables!variables!{\em ybox}}
\begin{wiplist}%
  \index{Commands!{\tt set}}
  \item {\tt set xbox bcnsthz}
\samepage
  \item {\tt set ybox bcnstvdyz}
\end{wiplist}
whenever they entered \wip.
A better alternative, however, would be for the user to specify
these string variables in their \wipinit file as:
\begin{wiplist}%
  \index{Commands!{\tt set}}
  \item[\ ] {\tt set xbox bcnsthz}
\samepage
  \item[\ ] {\tt set ybox bcnstvdyz}.
\end{wiplist}
Then, whenever \wip\ is started, these values will become the
default values for the string variables {\em xbox} and {\em ybox}.

In addition to setting these variables, a user may also
want to
automatically load files containing
macro\index{Macros}\index{Commands!{\tt macro}}%
\index{Macros!commands!{\tt macro}}
definitions
or set other user variables\index{User Variables}\index{Commands!{\tt set}}
to certain values whenever \wip\ is
started.
A user should use caution when using
the {\tt set} command in their \wipinit file:
calculations made with variables established by the device
(\eg X1, VY1, FILL, etc.) are undefined at this
point and, hence, so are results based on their values.
For this reason, the command {\tt set} should generally be used
cautiously in the user's \wipinit file except for string variables and
the most rudimentary of calculations.

As an example of how to define macros automatically,
suppose a user has a file named {\tt mymacro.wip} that
contains a macro definition that is used often.
Placing the line
\begin{wiplist}%
  \index{Commands!{\tt macro}}
  \item[\ ] {\tt macro mymacro.wip}
\end{wiplist}
in the user's \wipinit file will automatically load the
macro definitions from the named file every time \wip\ is started.
Remember everything in the named file but macro definitions
will be ignored by the {\tt macro} command.

\begin{table}
  \caption{Acceptable commands in the \wipinit File}
  \mylabel{t:wipinitem}
  \index{Commands!{\tt set}}
  \index{Macros!commands!{\tt macro}}
  \index{Macros}
  \index{Commands!{\tt macro}}
  \index{User Variables}
  \index{String Variables}
  \index{String Variables!variables!{\em print}}
  \index{String Variables!variables!{\em xbox}}
  \index{String Variables!variables!{\em ybox}}
  \index{String Variables!variables!{\em xheader}}
  \index{String Variables!variables!{\em yheader}}
  \index{String Variables!variables!{\em levtype}}
  \index{String Variables!variables!{\em device}}
  \index{String Variables!variables!{\em helpfile}}
  \index{Devices!device-type!/xwindow}
  \index{Devices!the {\em device} string}
  \index{Hardcopy!the {\em print} string}
  \index{Boxes!the {\em xbox} string}
  \index{Boxes!the {\em ybox} string}
  \index{Images!header files}
  \index{Images!setting contour levels}
  \index{Help!on-line help file!the {\em helpfile} string}
  \index{Help!on-line help file!\$WIPHELP}
  \index{Commands!{\tt new}}
  \index{Commands!{\tt free}}
  \index{User Variables!commands!{\tt new}}
  \index{User Variables!commands!{\tt free}}
  \index{User Variables!user defined variables}
  \centering
  \setlength{\tabentrylen}{\textwidth}
  \addtolength{\tabentrylen}{-2.0in}
  \begin{tabular}{|l|p{\tabentrylen}|l|} \hline\hline
      \multicolumn{1}{|c}{Command} & \multicolumn{1}{|c}{Required Argument}
     & \multicolumn{1}{|c|}{Default value} \\ \hline
    && \\
    set device   & ``device-name/device-type''      & /xwindow \\
    set xbox     & Box options (Table~\ref{t:box}) for the X-axis & BCNST \\
    set ybox     & Box options (Table~\ref{t:box}) for the Y-axis & BCNSTV \\
    set print    & Print command (System dependent) & {\tt lpr} \\
    set xheader  & X-axis header options (Section~\ref{ss:header}) & rd \\
    set yheader  & Y-axis header options (Section~\ref{ss:header}) & rd \\
    set levtype  & Contour scaling type (Section~\ref{ss:levels}) & absolute \\
    set helpfile & The on-line help file (System dependent)       & \$WIPHELP \\
    set maxarray & The maximum size of the {\bf X, Y, ERR},
                   and {\bf PSTYLE} arrays.
                   This must be set in the \wipinit file.         & 20,000 \\
    set   & Set options (see Chapter~\ref{c:uservar})             & (none) \\
    macro & Macro file name                                       & (none) \\
    new   & New user variable name (Section~\ref{s:varnew})       & (none) \\
    free  & User variable name to remove (Section~\ref{s:varnew}) & (none) \\
    && \\ \hline\hline
  \end{tabular}
\end{table}

Table~\ref{t:wipinitem} lists the commands and string variables
that may be specified in your \wipinit file.
Note that the user variable {\tt maxarray} can only be set in
the \wipinit file.
This is because once these arrays are initialized, they can not have
their maximum size changed.
In addition, comments\index{Comments} may be placed anywhere in
the \wipinit file as long as they begin with the comment character
(\#).\index{Comments!comment character(\#)}
For a further explanation of how to define macros, see
Chapter~\ref{c:macros}.
The command {\tt set}\index{Commands!{\tt set}}
and user variables\index{User Variables}
are discussed more fully in Chapter~\ref{c:uservar}.

A typical \wipinit file might contain the following lines:
\begin{wiplist}%
  \index{Commands!{\tt set}}
  \index{Commands!{\tt macro}}
  \index{User Variables!commands!{\tt set}}
  \index{Devices!device-type!/xserve}
  \index{Devices!the {\em device} string}
  \item[\ ] {\tt set device   /xserve}
\samepage
  \item[\ ] {\tt set xbox     BCNSTHZ}
  \item[\ ] {\tt set ybox     BCNSTVDYZ}
  \item[\ ] {\tt set print    lp -d ps -w}
  \item[\ ] {\tt macro /myroot/subdir/macro1.wip}
  \item[\ ] {\tt macro /myroot/macro2.wip}
\end{wiplist}
where the user has two files containing macros
({\tt /myroot/subdir/macro1.wip} and /{\tt myroot/macro2.wip})
that are always loaded whenever \wip\ begins.
In addition, this example illustrates how to set the default graphics
device to the {\tt /xserve} X window driver;
change the default arguments for the {\tt box} command;
and set the print command to send the plot (under Solaris) to the
printer named {\tt ps} with a notification to the user after
it is printed.

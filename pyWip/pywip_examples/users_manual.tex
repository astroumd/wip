\documentclass[12pt]{article}
\title{PyWIP User's Guide}
\author{Nicholas Chapman}
\date{6 April 2010}
\newcommand{\pywip}{PyWIP}
\begin{document}

\maketitle
\tableofcontents
\newpage

\section{Introduction}

Around the time I was supposed to be writing my thesis, I decided I was fed-up
with a plotting package called WIP.  In Astronomy, people generally use
IDL, SM, WIP, or Gnuplot for their plotting needs.  The first two cost money,
a lot of money in the case of IDL.  WIP has two main advantages over Gnuplot:
it can read and understand FITS/MIRIAD data files with world coordinates, and
I know how to use it.  In most other ways WIP sucks.  Big time.

So, like I said, I was supposed to be writing my thesis, which was going to
require a ton of figures, and the thought of doing it all in WIP gave me
nightmares.  Right then and there, I decided it would be much easier to write
a python wrapper around WIP to make it much easier to use.  Thus, \pywip{} was
born.

The only other python plotting package I know of is matplotlib.  I had played
with that off and on, but was ultimately unsatisfied by it.  It seemed to
require too much overhead to make any sort of plot.  Furthermore, it took major
effort and more overhead to do some simple things like changing the tick marks
on the axes.  Lastly, matplotlib seems to run very slowly, even on the simplest
of plots.  But, your mileage may vary.  If you hate \pywip{}, give it a try.
matplotlib can do many things that \pywip{} (and WIP) cannot.  However, I do not
need any of those features.

\section{Installation}

\pywip{} is easy to install.  You only need two things, WIP and python.  You
likely already have python installed.  If you have a semi-recent version,
greater than say, around 2.4, that should be fine.  WIP comes from installing
MIRIAD.

The most convenient way to install \pywip{} is to put it in some permanent
location and make that location your PYTHONPATH.  Then you won't need to make a
copy of pywip.py in every directory where you are plotting.  For this to work
you need to set the PYTHONPATH environment variable, e.g.:\\ \texttt{setenv
PYTHONPATH /Users/nchapman/data1/Applications/nlcpython}

\section{Documentation}

In addition to this guide, \pywip{} comes with numerous examples in the same
directory as this document. You should be able to run them all and get output
postscript plots.  Furthermore, you can find an HTML guide that lists all the
commands with brief descriptions of their keywords in the same directory as
pywip.py.  If this HTML document is missing you can create it easily:\\
\texttt{pydoc -w pywip}

\section{Individual Commands}

\subsection{axis}

If you issue the \texttt{axis()} command without any arguments, then \pywip{}
will attempt to guess the appropriate axis to draw.  This means if you set
the \texttt{logx} or \texttt{logy} keywords in any of your plots, then the
appropriate x- or y-axis will be labelled logarithmically.  Furthermore,
if you plotted an image with \texttt{header=`rd'}, then \pywip{} will label
the axes with world coordinates.  Finally, when you are plotting multiple
panels that join, \pywip{} will be smart enough to suppress number labelling
where it would interfere with an adajacent panel.

This command isn't perfect, but it's good enough for Rock-and-Roll.  Give it a
whirl, and if it doesn't do it the way you like, then you can fall back on
specifying some or of all the parameters manually as needed.  Table
\ref{tab:axis} lists the allowed parameters.

\begin{table}
\caption{\label{tab:axis} Keywords for the \texttt{axis()} command}
\begin{center}
\begin{tabular}{lp{4in}}
\hline
\hline
\multicolumn{1}{c}{Keyword} & \multicolumn{1}{c}{Meaning}\\
\hline
box       & Specifies the frame to be drawn.  Defaults to
            \mbox{(`bottom',`left',`top',`right')}.\\
number    & Where to put the numbers.  Defaults to (`bottom',`left')
            but \texttt{axis()} will attempt to change these
            depending if you have multiple panels.  The first string
            can be either `bottom' or `top' (for x) and the second
            one can be either `left' or `right' (for y).  If you
            don't want any numbers, set this to ().\\
tickstyle & Style for tickmarks.  Note, to control whether tickmarks
            are actually shown, see the \texttt{subtick} and \texttt{majortick}
            keywords below.  The default style is (`inside',`inside'), meaning
            the tickmarks for (x,y) axes are drawn inside of the
            frame.  Either string can be set to `inside', `outside',
            or 'both'.\\
format    & Format for the numbers.  The default is (`auto',`auto')
            for the (x,y) axes.  If you are plotting an image, these
            are automatically changed to (`wcs',`wcs').  Generally, you should
            never have to set this, but the other options are `dec'
            to force decimal labeling, and `exp' to force exponential
            labeling.\\
xinterval & A tuple of two numbers, where the first number is the
            distance between major xtick intervals, and the second
            number is the number of subticks between each major xtick.
            If you leave this unset, reasonable defaults will be
            chosen.\\
yinterval & A tuple of two numbers, where the first number is the
            distance between major ytick intervals, and the second
            number is the number of subticks between each major ytick.
            If you leave this unset, reasonable defaults will be
            chosen.\\
\hline
\end{tabular}
\end{center}
\end{table}

\setcounter{table}{0} % need to have a two-page table

\begin{table}
\caption{Keywords for the \texttt{axis()} command (continued)}
\begin{center}
\begin{tabular}{p{1in}p{4in}}
\hline
\hline
\multicolumn{1}{c}{Keyword} & \multicolumn{1}{c}{Meaning}\\
\hline
drawtickx, drawticky   & If set to False, this is a shortcut to setting
                         \texttt{subtickx=False} and \texttt{majortickx=False}.
                         Note, setting \texttt{drawtick=False} while also
                         setting the \texttt{majortick} or \texttt{subtick}
                         keywords to True may cause weird behavior.  You have
                         been warned. \\
firstx, firsty         & If set to False, write only the last part of the label
                         for the first time tick on the axis.  For example, if
                         the full first label is 17 42 34.4, then only write
                         34.4. This option is useful for sub-panels that join
                         each other. The default values are True.\\
gridx, gridy           & Draw a grid of lines at the major tick positions for
                         either x or y.  Defaults are False, meaning no grid is
                         shown.\\
logx, logy             & Label x or y axis logaritmically.  Defaults are set
                         automatically depending on whether the user
                         has specified log labeling with plot() or some other
                         command. Note that setting this parameter will override
                         any settings for xinterval and yinterval.\\
subtickx, subticky     & Draw subticks?  Defaults to True, meaning do draw
                         subticks for both the x and y axes.\\
majortickx, majorticky & Draw majorticks? Defaults to True, meaning do draw
                         major ticks for both the x and y axes.\\
verticaly              & Rotate the numbers on the y axis to be vertical?  The
                         default is False, but it is set to True automatically
                         if an image has been plotted.  There is not equivalent
                         command for the xaxis numbers.\\
zerox, zeroy           & If set to False, omit leading zero in numbers $< 10$
                         in time labels.  This option is only valid when format
                         is `wcs' for the given axis.  Furthermore, it is
                         ignored when verticaly=True because it becomes
                         impossible to align the numbers properly.  The default
                         values are True.\\
\hline
\end{tabular}
\end{center}
\end{table}

\subsection{blowup}

This command is somewhat tricky.  The input coordinates MUST be pixel values. I
usually determine them from DS9.  Secondly,  if you have plotted only a
sub-region of an image with halftone/contour, then you need to input the
relative pixel values.  For example, I plotted the image
'image.fits[172:194,136:158]'.  Later, I wanted to draw a blow-up box determined
by the pixel coordinates 181.156,183.018,145.567,146.802 (in the xmin, xmax,
ymin, ymax format).  However, I could not enter these values for blowup()
because they are incorrect.  The correct values would be: 10.156, 12.018,
10.567, 11.802.  (181.156 - 171, 183.018 - 171, 145.567 - 135, 146.802 - 135).
I subtracted 171 and 135 for x and y respectively, because pixels in FITS images
are counted from 1 not zero.

\subsection{legend}

The legend command is used to make it easier to put legends on your plots. A
legend is essentially a series of rows, where each row has two elements: a
symbol or line representing the data set or curve and a text label. After vast
amounts of work, I think I finally succeeded in making a general purpose
\texttt{legend()} command that doesn't require lots of fiddling by the user.
Just give it x and y coordinates for the upper-left corner (given as a fraction
of the plot width and height) and you should be good to go. Maybe someday I will
get around to adding the option to have a box drawn around the legend
automatically. The only caveat with \texttt{legend()} is that it should be the
last thing plotted (well, I think \texttt{xlabel},\texttt{ylabel}, and
\texttt{title()} are okay).

Also note, the curves listed by \texttt{legend()} are accumulated for all
panels as plots are made until a \texttt{legend()} command is issued.  Then
the list of plots is reset.  If you don't want a particular plot shown in the
legend, just set the \texttt{Text=None keyword}.

\subsection{vector}

The \texttt{vector} command is used to draw arrow fields.  So that the angles
and lengths will always make sense when plotting arrows on a FITS image with
WCS, internally \texttt{vector()} will reset the header to `px' (absolute pixel
value).  What this means for the user is that \texttt{vector()} should be the
last thing added to a plot (well, maybe before \texttt{legend()}, I'm not sure
about that).  Also, it means that you should always specify the x and y as
pixels, not wcs coordinates.  Otherwise, your postions, lengths, and angles will
be all wrong.  Finally, you can use the \texttt{align} keyword to  specify
whether the input coordinates are the left (end), center, or right (tip) of the
arrows.

\section{Keywords}

\subsection{color,font,size,style,width}

A number of the commands have some or all of these keywords as options.  If
some or all of these aren't specified, the default values for these will be
used.  The default color is black, font is roman, size is 1, style is
solid lines, and width is 1.  You can change the defaults at any time by
running the \texttt{default} command.  See Tables \ref{tab:colors},
\ref{tab:fonts}, and \ref{tab:lstyles} for details.

\begin{table}
\caption{\label{tab:colors} Valid Colors}
\begin{center}
\begin{tabular}{cl}
\hline
\hline
\multicolumn{1}{c}{Key} & \multicolumn{1}{c}{Description}\\
\hline
w  & white\\
k  & black (default)\\
r  & red\\
g  & green\\
b  & blue\\
c  & cyan\\
m  & magenta\\
y  & yellow\\
o  & orange\\
gy & green-yellow\\
gc & green-cyan\\
bc & blue-cyan\\
bm & blue-magenta\\
rm & red-magenta\\
dg & dark gray\\
lg & light gray\\
\hline
\end{tabular}
\end{center}
\end{table}

\begin{table}
\caption{\label{tab:fonts} Valid Fonts}
\begin{center}
\begin{tabular}{cl}
\hline
\hline
\multicolumn{1}{c}{Key} & \multicolumn{1}{c}{Description}\\
\hline
sf & sans-serif\\
rm & roman (default)\\
it & italics\\
cu & cursive\\
\hline
\end{tabular}
\end{center}
\end{table}

\begin{table}
\caption{\label{tab:lstyles} Line Styles}
\begin{center}
\begin{tabular}{cl}
\hline
\hline
\multicolumn{1}{c}{Key} & \multicolumn{1}{c}{Description}\\
\hline
  -  & solid (default)\\
 \texttt{--}  & dashed\\
 .-  & dot-dashed\\
  :  & dotted\\
-... & dash-dot-dot-dot\\
\hline
\end{tabular}
\end{center}
\end{table}

\subsection{image}

The \texttt{contour}, \texttt{halftone}, and \texttt{winadj} commands all have
the \texttt{image} keyword.  \texttt{image} is a string specifying the
FITS/MIRIAD image name, and optionally the sub-image to plot.  For example:\\
\texttt{image='Bob.fits'} Plot the entire image `Bob.fits'\\
\texttt{image='Bob.fits[10:20,5:15]'} Plot `Bob.fits' but only the subimage of
pixels 10-20 in the x and 5-15 in the y.  Finally, \texttt{image='Bob.fits(1)'}
will plot plane 1 from `Bob.fits'.  You can combine the subimage and plane
definitions in either order.

There are several other `gotchas' to watch out for when plotting images.
See \S\,\ref{sec:futz-images} for more details on these.

\subsection{limits}

The limits keyword can be have several different values.  First, if you do not
specify limits, then \pywip{} will attempt to guess what the limits should be.
If no limits have been set for the current panel, then \pywip{} will use the 
min/max of the dataset for the current limits.  Otherwise, the currently
existing limits will be reused.  Second, you can give a list/tuple of four
values, which will become the new limits and override any previously existing
ones.  Third, you can set \texttt{limits='last'}, meaning \pywip{} should
use the last known limits (generally from the previous panel) as the new limits.
Note, this will also carry over settings for \texttt{'logx'} and 
\texttt{'logy'}.

The fourth, and last possibility is to explicitly set \texttt{'limits=None'}. I
haven't fully explored the ramifications of this, but it seems to force \pywip{}
to set new limits based on the min/max of the dataset or image, even if limits
were already known for the current panel. I use it in \texttt{example11.py} to
ensure that the blowup box is not resized to the limits of the entire region. I
think it would work for non-image type plots as well.

\subsection{x,y}

Several commands have the \texttt{x} and \texttt{y} keywords to specify a
location on the plot.  This can be given as a number, but it may be more
convenient to specify a string in the case of world coordinates.  For example,
\texttt{x='3:30'} would move the cursor to 3 hours 30 minutes.  Similarly,
\texttt{x='52.5'} would also move the cursor to 3 hours 30 minutes since 52.5
degrees is the equivalent.  Note the importance of giving the value as a string.
Without the quotes, it will move the cursor to 52.5 hour-seconds, not 52.5
degrees, nor 3 hours 30 minutes.  \pywip{} assumes that \texttt{x} is for Right
Ascension and \texttt{y} is for Declination.

\section{Tips \& Tricks}

\subsection{Using hidden variables}

Say you want to make a series of plots in a for loop, and have each plot be a
different color.  You don't need to manually type in all the colors.  Instead,
they are part of a hidden variable that can be imported.  Hidden variables
the user may want to access are listed in Table \ref{tab:hidden}.  They
can be imported and used like so:\\
\texttt{from pywip import \_colors as pycolors}

\begin{table}
\caption{\label{tab:hidden} Hidden Variables}
\begin{center}
\begin{tabular}{ll}
\hline
\hline
\multicolumn{1}{c}{Variable} & \multicolumn{1}{c}{Meaning}\\
\hline
\_palettes & The palettes used for the \texttt{halftone} command\\
\_colors   & The valid colors for fonts, symbols, lines, etc. (see Table
             \ref{tab:colors})\\
\_fills    & The available fill styles\\
\_fonts    & The available fonts (See Table \ref{tab:fonts})\\
\_lstyles  & The available line styles (Table \ref{tab:lstyles})\\
\_symbols  & The available symbols to plot points\\
\hline
\end{tabular}
\end{center}
\end{table}

\subsection{Multi-Variable Axes}

To plot one variable (with numbers and tickmarks) on the bottom and a second
variable  on the top (with a different set of numbers and tickmarks), you
have to use the \texttt{style=None} keyword in the \texttt{plot()} command.
This generates a phantom plot.  \pywip{} goes through the motions of setting
limits, but won't actually plot any points.  Then, just plot the new axis:
\texttt{axis(box=['top'],number=['top'])}.  You will also have to change the 
axis command for the firt plot so that it doesn't interfere with the second
\texttt{axis()} command, e.g.\ \texttt{axis(box=['bottom','left','right'])}.

I did this, and it seems like it would work, so long as variable two is 
directly proportional to variable1.  For example, $var2 = 3\times var1$ should
be okay.  However, I had a case where $var2 = 1/var1$, and that did NOT work.
I think it is a problem with WIP however.

\subsection{\label{sec:futz-images}Futzing with images}

WIP is particular about the images it will accept. Images in \pywip{} will
always have x increasing to the right and y increasing upwards.  If you are
plotting an image with a world coordinate system (WCS), it is up to you to make
sure the image is rotated properly.  In astronomy you typically want Declination
upwards and Right Ascension increasing to the left.  You may need to rotate your
image with IRAF's \texttt{rotate} command or the \texttt{regrid} task in Miriad
to achieve the correct rotation.  If you neglect to do this, \pywip{} will still
do something, but the WCS will probably be wrong.

Secondly, at least one world coordinate projection (gls) is not allowed.  If
you get an error about ``unsupported projection'' then you may need to fix your
image.  This can be done with the Miriad task \texttt{regrid}.  Thanks to 
Jorge Pineda for this tip:
\begin{verbatim}
Assume you have file.fits
fits in=file.fits op=xyin out=file.mir
regrid in=file.mir out=file_tan.mir project=tan
fits in=file_tan.mir op=xyout out=file_tan.fits
\end{verbatim}

Lastly, for very large fields, I think the WCS displayed by WIP will be off. 
This can easily be seen if you plot the ra/dec, then plot their equivalent x/y
positions on an image (get the x/y through something like \texttt{sky2xy} or
\texttt{ds9}). They will \emph{not} line up.

\subsection{World Coordinates from a data file}

When reading from files, WIP needs ra/dec to be in hour-seconds and 
degree-seconds (ra * 240, when ra is in degrees, and dec * 3600 when dec is
in degrees).

\section{Limitations}

This is my current list of the limitations of \pywip.
\begin{itemize}
\item The colors should allow hexcodes, grayscale ranges, and rgb values
\item The axis() command should control logx, logy, and plot range limits
\item Cannot make shaded contour plots
\item Cannot make transparent fill styles
\item Only a limited number of filled symbols are available
\item The only output formats are .ps and .gif.
\item wedge() should probably be renamed to colorbar()
\item The legend() command is quite nice, but is still a kludge
\item No ability to make RGB plots like DS9
\item WCS projections do not seem correct for large regions
\item No way to vertically align text (in a left, center, right format)
\item Doesn't understand numpy arrays
\end{itemize}

\end{document}
